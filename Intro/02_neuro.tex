% #############################################################################
%                        PRESENTATION DE LA THEMATIQUE
% #############################################################################
\chapter{ICM et neurophysiologie}

% -----------------------------------------------------------------------------
% -----------------------------------------------------------------------------
%                            BASES PHYSIOLOGIQUES
% -----------------------------------------------------------------------------
% -----------------------------------------------------------------------------
\section{Bases physiologiques}
\figScaleX{0.6}{Grainmann_2002_Brain}{Blabla \citep{graimann_braincomputer_2009}}

\section{Apprentissage machine: applications aux neurosciences}
okok

% -----------------------------------------------------------------------------
\section{Encodage et d�codage moteur: bases physiologiques}
Pomper les r�sultas de rodrigo + besserve + inclure sch�ma global du cerveau

% -----------------------------------------------------------------------------
% -----------------------------------------------------------------------------
%                        MARQUEURS DE L'AN
% -----------------------------------------------------------------------------
% -----------------------------------------------------------------------------
\section{Signaux physiologiques pour le contr�le d'une ICM}
\label{sec_marqueurs_AN}

PE, ME, ERD, ERS
Les signaux peuvent class�s en deux cat�gories \citep{wolpaw_brain_2002, curran_learning_2003, pfurtscheller_rehabilitation_2008}:
\begin{itemize}
	\item \textbf{Les signaux �voqu�s}: sont produits, sans que le sujet en ai conscience, suite � un stimulus externe pouvant �tre tactile, auditif ou visuel. En \eeg, on parlera de \textit{Potentiel �voqu�} (PE) et en \meg, \textit{Champs magn�tique �voqu�}
	\item \textbf{Signaux spontan�s}: ceux-ci peuvent �tre volontairement modifi�s par l'utilisateur.
\end{itemize}

% ********************************************
%         Signaux �voqu�s
% ********************************************
\subsection{Signaux �voqu�s}

% -> P300
\subsubsection{P300}

% -> SSEP
\subsubsection{SSVEP}

% ********************************************
%         Signaux spontan�s
% ********************************************
\subsection{Signaux spontan�s}

% -> Rythmes sensorymoteur
\subsubsection{Rythmes sensorimoteurs (RSM)}

% -> Onde lente
\subsubsection{Ondes corticalee lentes}

% -> Spike
\subsubsection{Activit� \textit{spike}}


% //////////////////////////////////////
% POUR LES LIS ET CLIS
\citep{chaudhary_brain-machine_2015}

\citep{birbaumer_breaking_2006} conclue que:
- SMR et P300 donnent de meilleurs r�sultats que SCP-BCI chez les LIS mais les SCP semblent plus stables, et moins d�pendantes des fonctions cognitives, motrices et sensorimotrices, ce qui repr�sente un avantage pour les LIS et CLIS

\citep{kubler_braincomputer_2008} plus la LIS progresse, plus les performances des EEG-BCI d�cro�ssent et pour les CLIS c'est pas super. Etude qui a essay� l'EEG sur les CLIS \citep{de_massari_brain_2013} mais �a marche pas trop par contre la r�ponse h�modynamique en fNIRS semble plus prometteuse \citep{gallegos-ayala_brain_2014} (d�codage 72\%)

% POUR LES PROBLEMES MOTEUR
- utilisation en g�n�ral des SMR
- SMR 8-12hz diminue pendant la t�che motrice (ERD) mais augmente pendant le repos (ERS)
- \citep{pfurtscheller_brain_2000} premi�re ICM bas�e sur les SMR (A REGARDER)
- D�codage d'imagerie motrice par fNIRS \citep{sitaram_temporal_2007, zimmermann_detection_2013}

% CONCLUSION
- tDCS permet l'apprentissage moteur chez les sujets sains \citep{cuypers_is_2013, reis_time_2015} ou chez les stokes \citep{cuypers_is_2013, lefebvre2013dual}
% //////////////////////////////////////

\figScaleDesciption{0.7}{wolpow_2002_marqueurs}{Marqueurs de l'activit� neuronale \citep{wolpaw_brain_2002}}{Blblabla \citep{wolpaw_brain_2002}}
% -----------------------------------------------------------------------------
% -----------------------------------------------------------------------------
%                         IMAGERIE MOTRICE
% -----------------------------------------------------------------------------
% -----------------------------------------------------------------------------
\section{Imagerie motrice}

\subsection{Les diff�rents types d'imagerie}

\subsection{Imagerie, intention et ex�cution motrice}
\citep{hanakawa_motor_2008}
\figScale{hanakawa_2008}{Comparatif}

\subsection{Utilisation de l'imagerie motrice pour les ICM}

\subsection{Autre utilisation de l'imagerie motrice}

