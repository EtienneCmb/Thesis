% #############################################################################
%                        PRESENTATION DE LA THEMATIQUE
% #############################################################################
\chapter{ICM et neurophysiologie}

% -----------------------------------------------------------------------------
% -----------------------------------------------------------------------------
%                            BASES PHYSIOLOGIQUES
% -----------------------------------------------------------------------------
% -----------------------------------------------------------------------------
\section{Bases physiologiques}
\figScaleX{0.6}{Grainmann_2002_Brain}{Blabla \citep{graimann_braincomputer_2009}}

\section{Apprentissage machine: applications aux neurosciences}
okok

% -----------------------------------------------------------------------------
\section{Encodage et d�codage moteur: bases physiologiques}
Pomper les r�sultas de rodrigo + besserve + inclure sch�ma global du cerveau

% -----------------------------------------------------------------------------
% -----------------------------------------------------------------------------
%                        MARQUEURS DE L'AN
% -----------------------------------------------------------------------------
% -----------------------------------------------------------------------------
\section{Marqueurs de l'activit� neuronale}
\label{sec_marqueurs_AN}

\subsection{Activit� \textit{spike}}
\subsection{P300}
\subsection{Ondes lentes}
\subsection{Rythmes sensorimoteurs}
\subsection{SSVEP}
% -----------------------------------------------------------------------------
% -----------------------------------------------------------------------------
%                         IMAGERIE MOTRICE
% -----------------------------------------------------------------------------
% -----------------------------------------------------------------------------
\section{Imagerie motrice}

\subsection{Les diff�rents types d'imagerie}

\subsection{Imagerie, intention et ex�cution motrice}
\citep{hanakawa_motor_2008}
\figScale{hanakawa_2008}{Comparatif}

\subsection{Utilisation de l'imagerie motrice pour les ICM}

\subsection{Autre utilisation de l'imagerie motrice}

