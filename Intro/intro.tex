% *****************************************************************************
% *****************************************************************************
%                                   INTRODUCTION
% *****************************************************************************
% *****************************************************************************
\part{Introduction g�n�rale}
\pagestyle{headings}

\malettrine{B}{lablablabkiblablou}  INTRO \dots\\*

% Le sujet de la th�se
L'objectif de cette th�se a  �t� de \dots\\*

Totalit� des m�thodes explor�es durant ma th�se sont pr�sentes dans une toolbox python appel�e brainpipe, libre d'acc�s et de droit. 


% #############################################################################
%                                CORPS DE L'INTRO
% #############################################################################
% #############################################################################
%                        PRESENTATION DE LA THEMATIQUE
% #############################################################################
\chapter{Pr�sentation de la th�matique}

En 1964, Dr. Grey Walter connecte des �lectrodes directement dans le cortex moteur d'un patient et lui demande de presser un bouton pour faire avancer un r�tro-projecteur. En m�me temps, il enregistre l'activit� neuronale de telle sorte que elle aussi, puisse le faire avancer. L� o� l'exp�rience devient remarquable, c'est que le r�tro-projecteur avance avant que le patient ne presse le bouton ! Tout l'appareil musculaire du sujet est court-circuit� et le contr�le se fait sans mouvement. Contr�ler par la \textit{pens�e}, un sujet de science fiction qui devient une r�alit�.\\
Cette anecdote d�crite par \cite{graimann_braincomputer_2009}, permet de placer la naissance des \icm (ICM) dans l'histoire. C'est le point d'entr�e qui a ensuite conduit une grande diversit� de chercheurs � se passionner pour ce sujet. \\

% -----------------------------------------------------------------------------
% -> D�finition d'une ICM
\section{\icm: d�finition et objectifs}

\subsection{Interactions naturelles avec l'environnement}
Pour interagir avec son environnement, l'individu se sert des voies de communications naturelles, \cad via son syst�me nerveux et musculaire. Le processus de communication d�bute par une intention qui active certaines r�gions dans le cerveau. Il en r�sulte un signal c�r�brale qui est ensuite envoy� par le syst�me nerveux p�riph�rique en directions des muscles. C'est ce processus simplifi� qui permet � une personne d'interagir avec ce qui l'entoure.

\subsection{ICM: une communication alternative}
Une ICM (ou BCI en anglais pour \textit{Brain Computer Interface}) est un autre syst�me de communication o� les voies naturelles sont cout-circuit�es. Au lieu de passer par le syst�me nerveux puis musculaire, le signal c�r�bral est directement intercept� au niveau du cerveau et va ensuite �tre transform� en commandes. Une ICM est donc un syst�me permettant de traduire une activit� neuronale en commande ext�rieure. Le terme \textit{traduire} est � prendre au sens linguistique \cad que les signaux c�r�braux forment un langage, compos� de r�gles, de motifs ou \textit{pattern}, que l'on va essay� de d�coder (via un ordinateur) pour les transformer en op�rations. D'o� le terme "\icm".

\subsection{Composantes d'une ICM}
\cite{pfurtscheller_rehabilitation_2008} et \citep{graimann_braincomputer_2009} introduisent quatre �l�ments qui composent une ICM:
\begin{enumerate}
	\item Enregistrer l'activit� directement depuis le cerveau. Cet enregistrement pourra �tre invasif ou non-invasif (cf. \ref{invasif_non-invasif})
	\item G�n�rer un retour ou \textit{feedback} pour l'utilisateur
	\item L'enregistrement et le \textit{feedback} doivent �tre en temps r�el
	\item Enfin, l'interface doit �tre contr�lable par l'utilisateur, de mani�re active, via un ensemble d'intentions. 
\end{enumerate}
A titre d'exemple et pour illustrer ce dernier point, un utilisateur pourrait par exemple d�cider de bouger un curseur de souris sur un �cran en imaginant des mouvements soit de la main gauche soit de la main droite.

\figScaleDesciption{0.7}{Pfurtscheller2008_SchemaICM}{Sch�ma d'une \icm \citep{pfurtscheller_rehabilitation_2008}}{L'activit� neuronale est enregistr�e (Signal acquisition) puis nettoy�e (Preprocessing). Ensuite, on extrait des motifs ou patterns qui caract�risent la commande que souhaite envoyer le sujet (Feature extraction). Enfin, la machine tente de reconna�tre ces motifs (Classification) et de les transformer en commande (Application interface) \citep{pfurtscheller_rehabilitation_2008}}

\subsection{Acquisition de l'activit� neuronale}
L'acquisition de l'activit� neuronale constitue la premi�re �tape d'une ICM. Deux param�tres permettent de d�crire ces diff�rentes techniques:
\begin{enumerate}
	\item L'accessibilit�: \cad si cette technique est plus ou moins facile � mettre en place. 
	\item La qualit�: \cad le rapport signal sur bruit (RSB)
\end{enumerate}
Si la technique d'enregistrement n'est pas facilement accessible, cela pourrait limiter le nombre de sujets potentiel pour l'�tude. La qualit� quant � elle va permettre d'avoir acc�s � des signaux c�r�braux de plus ou moins grande qualit� ce qui va influer sur la performance et sur les limitations d'une ICM. \\
Les diff�rents types d'acquisition peuvent �tre diff�renci�s par leur degr�s de p�n�tration dans le corps. Les enregistrements dits \textit{invasifs} vont n�cessiter une intervention chirurgicale mais donnent acc�s � des signaux de grande qualit�. Les enregistrements \textit{non-invasifs} sont globalement plus faciles � mettre en place car ne n�cessitant aucune chirurgie mais la qualit� du signal est moindre car l'activit� neuronale est enregistr�e en dehors de la bo�te cr�nienne. \\
Au sein de ces deux cat�gories, on trouvera un ensemble de m�thodes d'acquisition qui se diff�rencient par la taille des populations de neurones qu'elles enregistrent ou par le type de signal.

\subsubsection{Enregistrements invasifs}
On parlera d'intracr�nien pour les enregistrements � l'int�rieur de la bo�te cr�nienne puis d'intracortical pour des �lectrodes implant�es dans le cortex \citep{jerbi_task-related_2009, engel_invasive_2005}. �tant donn� que les techniques ci-dessous n�cessitent une intervention chirurgicale, \\

\textbf{\sua (SUA)} et \textbf{\mua (MUA)} : ces micro-�lectrodes d�pos�es directement au contact de neurones, disposent de la plus haute r�solution et enregistrent des d�charges neuronales. SUA et MUA se distinguent par la taille des populations enregistr�es.\\

\textbf{\seeg (SEEG)} : macro-�lectrodes enregistrant des populations plus larges que les micro-�lectrodes. contrairement � la SUA ou MUA o� l'on peut compter le nombre de fois qu'un ou plusieurs neurones d�chargent, la SEEG enregistre des potentiels �lectriques $(\mu V)$. \\

\textbf{\ecog (ECoG)} : grille flexible compos�e d'une matrice d'�lectrodes. Cette grille est ensuite d�pos�e � la surface corticale. Parce que cette m�thode n'est pas intracortical elle est dite \textit{semi-invasive}.

\subsubsection{Enregistrements non-invasifs}
Les m�thodes \textit{non-invasives} repr�sentent le but ultime de l'impl�mentation concr�te d'une \icm. En effet, ne n�cessitant aucune intervention chirurgicale, leur utilisation est tr�s r�pandue. Actuellement, bien qu'ayant une bonne r�solution temporelle permettant ainsi de capturer les ph�nom�nes courts, elles souffrent encore d'un manque de r�solution spatiale et d'un rapport signal sur bruit plus faible que les techniques invasives. \\

\textbf{\eeg (EEG)} : c'est la technique la plus utilis�e dans le domaine des ICM pour son aspect pratique et portatif. On dispose � la surface de la bo�te cr�nienne un ensemble d'�lectrodes qui enregistrent l'activit� �lectrique �manant de populations larges de neurones. On trouve m�me denouveaux syst�mes sans fils, prometteurs.\\

\textbf{\meg (MEG)} : autre m�thode \textit{non-invasive} mais nettement moins portable que l'EEG, la MEG enregistre les champs magn�tiques r�sultant de l'activit� �lectrique du cerveau. Ces champs magn�tiques peuvent �tre jusqu'� 10 milliards de fois plus faibles que le champs magn�tique terrestre. L'aspect inamovible de la MEG limite son utilisation pour les ICM mais on trouve quand m�me quelque �tudes ayant test� son utilisation \citep{mellinger_meg-based_2007, waldert_hand_2008}.

\figScaleDesciption{1}{Waldert_2009_recording}{M�thodes d'acquisition de l'activit� c�r�brale \citep{waldert_review_2009}}{M�thodes d'acquisition de l'activit� c�r�brale \citep{waldert_review_2009} class�es par invasivit�. La figure indique �galement la taille des populations de neurones enregistr�s ainsi que la r�solution spatiale intimement li�e � l'invasivit�.}

\subsection{Neuro-r�habilitation et handicap moteur}
Pr�lever directement l'activit� neuronale et donc, \textit{bypasser} les voies naturelles, permet de s'affranchir d'�ventuelles limitations physiologiques. C'est pourquoi les ICM repr�sentent un enjeu majeur pour la r�habilitation motrice ou handicap moteur ou encore pour la communication palliative.
Les applications concr�tes des \icm visent donc les personnes disposant de leurs capacit�s cognitives mais qui sont priv�es de facult�s motrices. \\
Par exemple, le \textit{syndrome d'enfermement} ou \textit{locked-in syndrome} qui surviennent majoritairement � la suite d'un accident vasculaire c�r�brale. Les personnes touch�es par ce syndrome sont pleinement consciente de leur corps, de l'environnement, ils peuvent ressentir les sensations de toucher et de douleur mais n'ont plus de facult�s motrices, hormis peut-�tre, les mouvements de paupi�res ou des yeux. Un autre exemple est celui de la la scl�rose lat�rale amyotrophique (ou SLA) qui est une d�g�n�rescence des neurones moteur. Progressivement, les personnes atteintes de SLA perdent l'usage des bras, des jambes, de la parole des muscles faciaux et enfin de la d�glutition. Toutefois, a l'instar du \textit{locked-in syndrome}, la conscience et les facult�s cognitives demeurent intactes. \\

SLA et \textit{locked-in syndrome} ne sont que deux exemples expliquant l'int�r�t social du d�veloppement des ICM. Redonner un peu de contr�le ou �tablir un canal de communication avec les familles des patients expliquent l'engouement qui existe depuis maintenant plus de 40 ans pour les ICM.

% Une description d�taill�e des ICM existantes est propos�e dans la section \ref{etat_de_lart_ICM}.


% -----------------------------------------------------------------------------
% Etat de l'art
\section{�tat de l'art des interfaces cerveau-machine}
\label{etat_de_lart_ICM}

Mettre des ICM plus r�centes voir faire un sch�ma r�capitulatif � mettre en annexe. Good idea

\citep{bekaert_les_2009}
\figScale{bekaert_2009_ICM}{Pipeline g�n�ral d'un \icm}

\subsection{Les diff�rents types d'\icm}
Okokok, ref vers ces diff�rents types
\subsubsection{ICM \textit{synchrones} et \textit{asynchrones}}
Intro blabla \\
\textbf{ICM \textit{synchrones}: } description\\
\textbf{ICM \textit{asynchrones}: } description

\subsubsection{ICM \textit{invasives} et \textit{non-invasives}}
\label{invasif_non-invasif}
Blabla intro \\
\textbf{ICM \textit{invasives} :} description \\
\textbf{ICM \textit{non-invasives} :} description \\
La partie qui va suivre introduit les diff�rents types d'enregistrement c�r�braux.

\subsection{ICM non-invasives}
P300 speller

\subsection{ICM invasives}
\citep{hochberg_reach_2012}
\figScale{hochberg_2012}{Contr�le d'un bras robotis�}

% -----------------------------------------------------------------------------
\section{Apprentissage machine: applications aux neurosciences}
okok

% -----------------------------------------------------------------------------
\section{Encodage et d�codage moteur: bases physiologiques}
Pomper les r�sultas de rodrigo + besserve + inclure sch�ma global du cerveau

% -----------------------------------------------------------------------------
\section{Intention, ex�cution et imagerie motrice}
La figure \ref{fig:hanakawa_2008}   repr�sente  une   trajectoire  d'un
processus markovien de saut, avec les notations associ�es.
\citep{hanakawa_motor_2008}
\figScale{hanakawa_2008}{Comparatif}

% -----------------------------------------------------------------------------
\section{Delayed task: protocole exp�rimental}
okok     % Pr�sentation de la th�matique
% #############################################################################
%                           OBJECTIFS DE THESE
% #############################################################################
\chapter{objectifs de la th�se}

% -----------------------------------------------------------------------------
\section{D�codage c�r�brale � partir d'activit� intracr�nienne}
\begin{itemize}
\item Exemple d'un sch�ma d'implantation + IRM 
\item Bipolarisation: d�bruitage et augmentation de la sp�cificit� (article Karim)
\item Extraction de features (ici on pourrait mentionner que le deep learning pourrait marcher sur les donn�es brutes)
\end{itemize}

% -----------------------------------------------------------------------------
% Features
\section{Exploration et am�lioration des features}
\subsubsection{R�le physiologique du \pacEN}
\citep{hyafil_neural_2015}
\figScale{hyafil_2015}{M�canismes du couplage phase-amplitude}


% -----------------------------------------------------------------------------
\section{Comparatif des classifieurs}
Expliquer que, chaque classifier poss�de une m�thodologie propre permettant de r�pondre � des types de donn�es diff�rentes (en fonction des hypoth�ses de fonctionnement de chacun des classifiers)

% -----------------------------------------------------------------------------
\section{Exploration des r�gions non-motrices}
\citep{van_langhenhove_interfaces_2008}
\figScale{langhenhove_2008}{Localisation des aires sensorimotrices} % Objectifs de th�se
% #############################################################################
%                             METHODOLOGIE
% #############################################################################
\chapter{M�thodologie}

\todo[color=red!100]{Inclure une meilleure description des figures}
Cette partie m�thodologique sera divis�e en deux grandes sous parties visant � pr�senter :
\begin{enumerate}
	\item L'extraction des features: pr�sentation des m�thodes utilis�es dans le cadre de l'extraction d'attributs issus de l'activit� neuronale. De mani�re g�n�rale, nous avons �tudi�s des attributs spectraux comprenant:
	\begin{itemize}
		\item Phase et puissance spectrale
		\item Attributs de couplage
	\end{itemize}
	\item Le machine learning: pr�sentation des principaux algorithmes test�es dans le cadre du d�codage de l'activit� neuronale 
\end{enumerate}


\section{Extraction des features}
% -----------------------------------------------------------------------------
% -----------------------------------------------------------------------------
%                                   FEATURES
% -----------------------------------------------------------------------------
% -----------------------------------------------------------------------------
Comme nous l'avons d�crit pr�c�demment, l'objectif du d�codage de l'activit� neuronale est d'arriver � extraire des signaux c�r�braux une information suffisamment pertinente pour pouvoir discriminer diff�rents types de classes (exemple: mouvement vers la gauche Vs droite). \\
Tout les attributs test�s dans le cadre de cette th�se sont des attributs spectraux, donc issus de bandes de fr�quences. La plupart de ces outils partagent donc une partie m�thodologique commune � savoir, le filtrage. De plus, la plupart sont extraits en utilisant la transform�e d'Hilbert. Pour �viter une redondance � travers les attributs, nous allons tout d'abord introduire quelques pr�-requis.

\subsection{Pr�-requis}
\todo[color=red!100]{Cette partie ne devrait pas figurer dans l'extraction des features}
% ********************************************
%               PRE-REQUIS
% ********************************************
\subsubsection{Filtrage}
L'int�gralit� des filtrages dans cette th�se ont �t� effectu�s avec la fonction \textit{eegfilt} (qui a ensuite �t� reproduite pour le passage � python). De plus, afin d'�viter tout ph�nom�ne de d�phasage, le fonction \textit{filtfilt} a �t� syst�matiquement utilis�e afin que le filtre soit appliqu� dans les deux sens. Si cette derni�re fonctionnalit� n'est pas forc�ment indispensable dans le cadre d'un calcul de puissance, elle est absolument n�cessaire pour un calcul de \pacFR.  \\
L'ordre du filtre pr�sent� au dessus d�pend de la fr�quence de filtrage. Il a syst�matiquement �t� calcul� en utilisant la m�thode d�crite par \cite{bahramisharif_propagating_2013}:
\begin{equation}
FiltOrder = N_{cycle} \times f_{s}/f_{oi}
\end{equation} \\
o� $f_{s}$ est la fr�quence d'�chantillonnage, $f_{oi}$ est la fr�quence d'int�r�t et $N_{cycle}$ est un nombre de cycles d�finit par $N_{cycle}=3$ pour les oscillations lentes et $N_{cycle}=6$ pour les oscillations rapides. 

\subsubsection{Transform�e d'Hilbert}
Transform�e permettant de passer un signal temporel $x(t)$ du domaine r�el au domaine complexe. Le signal peut ensuite s'�crire $x_{H}(t)=a(t)e^{j\phi(t)}$ o� $a(t)$ est l'amplitude et $\phi(t)$, la phase. Cette transformation est particuli�rement exploit�e car le module de $x_{H}(t)$ permet de r�cup�rer l'amplitude et la phase est obtenue en prenant l'angle de $x_{H}(t)$.

\subsubsection{Transform�e en ondelettes}
La transform�e en ondelettes \citep{tallon-baudry_oscillatory_1997, worrell_recording_2012} permet de d�composer un signal dans le domaine temps-fr�quence. La d�composition en ondelettes d'une fonction $f$ est d�finie par:
\begin{equation}
f(a, b)=\int_{-\infty}^{\infty} \mathrm{f}(x)\overline{\psi}_{a, b}\mathrm{d}x 
\end{equation} \\
O� $\psi$ est appel� ondelette m�re dont la d�finition g�n�rale est donn�e par $\psi_{a,b}=\frac{1}{\sqrt{a}}\Psi(\frac{x-b}{a})$ o� $a$ est le facteur de dilatation et $b$ le facteur de translation. Le choix de l'ondelette m�re s'est port� sur l'ondelette de Morlet qui est tr�s largement utilis�e � travers la litt�rature et d�finie par:
\begin{equation}
w(t,f_{0})=A \mathrm{e}^{-t^{2}/2\sigma_{t}^{2}} \mathrm{e}^{2i\pi f_{0}t}
\end{equation}\\
O� $\sigma_{f}=1/2\pi \sigma_{t}$ et $A=(\sigma_{t}\sqrt{\pi})^{-1/2}$. L'ondelette de Morlet est caract�ris�e par le ratio constant $r=f_{0}/\sigma_{f}$ que nous avons fix� �gale � $7$ comme sugg�r� par \cite{tallon-baudry_oscillatory_1997}.\\ 
Cette d�composition peut �tre compar�e � la transform�e courte de Fourier qui d�compose le signal en une somme de combinaisons lin�aire de sinus et de cosinus mais part du principe qu'il existe une r�gularit� dans le signal permettant une telle d�composition. La transform�e en ondelettes r�sout plusieurs limitations:
\begin{itemize}
\item Elle permet d'obtenir l'�nergie d'un signal dans le temps, ce qui permet une bien meilleure exploration des ph�nom�nes.
\item Le rapport constant $r$ permet d'obtenir des ondelettes dont la r�solution fr�quentielle varie    en fonction des fr�quences et permet une meilleure co�ncidence avec la d�finition des bandes physiologiques \citep{bertrand_time-frequency_1994}
\end{itemize}
\vspace{1\baselineskip}
Tout les attributs qui vont �tre maintenant pr�sent�s, utilisent les m�thodes d�crites ci-dessus.

\subsubsection{�valuation statistique � base de permutations}
Pour une distribution de permutations construite � partir de deux sous-ensembles $A$ et $B$ et comportant $N$ observations et pour une valeur $p$ pr�d�finie, on pourra conlure que:
\begin{itemize}
	\item $A>B$ si $A$ est parmi les $N-N\times p$ derniers �chantillons (\textit{"One-tailed test upper tail"})
	\item $A<B$ si $A$ est parmi les $N\times p$ premiers �chantillons (\textit{"One-tailed test lower tail"})
	\item $A\neg B$ si $A$ est soit inf�rieur aux $(N\times p)/2$ premiers �chantillons soit sup�rieur aux $(N-N\times p)/2$
\end{itemize}  
\figScaleX{0.6}{stat_permutations}{\textit{"One-tailed"} et \textit{"two-tailed"} test}
Gr�ce � cette m�thode d'�valuation statistique, nous pourrons par exemple conclure si l'on a une augmentation, une diminution ou une diff�rence statistique entre une valeur de puissance et la puissance contenue dans une p�riode de baseline. Derni�re pr�cision, on comprend ainsi que pour obtenir une valeur $p$ il faut que la taille de la distribution $N$ soit au moins de $1/p$.

\subsubsection{Hyperplan}
Un hyperplan est un espace de co-dimension 1. Donc, dans un espace $3D$, l'hyperplan est un plan (dimension $2D+1$). De mani�re g�n�rale, un espace de dimension $N$ poss�de un hyperplan de dimension $N-1$
\begin{equation}
	dim_{ESPACE} = dim_{HYPERPLAN}+1
\end{equation}

% ********************************************
%                 PUISSANCE
% ********************************************
\subsection{Puissance spectrale}

\subsubsection{M�thodes explor�es}
Le calcul de la puissance spectrale a �t� approch� par deux m�thodologies et qui ont �t� utilis�s � des fins diff�rentes :
\begin{itemize}
	\item La transform�e d'Hilbert: souvent exploit� dans le cadre du d�codage ainsi que pour garder une uniformit� entre les attributs de phase et \pacFR bas�s eux aussi sur cette transform�e.
	\item La transform�e en ondelettes: principalement utilis�e pour la visualisation des cartes \tf � cause de l'adaptation des ondelettes aux bandes physiologiques.
\end{itemize}

\subsubsection{Normalisation}
On utilise la normalisation pour observer l'�mergence d'un ph�nom�ne par rapport � une p�riode d�finie comme baseline. A travers la litt�rature, quatre grands types de normalisation sont rencontr�s:
\begin{enumerate}
	\item Soustraction par la moyenne de la baseline
	\item Division par la moyenne de la baseline
	\item Soustraction puis division par la moyenne de la baseline
	\item Z-score: soustraction de la moyenne puis division par la d�viation de la baseline
\end{enumerate}
La normalisation z-score est certainement la plus fr�quemment rencontr�e � travers la litt�rature. Le choix du type de normalisation d�pend du type de donn�es utilis�es. Dans le cadre de nos donn�es, \textit{3.} �tait clairement la plus adapt�e pour la visualisation. En revanche, dans le cadre de la classification, nous obtenions syst�matiquement de meilleurs r�sultats sans normalisation.

\subsubsection{�valuation statistique}
La fiabilit� statistique de la puissance a �t� �valu�e en comparant chaque valeur de puissance � la puissance contenue dans une p�riode d�finie comme baseline. Pour ce faire, nous avons test� deux approches:
\begin{enumerate}
	\item Permutations : les valeurs de puissance et de baseline sont al�atoirement m�lang�es � travers les essais. Puis, on normalise cette puissance. En r�p�tant cet proc�dure $N$ fois, on obtient une distribution qui peut ensuite �tre utilis�e pour en d�duire la valeur $p$ de la v�ritable puissance (cf: \textit{pr�-requis})
	\item "Wilcoxon signed-rank test": ordonne les distances entre les paires de puissances (vraie valeur, baseline) \citep{demandt_reaching_2012, rickert_encoding_2005, waldert_hand_2008}
\end{enumerate}
\figScaleX{0.6}{ossandon_tf}{Exemple de repr�sentation temps-fr�quence de puissance normalis�es z-score \citep{ossandon_transient_2011}}



% ********************************************
%                     PHASE
% ********************************************
\subsection{Phase}
L'extraction de la phase se fait de la m\^{e}me mani�re que pour le \PACFR, en prenant l'angle de la transform�e d'Hilbert d'un signal filtr�. La significativit� peut �tre �valu�e en utilisant le test de Rayleigh \citep{jervis_fundamental_1983, tallon-baudry_oscillatory_1997}. Point de vue pratique, cela correspond � la fonction \textit{circ\_rtest} de la toolbox Matlab \textit{CircStat} \citep{berens_circstat_2009}




% ********************************************
%                   PAC
% ********************************************
\subsection{\PACEN}
Le calcul du \PACEN ne se limite pas uniquement � la m�thode. En r�alit�, pour obtenir une estimation fiable sur des donn�es r�elles, il est indispensable de suivre les trois �tapes suivantes:
\begin{enumerate}
	\item Estimation de la v�ritable valeur de PAC. Il existe plusieurs m�thodes.
	\item Calcul de "\textit{surrogates}": on va calculer des PAC d�structur�s. Idem, il existe de nombreuses m�thodes
	\item Correction du v�ritable PAC par les "\textit{surrogates}". Cette correction, qui est en faite une normalisation, aura pour but de soustraire � l'estimation du PAC de l'information consid�r�e comme bruit�e.
\end{enumerate}
Les sous-parties suivantes pr�senteront de mani�res succinctes les principales m�thodes rencontr�es dans la litt�rature, ainsi que diff�rents types de corrections applicables.

\subsubsection{M�thodologie du \pacEN}
Il existe une large vari�t� de m�thodes pour calculer le PAC, ce qui complique son exploration. Toutefois, il n'existe pas de consensus sur une m�thode plus polyvalente qu'une autre, chacune poss�dant ses points forts et limitations.
Pour aller un peu plus loin, et pr�senter quelques m�thodes, il est n�cessaire d'introduire quelques variable. Soit $x(t)$, une s�rie temporelle de donn�es de taille N. Pour cette s�rie temporelle, on souhaite savoir si la phase extraite dans une bande de fr�quence $f_{\phi}=[f_{\phi_{1}},f_{\phi_{2}}]$ est coupl�e avec l'amplitude contenue dans $f_{A}=[f_{A_{1}},f_{A_{2}}]$. Pour cela, on va tout d'abord extraire $x_{\phi}(t)$ et $x_{A}(t)$ les signaux filtr�s dans ces deux bandes. Enfin, la phase $\phi(t)$ est obtenue en prenant l'angle de la transform�e d'Hilbert de $x_{\phi}(t)$ tandis que l'amplitude $a(t)$ est obtenue en prenant le module de la transform�e d'Hilbert de $x_{A}(t)$. 

\begin{enumerate}

	% MEAN VECTOR LENGTH
	\item \mvl: \\
	Cette m�thode � �t� introduite par \cite{canolty_high_2006} et consiste � sommer, � travers le temps, le complexe form� de l'amplitude des hautes fr�quences avec la phase des basses fr�quences. L'�quation est donn�e par:
	\begin{equation}
	MVL = |\sum_{j=1}^{N} a_(j) \times e^{j\phi(j)}|
	\end{equation} \\
	
	% KULLBACK-LEIBLER DIVERGENCE
	\item \kld:\\
	A l'origine, la divergence de Kullback-Leibler (KLD), qui est issue de la th�orie de l'information, permet de mesurer les dissimilarit�s entre deux distributions de probabilit�s. Ainsi, pour pouvoir utiliser cette mesure dans le cadre du PAC, \cite{tort_measuring_2010} propose une solution �l�gante qui consiste � g�n�rer une distribution de densit� probabilit�s de l'amplitude (DPA) en fonction des valeurs de phase et d'ensuite utiliser le KLD pour comparer cette distribution � la densit� de probabilit� d'une distribution uniforme (DPU). Plus la DPA s'�loigne de la DPU, plus le couplage entre l'amplitude et la phase est consistant. \\
	Pour construire la DPA, l'astuce consiste � couper le cercle trigonom�trique en N tranches (dans l'article il est propos� de couper en 18 tranches de 20�). Puis, si on prend l'exemple de la tranche $[0,20�]$, on va chercher tout les instants temporels o� la phase prend des valeurs comprises entre $[0,20�]$ ($t, \phi(t) \subset [0,20�]$). On prend ensuite la moyenne de l'amplitude pour ces valeurs de $t$ et on r�p�te cette proc�dure pour chacune des tranches de phase. On obtient ainsi la densit� d'amplitudes en fonction des valeurs de phase. Il ne reste plus qu'� normaliser cette distribution par la somme des amplitudes � travers les tranches et on r�cup�re une distribution de densit� de probabilit�s. La figure \ref{fig:PAC_plot_Tort_2010} \citep{tort_measuring_2010} pr�sente un exemple de DPA en fonction de tranches de phase. \\

	\figScaleX{0.6}{PAC_plot_Tort_2010}{Densit� de probabilit� d'une distribution d'amplitudes en fonction de tranches de phases}
	
	Le calcul de la divergence de Kullback-Leibler est ensuite appliqu� pour mesurer les dissimilarit�s entre la DPA et la DPU et c'est cette mesure qui servira d'estimation du \pacFR:
	\begin{equation}
	D_{KL}(P, Q) = \displaystyle\sum_{j=1}^{N} P(j) \times \log{\frac{P(j)}{Q(j)}}
	\end{equation} \\
	o� $P(j)$ est la densit� de probabilit� de $a(t)$ en fonction de $\phi(t)$ et $Q(j)$ est la densit� de probabilit� d'une distribution uniforme. \\
	
	% HEIGHT-RATIO
	\item \hr \\
	La m�thode du \hr \citep{lakatos_oscillatory_2005} est extr\^{e}mement proche du \kld. En effet, l'amplitude sera bin�e de la m�me fa�on en fonction des tranches de phase. La mesure du PAC est ensuite donn�e par:
	\begin{equation}
	hr=(f_{max}-f_{min})/f_{max}
	\end{equation} \\
	o� $f_{max}$ et $f_{min}$ sont respectivement le maximum et le minimum de la de la densit� de probabilit� de l'amplitude en fonction des valeurs de phase. \\
	
	% NDPAC
	\item \ndpac \\
	Le \ndpac, qui n'est pas une des m�thodes les plus fr�quemment rencontr�es, pr�sente toutefois une avantage certain. En plus de fournir une estimation fiable du \pacFR, \cite{ozkurt_statistically_2012} d�montre l'existence d'un seuil � partir duquel on peut consid�rer l'estimation du PAC comme �tant statistiquement fiable. La beaut� de cette m�thode, c'est que ce seuil statistique, qui est une fonction de la valeur p d�sir�e, ne d�pend que de la taille de la s�rie temporelle. Ce qui rend son utilisation particuli�rement simple. \\
	Pour estimer le PAC, une des hypoth�ses ayant permis d'aboutir � ce seuil statistique est de devoir normaliser l'amplitude par un z-score d�not�e $\tilde{a}(t)$. L'estimation du PAC est quasiment identique au MVL puisque c'est en r�alit� le carr� de celle-ci. Enfin, pour une valeur p d�sir�e, l'article introduit le seuil statistique: \\
	\begin{equation}
	x_{lim}=N \times [erf^{-1}(1-p)]^{2}
	\end{equation} \\
	
	o� $erf^{-1}$ est la fonction d'erreur inverse. On d�duira que l'estimation PAC est significative si et seulement si cette valeur est deux fois sup�rieur � ce seuil. \\
	
	% AUTRES
	\item Autres m�thodes:
	Tout les algorithmes pr�sent�s ci-dessus ont �t� test�s, impl�ment�s et compar�s. En compl�ment, voici une liste non exhaustive d'autres m�thodes existantes:
	\begin{itemize}
		\item \textit{Phase Locking Value (PLV)} \citep{cohen_assessing_2008, penny_testing_2008}: d�tournement du PLV propos� par \cite{lachaux_measuring_1999} qui mesure la synchronie de phase entre deux �lectrodes. Cette m�thode va comparer la phase des basses fr�quences avec la phase de l'amplitude des hautes-fr�quences.
		\item \textit{Generalized Linear Model (GLM)} \citep{penny_testing_2008}: outil d�crit comme adapt� aux donn�es courtes et bruit�es.
		\item \textit{Generalized Morse Wavelets (GMW)} \citep{nakhnikian_novel_2016}: bas�e sur des ondelettes, semble particuli�rement utile dans le cadre de l'exploration des donn�es.
		\item \textit{Oscillatory Triggered Coupling (OTC)} \citep{dvorak_toward_2014, watrous_phase-amplitude_2015}: issue d'une d�tection de maximums des hautes fr�quences.
	\end{itemize}

\end{enumerate}

% PERMUTATIONS ET NORMALISATION
\subsubsection{Correction du \pacEN et �valuation statistique}
Nous avons vu dans la section pr�c�dente diff�rentes m�thodes permettant de calculer un \PACFR. Toutefois, celui-ci peut �tre largement am�lior�e en faisant une estimation du PAC contenu dans le bruit des donn�es. Une fois que cette estimation sera faite, on pourra retrancher ce PAC bruit� � la valeur initiale. Tout comme il existe plusieurs m�thodes de PAC, les �quipes de recherche proposent � tour de r�le de nouvelles m�thodes. Parmi elles, on peut citer:
\begin{itemize}
	\item \textit{Time-lag}: propos�e par \cite{canolty_high_2006}, on introduit un d�lai sur l'amplitude compris entre $[f_{s},N-f_{s}]$ o� $f_{s}$ est la fr�quence d'�chantillonnage et $N$ est le nombre de points de la s�rie temporelle
	\item \textit{Shuffling des couples [phase,amplitude]}: ici, on m�lange al�atoirement les essais de phase et d'amplitude \citep{tort_measuring_2010}
	\item \textit{Swapping temporel d'amplitudes (ou de phase)}: on m�lange al�atoirement les essais d'amplitude puis on recalcule le PAC avec la phase originale \citep{bahramisharif_propagating_2013, lachaux_measuring_1999, penny_testing_2008, yanagisawa_regulation_2012} 
\end{itemize}
Ces trois m�thodes produisent une distribution de \textit{surrogates}. On pourra ensuite appliquer un z-score � la v�ritable estimation en utilisant la moyenne et la d�viation de cette distribution. Enfin, l'�valuation statistique se fait �galement � partir de cette distribution  (cf: \textit{pr�-requis}) \\
A ma connaissance, il n'existe pas de comparatif entre ces corrections et je n'ai jamais rencontr� d'articles mentionnant que l'on ne puisse pas combiner les m�thodes de PAC avec les diff�rentes corrections. En revanche, ce qui est relat� c'est que le \textit{time-lag} n�cessite des donn�es longues d\^{u} � l'introduction de ce d�lai temporel. 

% COMPARATIF
\subsubsection{Comparatif des m�thodes}
\cite{penny_testing_2008} ont compar� plusieurs m�thodes dont le \textit{MVL}, \textit{PLV} et le \textit{GLM} et \cite{tort_measuring_2010} ont compl�t� cette �tude avec d'autres m�thodes (cf. \ref{comp_pac}). Enfin, \cite{canolty_functional_2010} a fait une review qui comprend un descriptif tr�s instructif.

% VISUALISATION
\subsubsection{Repr�sentation du \pacEN}
Compar�e � la puissance, l'exploration du PAC peut s'av�rer plus complexe d\^{u} � sa dimensionnalit� plus grande. Il existe donc des outils et des m�thodes destin�es � simplifier cette exploration et � visualiser ces r�sultats. \\
Exemple concret, si on cherche � conna\^{i}tre les modulations de puissance contenue dans un signal, on peut repr�senter une carte \tf. Pour le PAC, id�alement on voudrait visualiser les phases, les amplitudes et le temps mais ces trois dimensions emp�che une repr�sentation simple. On peut donc avoir recours � diff�rents types de repr�sentations compl�mentaires:
\begin{itemize}
	\item Puissance phase-locked : cette repr�sentation permet de faire �merger l'existence d'un couplage, pour une phase donn�e, et d'observer sa dur�e. Pour cela, on aligne les phase en d�tectant le pic le plus proche de l'instant temporel �tudi�. On calcul les cartes \tf que l'on va ensuite moyenner apr�s les avoir recal�es de la m�me fa�on que les phases (\cad avec la m�me latence).
	\item Comodulogramme : pour une tranche temporelle d�finie, on repr�sente les valeur de PAC pour diff�rentes valeurs de phase et d'amplitude
\end{itemize}

\figScaleX{1}{hemptinne_2013}{\textbf{(A)} Exemple de cartes temps-fr�quence phase locked sur le $\beta$, \textbf{(B)} Exemple de comodulogramme}

La figure \ref{fig:hemptinne_2013} \citep{hemptinne_exaggerated_2013} met en �vidence que la repr�sentation des cartes temps-fr�quence phase-locked \textbf{(A)} est limit�e d'une part, par la phase sur laquelle on choisit de recaler et d'autre part cette m�thode est �galement limit� par l'instant o� l'on choisit de recaler. Pour la figure \textbf{(B)}, le calcul du PAC se faisant � travers la dimension temporelle, on a aucune id�e de l'�volution du couplage dans le temps. \\

% ERPAC
\subsubsection{\PACEN: r�solution temporel?}
Comment peut-on savoir si un ensemble de musiciens jouent ensemble, en rythme? L'approche traditionnelle consiste � dire que, en fonction de la prestation du groupe, on sera en mesure de dire si ils �taient en rythme ou non. Donc on focalise notre attention sur chaque instant du morceau et on analyse chaque note, chaque d�calage. Cela signifie aussi que toute notre attention a �t� mobilis�e par l'analyse du rythme et finalement, on passe � c�t� de la musique. Notre attention au d�tail nous a �cart� du morceau global. On pourrait dire que l'on a �cras� la dimension temporelle du morceau. Une autre approche consiste � assister � toute les r�p�titions du fameux groupe. Ce faisant, on est capable de dire si d'une mani�re g�n�rale les musiciens ont tendance � jouer ensemble. Ainsi, le jour d'une repr�sentation, toute notre attention peut rester uniquement sur le concert. On garde donc la dimension temporelle.\\ 
C'est par ce changement de positionnement face au probl�me de r�solution temporelle que \cite{voytek_method_2013} introduit le \erpac . L'approche traditionnelle du PAC n�cessitant de conna�tre un nombre de cycles afin d'en d�duire l'existence ou non du couplage, et donc perdre la dimension temps, l'article propose de calculer le PAC � travers les essais (ou r�p�titions). Pour un jeu de donn�es de M essais de longueur N, on extrait respectivement les phases et les amplitudes $\phi_{M}(t)$ et $a_{M}(t)$ puis, pour chaque point temporel, on calcul la corr�lation � travers les essais (corr�lation lin�aire-circulaire \citep{berens_circstat_2009} qui se fait entre l'amplitude et des sinus/cosinus de la phase). Il en r�sulte une valeur de corr�lation pour chaque instant et donc, de couplage. \\



\section{Apprentissage supervis�}
% -----------------------------------------------------------------------------
% -----------------------------------------------------------------------------
%                           APPRENTISSAGE SUPERVISE
% -----------------------------------------------------------------------------
% -----------------------------------------------------------------------------
Le travail effectu� durant cette th�se s'est exclusivement port� sur l'apprentissage supervis�. Celui-ci consiste � apprendre � la machine � reconna\^{i}tre des �v�nements qui ont �t� labellis� au pr�alable (cf. \ref{labellisation}). A contrario, l'apprentissage non supervis� laisse la machine apprendre par elle-m\^{e}me. En pratique, l'apprentissage se fait sur des attributs. Par exemple, pour diff�rencier des chats et des chiens, ou pourra utiliser l'angle form� par le sommet des oreilles. Les attributs doivent contenir une information pertinente permettant de diff�rencier les classes. Enfin, les algorithmes de classification vont se servir de ces attributs pour d�finir une fronti�re entre les classes �tudi�es. A ce stade, il semble important de pr�ciser que l'utilisation des outils d'apprentissage machine peut s'orienter (globalement) suivant deux axes:
\begin{enumerate}
	\item Optimisation des attributs: on travail sur un raffinement des attributs afin que ceux-ci soient les plus performants possibles pour s�parer les classes
	\item Optimisation des param�tres de classification: on consid�re une base de donn�es comme �tant fixe, d�finitive, optimale et l'on va faire varier les diff�rents param�tres li�s � l'apprentissage machine (classifeurs, cross-validation...). C'est le cas des comp�titions \textit{BCI} o� tout le monde travail sur une m�me base de donn�es.
\end{enumerate}
Bien s\^{u}r, ces deux axes peuvent �tre cumul�s. Dans le cadre de cette th�se, le machine learning a �t� utilis� comme outil de validation d'hypoth�ses donc essentiellement port� sur l'optimisation des attributs. Le raffinement des param�tres de classification a �galement �t� �tudi�, mais, au final, il ne constitue pas la majeure partie de l'�tude.
\vspace{1\baselineskip}

Un sch�ma classique d'analyse peut-�tre d�crit par:
\begin{enumerate}
	\item Labellisation des donn�es
	\item Constitution de donn�es d'entra\^{i}nement (\train) et de test (\test)
	\item Choix d'un classifieur puis entra\^{i}nement de celui-ci sur les donn�es \train
	\item Test de ce classifeur entra\^{i}n� sur les donn�es \test et �valuation de la performance
	\item �valuation statistique de cette acuit� de d�codage
\end{enumerate}

% ********************************************
%              LABELLISATION
% ********************************************
\subsection{Labellisation et apprentissage}\label{labellisation}
La labellisation c'est le fait d'associer � chaque �v�nement l'appartenance � une classe ou � une condition. C'est par ce proc�d� que l'on va pouvoir apprendre ensuite au classifieur � identifier les classes. Par exemple, consid�rons \textit{up} et \textit{down} deux classes qui refl�tent des mouvements de la main vers le haut ou vers le bas. On va donc construire un vecteur $y_{direction}$ qui labellise chaque essais avec direction effectu�e (ce vecteur peut aussi \^{e}tre bool�en ou contenir des entiers. L'essentiel est que � chaque classe soit attribu� une valeur qui lui est propre). Ce vecteur $y$ est appel� \textit{vecteur label}, qui vient labelliser chaque essais d'un vecteur d'attributs $x$. \\ 
\begin{equation}
	y_{direction}=
	\begin{pmatrix}
		up \\
		down \\
		down \\
		\vdots \\\
		up
	\end{pmatrix},
	y_{bool}=
	\begin{pmatrix}
		0 \\
		1 \\
		1 \\
		\vdots \\\
		0
	\end{pmatrix},
	x=
	\begin{pmatrix}
		x_{trial_{1}} \\
		x_{trial_{2}} \\
		x_{trial_{3}} \\
		\vdots \\
		x_{trial_{N}} \\
	\end{pmatrix}
\end{equation}
D'o� le nom apprentissage supervis�. Finalement, l'apprentissage machine se fera gr\^{a}ce � ce vecteur label $y$ et cette matrice d'attributs $x$. Ce qui nous am�ne directement aux notions de \textit{training set} et de \textit{testing set}.\\
\figScaleX{0.5}{clf_labeling}{Labellisation de donn�es}


% ********************************************
%           TRAINING ET TESTING
% ********************************************
\subsection{\textit{Training}, \test et \cvfr}
Cette section est sans aucun doute la plus importante pour le machine learning puisque c'est elle qui assure la conformit� m�thodologique. \\
Un bon exemple pour comprendre cette partie est celui des contr\^{o}les de math�matiques. Avant l'examen, l'�tudiant s'entra\^{i}ne sur une s�rie d'exercices. C'est la phase de \train. D'ailleurs, plus il s'entra\^{i}ne, plus ses chances de r�ussir � l'examen sont grandes. Le jour du contr\^{o}le, le professeur teste l'�tudiant sur une s�rie de nouveaux exercices en lien avec ce qu'il a �tudi�. C'est le \test. Ici, c'est un test parfait puisque l'�tudiant est na�f sur le contenue de l'examen ce qui veut dire que l'on teste ses capacit�s math�matiques pures. Toutefois, il peut arriver durant la scolarit� que l'on soit test� sur des exercices que l'on a d�j� vu dans la phase de \train. Dans ce cas, la moyenne des notes des �tudiants est g�n�ralement beaucoup plus �lev�e puisque l'on ne teste plus des capacit�s math�matiques, mais la capacit� � restituer un apprentissage.

\subsubsection{\textit{Training set}, \textit{testing set} et na�vet�}
Pour en revenir � la question du machine learning, on d�finit un une partie des donn�es pour entra�ner la machine. Ensuite, on teste cette machine entra\^{i}n�e sur un nouveau jeu de donn�es de test. Il est essentiel d'avoir une s�paration stricte entre des donn�es d�finies comme \train et des donn�es de \test afin d'assurer la na�vet� du classifieur. M\^{e}me si cela peut par\^{i}tre �vident, nous verrons que �a n'est pas toujours aussi facile que �a. \\
Se pose maintenant la question de comment l'on choisit de couper les donn�es en \train et \test. Une m�thode serait de prendre une partie des donn�es de mani�re al�atoire, de la d�finir comme \train et sur tester sur les donn�es restantes. Toutefois, ce choix ne repr�senterai qu'une partie des donn�es. Une m�thode plus exhaustive et plus rigoureuse consiste � utiliser une \cvfr (ou \cv).

\subsubsection{Validation-crois�e}
La \cvfr (CV) est une proc�dure permettant de s�parer les donn�es en \train et \test. Pour comprendre comment cela fonctionne, prenons un ensemble compos� de $N$ �chantillons. Il existe plusieurs de CV mais de mani�re g�n�rale, toutes d�rivent du m�me principe qui est la \cv \kfold \citep{efron1994introduction, kohavi_study_1995}. On coupe les $N$ �chantillons en $k$ paquets de tailles �gales (ou proches). Ensuite, le classifieur est entra�n� sur $k-1$ paquets puis on le test sur le paquet restant. Cette proc�dure est ensuite appliqu�e $k$ fois afin que chaque paquet passe au \test. On dira que la \cv est \textit{stratified} si la proportion de classes repr�sent�es au sein de chaque dossier est approximativement uniforme � travers les folds. on pourra aussi rencontrer le terme \textit{shuffle} si il y a un m�lange suppl�mentaire. Tout cela nous emm�ne � des CV k-fold, k-fold stratified, k-fold shuffle ou encore k-fold stratified shuffle. \\
Concernant le nombre de folds, on rencontre en g�n�rale 3 valeurs � travers la litt�rature: 3-folds, 5-folds ou 10-folds \citep{latinne2001limiting, yanagisawa_neural_2009, besserve_classification_2007, waldert_hand_2008}. Un cas particulier, mais si le nombre de folds $k=N$, �a revient � entra�ner la machine sur $N-1$ �chantillons tester sur celui qui a �t� isol� et on r�p�te cette proc�dure $N$ fois. C'est ce que l'on appelle le \textit{\loo}. Toutefois cette derni�re poss�de une grande variance et peut conduire � des estimations non fiables \citep{efron1994introduction, kohavi_study_1995}. \\
Un autre cas particulier, est celui du \textit{Leave-p-Subject-Out} \citep{vidaurre2009time, lajnef_learning_2015} qui consiste � entra�ner sur $p$ sujets et tester sur les sujets restants. Cette proc�dure est particuli�rement exigeante puisqu'elle n�cessite d'avoir une certaine reproductibilit� entre les sujets. Cette \cvfr est fr�quente avec des donn�es EEG mais impossible � mettre en  \oe{}uvre pour la sEEG � cause de l'implantation unique de chaque sujet. 
\figScaleX{1}{clf_cv}{Exemple d'une cross validation 3-folds}


% ********************************************
%                 CLASSIFIEURS
% ********************************************
\subsection{Classifieurs}
\begin{enumerate}

	% LINEAR DISCRIMINANT ANALYSIS
	\item \lda (LDA) \\
	Le LDA \citep{fisher_use_1936} est un classifieur lin�aire. Pour un probl�me � deux classes, le LDA tente de trouver un hyperplan qui va maximiser la distance entre les classes tout en minimisant la variance inter-classes. Ce classifieur fait l'hypoth�se que les donn�es sont normalement distribu�es avec la m�me co-variance. Un probl�me multi-classes pouvant �tre transform�e en multiple bi-classes, le LDA tente de trouver un hyperplan s�parant la classe du reste (\textit{One-vs-All}) \\	
	\figScaleX{0.4}{LDA_Lotte_2007}{Principe du \lda  \citep{lotte_review_2007}}
	
	% SUPPORT VECTOR MACHINE
	\item \svm (SVM)\\
	Le SVM \citep{boser_training_1992, cortes_support-vector_1995, vladimir1995nature} utilise �galement un hyperplan pour s�parer deux classes. Toutefois, cet hyperplan optimal est trouv� en maximisant les marges (ou distance) entre ce plan et les attributs les plus proches. Le SVM poss�de une particularit�, il utilise un noyau qui peut permettre de r�soudre les probl�mes lin�aire (\textit{linear SVM}) mais �galement les probl�mes non-lin�aire en projetant les donn�es dans un espace de dimension sup�rieure (\textit{kernel trick}). Un noyau que l'on retrouve assez r�guli�rement est le \textit{Radial Basis Function (RBF)} \citep{burges_tutorial_1998}. Les probl�mes multi-classes peuvent �galement �tre trait�s en \textit{One-vs-All}
	\figScaleX{0.4}{SVM_Lotte_2007}{Principe du \svm \citep{lotte_review_2007}}
	
	% K-NEAREST NEIGHBOR
	\item \knn (KNN) \\
	Pour un nouveau point de testing, le KNN \citep{fix1951discriminatory} mesure la distance avec les $k$ plus proches voisins et d�duit la classe de ce point en fonction des classes de ces k-voisins (l'attribution de la classe se fait donc par vote)
	\figScaleX{0.4}{knn}{Principe du \knn \citep{weinberger2005distance}}
	
	% NAIVE BAYES
	\item \nb (NB) \\
	Le NB \citep{fukunaga1990introduction} est un classifieur probabiliste. Une des hypoth�ses du NB est que les donn�es dans les classes doivent �tre normalement distribu�es et ind�pendante.
\end{enumerate}

La figure \ref{comp_clf} en annexe, issue de l'excellentissime librairie python scikit-learn d�di�e au machine learning, illustre le comportement de chaque classifieur fa�e � trois types de donn�es. D'autres informations d�taill�es � propos des classifieurs peuvent �tre trouv�es dans  \cite{lotte_review_2007, wieland_performance_2014, wu_top_2008}
\figScaleX{0.9}{clf_classifier}{Entra�nement puis test d'un classifieur lin�aire}

% ********************************************
%               DECODING ACCURACY
% ********************************************
\subsection{�valuation de la performance de d�codage}
La question qui se pose maintenant, c'est comment �valuer la performance de d�codage. Pour cela, on peut par exemple utiliser le \textit{\da} ou le \textit{roc}
\begin{enumerate}
	\item \da (DA)\\
	L'utilisation (DA) est ce que l'on retrouve le plus fr�quemment. Le calcul est simple, on compare les v�ritables labels avec les labels pr�dits par le classifieur. En faisant la somme des labels correctement pr�dits divis� par le nombre d'essais, on obtient un ratio qui correspond � l'acuit� de d�codage. Le plus souvent, ce ratio est ensuite exprim� en pourcentage. Le taux d'erreurs peut-�tre calcul� en prenant $1-DA$. 
	\figScaleX{0.2}{clf_da}{Calcul de l'acuit� de d�codage}

	\item \roc (ROC) \\
	Une autre m�thode pour �valuer la performance de d�codage est l'utilisation de l'aire sous la courbe (AUC) ROC \citep{ling_auc_2003,huang_using_2005,bradley_use_1997}. Celle-ci prend en compte le nombre d'essais correctement et incorrectement classifi�s et pourrait donc prendre davantage de valeur possible compar� au \da.
\end{enumerate}


% ********************************************
%                STATISTIQUE
% ********************************************
\subsection{Seuil de chance et �valuation statistique de la performance de d�codage}
\label{chance_level_stat}
De mani�re th�orique, le seuil de chance est donn� par $1/c$ o� $c$ est le nombre de classes. Par exemple, un probl�me � quatre classes donne un seuil de chance de $25\%$. Toutefois, ce seuil de chance est atteint pour un nombre de sample $n$ infinis. En pratique, nous travaillons avec un nombre r�duis de donn�es, parfois m�me, avec tr�s peu de sample. Dans ce cas, on peut obtenir des DA tr�s �lev�s qui pourtant, ne sont pas pertinents. Les m�thodes pr�sent�es ci-dessous ont pour but de trouver le seuil de chance associ� � un jeu de donn�e et de trouver pas la m�me occasion, la valeur $p$.
\begin{enumerate}
	\item Loi binomiale \\
	En faisant l'hypoth�se que l'erreur de classification suit une distribution binomiale cumulative, on peut utiliser la loi suivante pour en d�duire la probabilit� de pr�dire au moins $z$ fois la classe $c$:
	\begin{equation}
		P(z)=\sum_{i=z}^{n}
		\begin{pmatrix} n \\ i \end{pmatrix} \times 
		\left(\frac{1}{c}\right)^{i} \times 
		\left(\frac{c-1}{c}\right)^{n-1}
	\end{equation}
	\item Permutation \\
	Les permutations pr�sentent l'avantage d'�tre calcul�es � partir des donn�es (\textit{data driven}).   \cite{ojala_permutation_2010} nous renseigne sur les diff�rents types de permutations possibles dans le cadre du d�codage:
	\begin{enumerate}
		\item \textit{Full permutation}: les donn�es sont m�lang�s
		\item \textit{Shuffle y}: le vecteur de label est m�lang�. C'est la proc�dure la plus fr�quemment rencontr�e.
		\item \textit{Intra-class shuffle}: les donn�es sont m�lang�es � travers la dimension $features$ (colonne) et ce, � l'int�rieur de chaque classe.
	\end{enumerate}
	Autant les m�thodes $(a)$ et $(b)$ nous renseigne v�ritablement sur la consistance d'un d�codage par rapport aux donn�es, autant la m�thode $(c)$ donne des informations un peu diff�rentes. En effet, en cas de d�codage non-significatif, on pourra soit conclure qu'il n'y a pas de consistance dans les attributs � l'int�rieur des classes, soit que le classifieur est incapable d'utiliser cette l'inter-d�pendance. \cite{ojala_permutation_2010} pr�cise que dans ce cas, il n'est pas n�cessaire d'utiliser un classifieur compliqu� et qu'un classifieur simple devrait suffire.
\end{enumerate}

Cette partie est volontairement synth�tique puisqu'elle a fait l'objet d'une publication scientifique (cf. \ref{seuil_chance}). \\
Un pipeline standard de classification est propos� en annexe (cf. \ref{clf_pip}).

% -> Single et Multi-features
\subsection{Du single au multi-features}
% -----------------------------------------------------------------------------
% -----------------------------------------------------------------------------
%                       SINGLE ET MULI-FEATURES
% -----------------------------------------------------------------------------
% -----------------------------------------------------------------------------
Dans les sections pr�c�dentes, nous avons vu comment extraire des attributs de l'activit� neuronale et comment les classifier. C'est ce que l'on appelle le \textit{single feature} (SF), \cad que l'on �value la performance de chaque attribut s�par�ment. Cette approche permet de constituer un set de features pertinents et r�pond � des questions neuro-scientifique. Cette d�marche de SF a donc un but exploratoire. \\
La question que l'on peut maintenant se poser, c'est quelle performance de d�codage puis-je obtenir si je combine ces attributs et dans quel cas est-ce utile? C'est le multi-attributs (ou \textit{multi-features} (MF)). Tout d'abord, le MF est utilis� lorsqu'il y a soit un d�sir soit un besoin de performances accrue. Par exemple, on utilisera le MF dans les comp�titions de d�codage ou tout simplement, pour une BCI o� la performance est essentielle. Si l'on construit un syst�me de bras robotis� pilot� par activit� neuronale, on comprend sans peine que celui-ci doit �tre le plus efficace possible et donc, le MF s'impose. Le dernier cas o� l'on rencontre du MF, et ce n'est pas le cas le plus glorieux, c'est le cas o� il y a un besoin de pallier � des r�sultats de SF assez faibles. La litt�rature expose des \da toujours plus hauts, des m�thodes toujours plus complexes et donc, pour publier correctement un article, il faut avoir des r�sultats au-moins aussi perspicaces. \\
Le \mf c'est donc l'utilisation de multiples attributs pour aboutir � une classification et ce, sans s�lection particuli�re. Individuellement, les attributs d'un m�me set n'auront pas la m�me performance. Certains seront des bons marqueurs et d'autres, n'ajouteront pas ou peu d'information. Donc en combinant ces features, il est probable que l'acuit� de d�codage soit moins bonne que la performance en attribut unique. Pour cela, on pourra donc utiliser des algorithmes de s�lections de marqueurs (\textit{feature selection}). Le but de cette s�lection est de trouver dans un set d'attributs, un sous-ensemble dont la performance group�e est meilleure que la performance individuelle. \\
Cette s�lection est une proc�dure exigeante o� le risque de sur-apprentissage est grand. C'est la raison pour laquelle cette s�lection doit �tre mise � l'int�rieur d'une \cv. Donc on d�finis un set de \train et de \test gr�ce � la validation crois�e, puis sur le \train, on lance la \textit{feature selection}. On aboutit � un sous-ensemble de marqueurs qui va servir � entra�ner le classifieur. Ensuite, on s�lectionne ce subset dans le \test et on test le classifieur avec ce subset. Toute ceci �tant enfin r�p�t� pour chaque \textit{fold} de la \cv. A la vue de cette proc�dure, deux probl�mes �mergent:
\begin{itemize}
	\item La s�lection d'attributs se faisant � l'int�rieur des folds de la \cv, on peut tr�s bien aboutir � des listes d'attributs diff�rentes. Pour obtenir une information finale, on pourra donc parler des attributs les plus fr�quemment choisis. Par exemple, si la s�lection se fait dans un \cv 10-folds, on pourra dire que le feature 1 a �t� choisi 7/10, le feature 2, 3/10... 
	\item En fonction de la s�lection choisie et de la \cv, le pipeline complet peut �tre tr�s (tr�s) lourd et long.
\end{itemize}
\vspace{1\baselineskip}
Les m�canismes de \textit{feature selection} peuvent �tre regroup�s en deux grandes familles \citep{guyon_introduction_2003, liu_classification_2008, das_filters_2001}: les \textit{Filter methods} et les \textit{Wrapper methods}.

\subsubsection{\textit{Filter methods}}
Ces m�thodes sont bas�es sur un crit�re et sont ind�pendantes du classifieur. Parmi elles, on retrouve des outils de corr�lation, d'information mutuelle ou encore de statistiques. Ces derniers outils �valuent la contribution de chaque feature de mani�re ind�pendante sans tenir compte de la corr�lation entre ces features. Pour r�soudre ce probl�me, \cite{yu_redundancy_2004, ding_minimum_2005} introduisent le \textit{minimal-redundancy-maximal-relevance} qui en plus de trouver les features les plus pertinents, va permettre d'�liminer ceux qui sont redondants. \\
Pour terminer, ces m�thodes sont effectivement ind�pendantes de l'algorithme de classification mais elles peuvent s'av�rer optimales pour tel ou tel classifieur (ex: l'utilisation du crit�re de Fisher pour filtrer les features est tr�s performant lorsqu'il est ensuite associ� au \lda \citep{duda2001pattern}).

\subsubsection{\textit{Wrapper methods}}
Contrairement aux m�thodes de filtrage, les \textit{wrapper} utilisent le classifieur comme outil de s�lection. Le premier inconv�nient que l'on peut d'ors et d�j� leur reprocher, c'est que le r�sultat final sera donc classifieur-d�pendant, donc difficile pour la g�n�ralisation. \\
Parmi ces \textit{Wrapper methods}, on peut citer:
\begin{enumerate}
	\item S�lection exhaustive: on teste toutes les combinaisons de features possibles puis on s�lectionne la meilleure. Proc�dure qui ne peut �tre faisable qu'en pr�sence d'un jeu de donn�es particuli�rement restreint.
	\item S�lection sur la statistique de d�codage: on utilise le classifieur pour �valuer l'acuit� de d�codage de chaque feature s�par�ment pour en d�duire une valeur p (cf: \ref{chance_level_stat}). Enfin, on s�lectionne les features dont la valeur p est inf�rieur � un seuil d�sir�.
	\item S�lection s�quentielle: processus o� l'on va ajouter/enlever des features de mani�re s�quentielle jusqu'� atteindre un d�codage optimal. Ce type de s�lection se fait suivant deux directions:
	\begin{enumerate}
		\item \ffs (FFS): la premi�re �tape consiste � �valuer la performance de chaque attributs. On s�lectionne le meilleur que l'on va ensuite combiner en couple avec tout les features restant. On s�lectionne le meilleur couple puis on teste les combinaisons des meilleures triplettes... On continu tant que la performance s'am�liore. Si le DA d'une �tape $i$ est inf�rieur au DA de l'�tape $i-1$, on consid�re le nouveau subset de features � $i-1$.
		\figScaleX{1}{clf_forward}{Exemple d'une \ffs appliqu�e sur six features}
		\begin{center} \rule{5cm}{0.2pt} \end{center}
		
		\item \bfe (BFE): la philosophie est la m�me que pour un \textit{forward}. On classifie d'abord les $N$ features pris ensemble, puis on enl�ve � tour de r�le chaque marqueur. On s�lectionne le subset compos� de $N-1$ features ayant fournit le meilleur r�sultat, puis on enl�ve de nouveau chaque feature... L'algorithme s'arr�te de la m�me fa�on que le \textit{forward}.  
		\figScaleX{1}{clf_backward}{Exemple d'une \bfe appliqu�e sur six features}
		\begin{center} \rule{5cm}{0.2pt} \end{center}
	\end{enumerate}
	De mani�re g�n�rale, il est rapport� que la FFS converge plus rapidement que la BFE \citep{guyon_introduction_2003}. Toutefois, la FFS tombe plus facilement dans des minimums locaux et donc, m�ne � un d�codage moins bon. En effet, la \textit{forward} s�lectionne pas-�-pas les meilleurs attributs, elle est donc moins ensembliste que la \textit{backward}.
\end{enumerate}

\vspace{1\baselineskip}
Les m�thodes de filtrage demandent moins de ressources et repr�sentent donc un premier choix pour les larges sets de donn�es. En revanche, elles peuvent ne pas d�celer les ph�nom�nes de compl�mentarit� entre features. Pour cette derni�re raison, les m�thodes de wrapper fournissent en g�n�ral de meilleurs r�sultats \citep{chai_evaluation_2004}.


% -> G�n�ralisation temporelle
\subsection{G�n�ralisation temporelle}
L'introduction du \textit{single-feature} faite plus haut �tait une pr�sentation g�n�rique, \cad que celle-ci est vraie quelque soit les features �tudi�s. On pourra donc classifier des attributs de puissance, de PAC, de phase, d'entropie... On peut �galement envisage l'�tude un seul marqueur mais dans sa dimension temporelle. En effet, cela consiste � entra�ner et tester un classifieur � diff�rents instants temporels pour voir si le d�codage varie dans le temps. Une des limitations de cette utilisation d'un classifieur est que, � chaque instant, celui-ci change. Donc on ne peut inf�rer aucune g�n�ralisation. Pour envisager une g�n�ralisation, il faut entra�ner le classifieur � un instant puis le tester � travers toute la dimension temporelle restante. Dans ce cas, on pourra parler de g�n�ralisation mais reste encore le probl�me du choix de l'instant temporel qui servira � entra�ner le pr�dicateur.\\
\figScaleX{0.55}{waldert_timeresolved}{Exemple de d�codage temporel \citep{waldert_hand_2008}. Ici, l'auteur d�code 4-directions de mouvements de la main dans le temps. A chaque instant, un classifieur est cr��, entra�n� puis test� � ce m�me instant.}

Pour r�pondre � cette question, \cite{king_characterizing_2014} introduisent une id�e particuli�rement esth�tique visant � g�n�raliser le comportement d'un classifieur � travers le temps. Sans trop de surprise, ils ont nomm� cette m�thode la \textit{g�n�ralisation temporelle}. Elle permet de r�pondre � deux limitations:
\begin{itemize}
	\item Comment g�n�raliser le comportement d'un classifieur lors d'une �tude temporelle?
	\item Comment choisir l'instant qui servira � entra�ner le classifieur et quel impact ce choix aura-t-il sur le reste du d�codage temporel? 
\end{itemize}
La m�thodologie consiste � prendre un instant $i$, entra�ner le classifieur et tester celui-ci sur tout les instants. Puis, on prend l'instant suivant $i+1$, on entra�ne un nouveau classifieur et on teste... Et on r�p�te cette proc�dure pour tout les instants. On obtient ainsi une repr�sentation $2D$ o�, par convention, l'axe des ordonn�es mat�rialise l'endroit o� le classifieur a �t� entra�n� (\textit{Training time}) et l'axe des abscisses pour tester ce pr�dicateur sur le reste de la dimension temporelle (\textit{Generalization time}). La couleur permettra de signaler la performance de d�codage.

\figScaleX{0.5}{king_timegene}{Exemple de g�n�ralisation temporelle \citep{king_characterizing_2014}}

\todo[color=orange!100]{Il y a aussi des interpr�tations neuro-scientiques possibles. Comme le fait d'entra�ner dans une phase et de tester dans une autre. Si �a marche, �a veut aussi dire que ces phases partagent des corr�lats neuronaux. A voir si j'inclue ceci.}
%D'un point de vue neuro-scientifique, la \textit{g�n�ralisation temporelle} apporte �galement son lot d'informations compl�mentaires, m�me si, l'interpr�tation n'est pas toujours �vidente. Si l'on imagine t�che au cours de laquelle se succ�dent des phases (eg. Phase de repos/pr�paration motrice/ex�cution motrice). Ce n'est pas tr�s �tonnant si on entra�ne


% ********************************************
%              CONCLUSION
% ********************************************
\section{Configuration pour d�buter}
Dans la jungle des m�thodes, il peut parfois �tre difficile de s'y retrouver. Cette section a pour objectif de fournir une liste de m�thodes conseill�es pour d�buter. Rien ne dit que ce sont les meilleures m�thodes mais elles ont le m�rite d'avoir fait leurs preuves, que ce soit dans cette th�se et surtout, dans la litt�rature. Gardons � l'esprit que les meilleures m�thodes d�pendent des donn�es mais certaines, sont plus polyvalentes.

\vspace{1\baselineskip}

\todo[inline,caption={},color=blue!20]{
	\textbf{Couplage phase-amplitude :} \\
	\kld avec swapping des essais de phase et d'amplitude \citep{tort_measuring_2010}
}

\vspace{1\baselineskip}

\todo[inline,caption={},color=green!20]{
	\textbf{Classification :} \\
	\begin{enumerate}
		\item Cross-validation: 10-folds
		\item Classifieur: \svm avec noyau lin�aire \citep{vladimir1995nature,lotte_review_2007}
		\item �valuation statistique: permutations \citep{ojala_permutation_2010,combrisson_exceeding_2015}
		\item Multi-features: \bfe \citep{guyon_introduction_2003}
	\end{enumerate}
}

% ********************************************
%              BRAINPIPE
% ********************************************
\section{Impl�mentation des m�thodes pr�sent�es}
A force d'impl�menter diff�rentes m�thodes afin de les tester, il se trouve que cet amas de fonction a form� assez naturellement une \textit{toolbox}. Avec un peu de mise en forme, une uniformisation � travers les fonctions et la constitution d'une aide, nous avons d�cid� de mettre cet ensemble d'outils \textit{online} et en libre acc�s. \\
La toolbox s'appelle \brainpipe, enti�rement cod�e en Python. Le choix de python a �galement �t� m�rement r�fl�chis. La premi�re version �tait en Matlab. Toutefois, Python pr�sente les avantages d'�tre libre, beaucoup plus souple et intuitif et bien plus puissant. De plus, la communaut� s'est appropri�e ce langage et est particuli�rement r�active. Quant � la documentation, elle est excellente et permet d'apprivoiser ce langage tr�s rapidement. Autre point fort qui est une cons�quence du libre, c'est la multitudes d'outils existants. Par exemple, pour la classification, il existe \textit{scikit-leanr} dont la qualit� est sans comparaison possible avec Matlab. Enfin, les derniers outils � la pointe (tel que \textit{tensorflow} de Google) ne seront jamais propos� en Matlab. Le Julia (qui est un autre langage de programmation) semble particuli�rement prometteur mais, �tant tr�s jeune et ne disposant que d'une petite communaut�, il a �t� �cart�. Autre point, les outils que nous avons pr�sent� dans cette partie m�thode peuvent n�cessiter �norm�ment de ressources. \brainpipe a donc enti�rement �t� d�velopp� pour le calcul parall�le qui est int�gr� par d�faut. A titre de comparaison, le calcul du PAC sur tout un jeux de donn�es a pris une semaine Matlab, contre une journ�e en Python.

\vspace{1\baselineskip}
\brainpipe est d�di�e � la classification de signaux c�r�braux. La construction globale peut �tre r�sum�e avec les points suivants:
\begin{itemize}
	\item \textit{Pre-processing}: pr�-traitement de donn�es (filtrage, re-r�f�rencement, informations anatomiques en fonction de coordonn�es MNI...)
	\item \textit{Features}: extraction d'une large vari�t� de features (amplitude, puissance, phase, PAC, PLV, ERPAC, entropie...)
	\item \textit{Classification} des signatures c�r�brales: repose sur \textit{scikit-learn} pour l'impl�mentation des algorithmes de classification. \brainpipe sert � faire le lien entre l'analyse de donn�es neuro-scientifiques et le \textit{machine-learning}. Il y a �galement un module d�di� au \textit{multi-features}.
	\item \textit{Statistics}: tout les attributs disposent des m�thodes statistiques ainsi que pour la classification et la correction multiple. 
	\item \textit{Visualization}: enfin, un ensemble d'outils de visualisation sont �galement disponible afin de faciliter les repr�sentations graphiques.
\end{itemize}

\vspace{1\baselineskip}
Liens vers \brainpipe:
\begin{itemize}
	\item T�l�chargement: \url{https://github.com/EtienneCmb/brainpipe}
	\item Documentation en ligne: \url{https://etiennecmb.github.io/}
\end{itemize}


\begin{enumerate}
	\item \textit{ipywksp}: pour les personnes d�sirant ou utilisant d�j� python, un environnement de d�veloppement particuli�rement agr�able et puissant s'impose de plus en plus. Ce projet s'appelle 
	\href{http://jupyter.org/}{Jupyter}. Pour l'introduire bri�vement, c'est un notebook dans lequel le code est int�gr� dans des cellules qui peuvent �tre lanc�es individuellement. Une des lacunes de ce notebook pour les utilisateurs Matlab, c'est le manque d'un \textit{workspace} pour voir les variables en cours. \textit{ipywksp} est un petit module, �galement d�velopp� dans cette th�se qui vient combler ce manque. Toujours en libre acc�s: \url{https://github.com/EtienneCmb/ipywksp}  
	\item \textit{visbrain}: enfin, un dernier module Python qui est en cours de d�veloppement mais qui proposera une interface graphique pour toute visualisation n�cessitant un cerveau MNI $3D$ (connectivit�, projection corticale...). Il existe de nombreux logiciels existants d�j�. Le but de celui-ci sera de permettre une int�gration totale avec \brainpipe. Le projet d�bute, mais le lien suivant permet de suivre son �volution: \url{https://github.com/EtienneCmb/visbrain} 
\end{enumerate}


   % M�thodologie
% #############################################################################
%                             DONNEES EXPERIMENTALES
% #############################################################################
\chapter{Donn�es exp�rimentales}
Durant cette th�se, l'exploration s'est faite chez l'Homme par le biais, principalement, de donn�es de type \seeg (SEEG). Ces donn�es rares de tr�s grande qualit� (cf. \ref{seeg_av-in}) ont �t� acquise avant le d�but de la th�se, ce qui a permis de rentrer dans le vif du sujet tr�s rapidement, apr�s une p�riode d'acclimatation aux diff�rents traitements, propres � ce type d'enregistrement. D'autres types tel que l'EEG, la MEG ou les micro-�lectrodes ont �galement �t� approch�s mais de mani�re ponctuelle, comme ce f�t le cas dans l'�tude $1$ (cf. \ref{seuil_chance}) ou dans diff�rentes collaborations. Toutefois, �tant donn� que le temps consacr� � ces donn�es ne repr�sente qu'une faible portion du travail total, nous allons ici nous concentrer uniquement sur l'intra.

\vspace{1\baselineskip}
Pour commencer, nous verrons ce que l'analyse de la \seeg a de particulier (richesse des donn�es, qualit�, les traitements associ�s, les avantages et les limitations). Enfin, nous verrons concr�tement les enregistrements qui ont �t� utilis�s dans le cadre de cette th�se.

% -----------------------------------------------------------------------------
% -----------------------------------------------------------------------------
%                     DONNEES INTRA
% -----------------------------------------------------------------------------
% -----------------------------------------------------------------------------
\section{Donn�es intracr�niennes}

% -> Acquisition des donn�es
\subsection{Acquisition}
La premi�re question que l'on est en droit de se poser, c'est comment est-il possible de travailler, chez l'Homme, avec des enregistrements qui n�cessitent une implantation invasive, \cad dans le cortex? Certaines personnes pr�sentent des formes agressives d'�pilepsies, pouvant s'av�rer pharmacor�sistantes. En fonction de la localisation du foyer �pileptog�ne, les m�faits engendr�s par les d�charges �pileptiques peuvent �tre vari�s. Dans ce cas, il est n�cessaire de localiser ce foyer avec, de pr�f�rence, des techniques non-invasives telles que l'EEG ou la MEG. Mais si ces derni�res ne permettent pas une localisation pr�cise le patient sera implant� avec des macro-�lectrodes comme la SEEG pour tenter de localiser puis d'enlever ce foyer par intervention chirurgicale. Cette implantation a un second objectif, d�terminer quel est le r�le fonctionnel de la structure l�s�e (r�le moteur, langage, vision...). C'est dans ce contexte que les chercheurs proposent au patient de participer � une �tude scientifique.

% -> Avantages et limitations
\subsection{Avantages et limitations des donn�es intracr�niennes}
\label{seeg_av-in}
Le paragraphe pr�c�dent met en exergue la raret� de ces donn�es. De plus, ce type d'acquisition enregistre l'activit� c�r�brale d'une population relativement restreinte de neurones. En cons�quence, on peut esp�rer que ce petit groupe s'active dans des processus pr�cis et ainsi, �tudier des ph�nom�nes fins. Enfin, le rapport signal sur bruit (RSB) de la SEEG est excellent, ce qui doit permettre l'�tude de processus, m�me en essais unique l� o� d'autres types d'enregistrements auront besoin d'une large banque d'essais avant de pouvoir constater l?�mergence d'un ph�nom�ne. \\
La SEEG pr�sente toutefois quelques limitations que l'on peut nuancer. Le probl�me majeur est certainement la g�n�ralisation d'un ph�nom�ne ou la reproductibilit� � travers les sujets. La pathologie est propre � chaque patient, donc son implantation aussi. Ce qui signifie qu'il n'y a aucune chance que plusieurs sujets pr�sentent rigoureusement la m�me implantation. Pour contourner cette limitation, ou pourra utiliser:
\begin{itemize}
	\item Des r�gions d'int�r�t (ROI): on va regrouper les �lectrodes des diff�rents sujets par "proximit�" en faisant l'hypoth�se que celles-ci s'activent de fa�on similaire face � un processus. Ces ROI pourront �tre par exemple les gyrus ou les aires de Brodmann. Bien s�r, ce que l'on gagne on g�n�ralisation, on le perd en pr�cision.
	\item Projection corticale: autour de chaque �lectrode, on d�finit une sph�re d'int�r�t (g�n�ralement 10 millim�tres de rayon) puis on prend l'intersection de ces sph�res avec la surface du cerveau. Cette technique permet une visualisation des activit�s proches de la surface � travers les sujets mais on perd de la lisibilit� sur ce qui se passe en profondeur.
\end{itemize}
Une utilisation combin�e des ROI et de la projection corticale permet de palier, au moins partiellement, au probl�me de reproductibilit� inter-sujets. \\
Autre limitation, on ne dispose que d'une couverture partielle puisque le neuro-chirurgien implante une quantit� limit�e d'�lectrodes. Ce dernier point est trait� en augmentant le nombre de sujets. Enfin, la derni�re limitation que l'on soul�vera ici, concerne le fait de travailler sur un cerveau "malade" emp�chant donc une g�n�ralisation � des sujets sains. On limite ce probl�me par un ensemble de pr�traitements \citep{jerbi_task-related_2009} d�cris dans le prochain paragraphe. \\
Un dernier point que l'on peut argumenter � la fois comme avantage ou limitation, c'est de ne pas pouvoir contr�ler l'implantation pour �tudier un ph�nom�ne pr�cis. Par exemple, si l'on analyse l'encodage moteur, on s'attendrait � concentrer les efforts sur le cortex moteur primaire ou pr�-moteur. Or l'implantation SEEG peut tr�s bien contenir du frontal, du pari�tal, du temporal. L� o� finalement on peut consid�rer �a comme un avantage, c'est que l'on a acc�s � un ensemble de structures jug�es non-primordiales mais dont l'ajout pourrait permettre la compr�hension d'un processus de mani�re plus globale.

% -> Pr�-processing
\subsection{Pr�traitements}
Le premier pr�traitement appliqu� a �t� une r�jection des sites bruit�s ou pr�sentant une activit� pathologique, \cad des d�charges �pileptiques. C'est par cette inspection manuelle que l'on augmente le potentiel de g�n�ralisation aux sujets sains. \\
Autre pr�traitement, les donn�es peuvent �tre bipolaris�es comme c'est le cas dans de nombreuses �tudes \citep{bastin_direct_2016, ossandon_transient_2011, jerbi_task-related_2009}. La bipolarisation part du principe que deux sites proches enregistrent des activit�s neuronales diff�rentes mais que toute source de bruit, ou influence de sources lointaines, se retrouvera sur ces deux sites. La technique de bipolarisation consiste donc � soustraire les activit�s neuronales de sites proches ce qui a pour effet de supprimer la partie commune, le bruit. Par exemple prenons une �lectrode contenant les sites $k9$, $k10$, $k11$ et $k12$. Apr�s bipolarisation, on consid�rera les sites mat�rialis�s par $k10-k9$, $k11-k10$ et $k12-k11$. Les b�n�fices de la bipolarisation peuvent �tre r�sum�s par:
\begin{enumerate}
	\item Limitation des influences des sources lointaines et de la tension secteur (50hz)
	\item Augmentation de la sp�cificit� qui, pour un site bipolaris�, est estim�e � 3mm \citep{kahane_bancaud_2006, lachaux_intracranial_2003, jerbi_task-related_2009}
\end{enumerate} 

% -----------------------------------------------------------------------------
% -----------------------------------------------------------------------------
%                            DONNEES CENTER OUT
% -----------------------------------------------------------------------------
% -----------------------------------------------------------------------------
\section{Donn�es d'�tude}
Trois jeux de donn�es intracr�niaux ont �t� explor�:
\begin{enumerate}
	\item \co: �tude de l'encodage et du d�codage des actions et des intentions motrices lors de mouvements de main
	\item \textit{Occulo}: �tude des intentions et d�cision de mouvements oculaires
	\item \textit{Emotions}: �tude de l'encodage des �motions
\end{enumerate}

% -> Donn�es Center-out
\subsection{Donn�es \co}
C'est le jeu de donn�es qui a �t� le plus largement exploit�. En effet, celui-ci a servis � �tudier l'encodage (cf. \ref{Etude2_encodage}) et le d�codage (cf. \ref{Etude3_decodage}) des actions motrices chez l'homme. 

% -> Implantation
\subsubsection{Descriptif des donn�es}
Six sujets (six femmes), implant�s au d�partement de l'�pilepsie de l'h�pital de Grenoble ont donn� leur consentement �crit pour passer l'exp�rience, sous la supervision du personnel m�dical. Le tableau \ref{subject_table} r�sume les d�tails clinique des diff�rents sujets. \\

\begin{figure}[H]
	\begin{center}
		\begin{tabular}{|c|c|c|c|c|}
		  \hline
		   & Dominance & Age & Genre & Zone �pileptique \\
		   \hline
		   P1 & D & 19 & F & Frontal (RH) \\
		   \hline
		   P2 & D & 23 & F & Frontal (LH) \\
		   \hline
		   P3 & D & 18 & F & Frontal (RH) \\
		   \hline
		   P4 & D & 18 & F & Frontal (RH) \\
		   \hline
		   P5 & D & 31 & F & Insula (RH) \\
		   \hline
		   P6 & D & 24 & F & Frontal (LH) \\
		   \hline
		  \multicolumn{5}{|l|}{Moyenne: $22.17 \pm 4.6$}\\
		  \hline
		\end{tabular}
	\end{center}
	\caption{D�tails cliniques des sujets ayant particip� � la t�che \co}
\end{figure}
\label{subject_table}

% -> Mat�riel d'acquisition
\subsubsection{Mat�riel d'acquisition}
De 12 � 15 multi-�lectrodes ont �t� implant�es dans diff�rentes structures. Chaque multi-�lectrode poss�de entre 10 et 15 sites mesurant 0.8mm et s�par�s de 1.5mm. La localisation anatomique des �lectrodes s'est faite en utilisant le sch�ma d'implantation (exemple en annexe \ref{schema_implantation}) et l'atlas proportionnel de Talairach et Tournoux \citep{talairach_referentially_1993}. La visualisation de la pr�-implantation s'est faite par $IRM-3D$ et un $CT-scan$ a �t� utilis� pour la post-implantation. Enfin, un $IRM$ a �galement servis pour visualiser les �lectrodes implant�es dans la mati�re blanche. Les coordonn�es Talairach ont �t� d�duites du $CT-scan$ puis ont �t� transform�es en MNI afin de pouvoir les superposer dans un cerveau standard (cf. figure ci-dessous).\\
Le syst�me Micromed a �t� utilis� pour visionner l'acquisition de l'activit� neuronale. Une �lectrode prise dans la mati�re blanche a �t� prise comme r�f�rence et un filtrage \textit{passband} entre $[0.1, 200hz]$ a �galement �t� effectu� \textit{online}. La fr�quence d'�chantillonnage est de $1024hz$.

\figScaleX{1}{data_Cover}{Implantation intracr�niale et couverture corticale de six sujets �pileptiques ayant pass�s la t�che \co}


% -> Descriptif de la t�che
\subsubsection{Descriptif de la t�che}
La t�che est compos�e de trois phases:
\begin{enumerate}
	\item Phase de repos: on demande au sujet de rester immobile pendant une dur�e de une seconde
	\item Phase de pr�paration motrice ($\textbf{CUE 1}$): une direction est impos�e � l'�cran (haut/bas/gauche/droite). On demande au sujet de se pr�parer pendant 1.5s � bouger la souris dans la direction impos�e.
	\item Phase d'ex�cution motrice ($\textbf{CUE 2}$): le sujet ex�cute le mouvement en bougeant la souris du centre � l'extr�mit� de l'�cran indiqu�e (environ 1.5s) puis de revient vers le centre.
\end{enumerate}
\figScaleX{1}{data_Task}{Descriptif de la t�che \co}

Cette t�che � conduit � deux �tudes essentielles:
\begin{itemize}
	\item L'�tude de l'encodage des intentions motrices (cf. \ref{Etude2_encodage}): on �tudiera le d�codage du repos vs pr�paration, repos vs ex�cution et de la pr�paration vs ex�cution
	\item L'�tude du d�codage des directions de mouvement (cf. \ref{Etude3_decodage}) que ce soit pendant la pr�paration ou pendant l'ex�cution motrice.
\end{itemize}

% -----------------------------------------------------------------------------
% -----------------------------------------------------------------------------
%                                A VOIR
% -----------------------------------------------------------------------------
% -----------------------------------------------------------------------------
\subsection{Autres donn�es}
\todo[inline,caption={},color=red!20]{
  \begin{itemize}
	\item Occulo: Donn�es occulo mais un seul sujet donc pas super cool. De plus, les r�sultats sont un peu vieux et m�riteraient de se pencher dessus une fois pour toute.
	\item Emotions: Ce serait pas mal que j'ai quelques r�sultats sur les �motions, histoire de montrer aussi un peu la diversit� des donn�es utilis�es.
  \end{itemize}
}



      % Donn�es exp�rimentales


% #############################################################################
%                                   OUVERTURE
% #############################################################################
\chapter{ouverture}
Nos contributions portent sur : \dots \\*

Le \emph{premier chapitre} expose  la probl�matique de la th�se.

Le \emph{deuxi�me chapitre} pr�sente  en d�tail \dots \\*

etc.

Cette th�se  a fait l'objet de  divers travaux �crits : \dots

