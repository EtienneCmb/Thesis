% #############################################################################
%                           OBJECTIFS DE THESE
% #############################################################################
\chapter{Objectifs de la th�se}

% -----------------------------------------------------------------------------
\section{D�codage c�r�brale � partir d'activit� intracr�nienne}
Expliquer que le terme motif utilis� en intro se transforme ici en feature ou attribut. Ce qui nous permet d'effectuer une succession de mouvements identiques c'est que il y a dans l'AN des pattern permettant de reproduire des mouvements. D'ailleurs, c'est gr�ce � �a que l'on peut se perfectionner dans un sport ou en zik car il y a apprentissage.
\begin{itemize}
\item Exemple d'un sch�ma d'implantation + IRM 
\item Bipolarisation: d�bruitage et augmentation de la sp�cificit� (article Karim)
\item Extraction de features (ici on pourrait mentionner que le deep learning pourrait marcher sur les donn�es brutes)
\end{itemize}

% -----------------------------------------------------------------------------
% Features
\section{Exploration et am�lioration des features}
\subsubsection{R�le physiologique du \pacEN}
\citep{hyafil_neural_2015}
\figScale{hyafil_2015}{M�canismes du couplage phase-amplitude}


% -----------------------------------------------------------------------------
\section{Comparatif des classifieurs}
Expliquer que, chaque classifier poss�de une m�thodologie propre permettant de r�pondre � des types de donn�es diff�rentes (en fonction des hypoth�ses de fonctionnement de chacun des classifiers)

% -----------------------------------------------------------------------------
\section{Exploration des r�gions non-motrices}
\citep{van_langhenhove_interfaces_2008}
\figScale{langhenhove_2008}{Localisation des aires sensorimotrices}