% #############################################################################
%                        PRESENTATION DE LA THEMATIQUE
% #############################################################################
\chapter{Pr�sentation de la th�matique}

% -----------------------------------------------------------------------------
% Enregistrement de l'activit� neuronale
\section{Enregistrement de l'activit� neuronale}

\begin{itemize}
\item Pr�sentation de chacun des types de donn�es
\item R�solution et RSB
\item Avantages // inconv�nients
\end{itemize}

\citep{waldert_review_2009}
\figScale{Waldert_2009_recording}{Techniques d'enregistrement de l'activit� c�r�brale}

\subsection{Enregistrement non-invasif}
\subsubsection{\eeg}
\subsubsection{\meg}

\subsection{Enregistrement invasif}
\subsubsection{\sua}
\subsubsection{\mua}
\subsubsection{\seeg}
\subsubsection{\ecog}

% -----------------------------------------------------------------------------
% Etat de l'art
\section{�tat de l'art des interfaces cerveau-machine}
\citep{bekaert_les_2009}
\figScale{bekaert_2009_ICM}{Pipeline g�n�ral d'un \icm}

\subsection{ICM non-invasives}
P300 speller

\subsection{ICM invasives}
\citep{hochberg_reach_2012}
\figScale{hochberg_2012}{Contr�le d'un bras robotis�}

% -----------------------------------------------------------------------------
\section{Apprentissage machine: applications aux neurosciences}
okok

% -----------------------------------------------------------------------------
\section{Encodage et d�codage moteur: bases physiologiques}
La figure \ref{fig:hanakawa_2008}   repr�sente  une   trajectoire  d'un
processus markovien de saut, avec les notations associ�es.
\citep{hanakawa_motor_2008}
\figScale{hanakawa_2008}{Comparatif}

% -----------------------------------------------------------------------------
\section{Intention et ex�cution}
okok

% -----------------------------------------------------------------------------
\section{Delayed task: protocole exp�rimental}
okok