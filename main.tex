\documentclass[sommairechap,stylejchiquet]{these_gi}



\begin{document}
 
% ==================================================================
% OPTIONS D'AFFICHAGE
% non-d�finitif (soumis aux rapporteurs) ou  d�finitif
\definitiftrue
% \definitiffalse
 
% ==================================================================
% RENSEIGNEMENTS SUR LA TH�SE
\titleFR{D�codage des intentions et des repr�sentations motrices chez l'homme: analyse multi-�chelle et application aux interfaces cerveau-machine}
\titleEN{Le titre en anglais}
\abstractFR{Le r�sum� en fran�ais ($\approx$ 1000 caract�res)}
\abstractEN{Le r�sum� en anglais ($\approx$ 1000 caract�res)}
\keywordsFR{Les mots-cl�s en fran�ais}
\keywordsEN{Les mots-cl�s en anglais}
 
\author{Etienne Combrisson}
\address{e.combrisson@gmail.com}
\universite{universit� claude bernard lyon 1}
\laboratoire{}
\specialite{sp�cialit� de la th�se}
\datesoutenance{09/2016}
\datesoumission{la date de soumission aux rapporteurs}
\jury{\begin{tabular}{llll}
    M\up{me} & \textsc{Erika Rat�} & Uiversit� � la Menthe & (Rapporteur) \\
    M. & \textsc{Jacques Ouille} & Universit� � la Fraise & (Rapporteur) \\
    M. & \textsc{Henri Zoto} & Laboratoire laborieux & (Rapporteur) \\
    M. & \textsc{Jean File} & UTC & (Directeur) \\
       & etc. &  \\
  \end{tabular}    
}
 
% ==================================================================
% D�DICACE
\dedicate{� Isabelle et Didier, mes deux parents,\\ qui ont tout donn� pour que ceci me soit un jour possible. \\ Merci}
 
% ==================================================================
% DEBUT DE LA PR�FACE
\beforepreface
 
% remerciements
\include{Remerc/remerc}
 
% table des mati�res g�n�rale
\tableofcontents
 
% affiche la liste des figures
\newpage
\listoffigures

% ==================================================================
\afterpreface

% ==================================================================
% NOTATIONS
\chapter*{Notations}
\addcontentsline{toc}{chapter}{Notations}

\pagestyle{plain}

%----------------------------------
% GENERAL
%----------------------------------
\Large G\'en\'eral \\% \bigskip
\normalsize
\begin{supertabular}{ll}
  ICM & Interface Cerveau-Machine \\
  BCI & Brain Computer Interface \\
\end{supertabular}

%----------------------------------
% ENREGISTREMENTS
%----------------------------------
\vspace{1cm}
\Large Enregistrements \\% \bigskip
\normalsize
\begin{supertabular}{ll}
  EEG & \eeg \\
  MEG & \meg \\
  SUA & \sua \\
  MUA & \mua \\
  SEEG & \seeg \\
  ECoG & \ecog \\
\end{supertabular}

%----------------------------------
% FEATURES
%----------------------------------
\vspace{1cm}
\Large Features \\% \bigskip
\normalsize
\begin{supertabular}{ll}
  PAC & Phase Amplitude Coupling \\
\end{supertabular}

%----------------------------------
% CLASSIFIEURS
%----------------------------------
\vspace{1cm}
\Large Classifieurs \\% \bigskip
\normalsize
\begin{supertabular}{ll}
  LDA & \lda \\
  SVM & \svm \\
  RF & \rf \\
  KNN & \knn \\
  NB & \nb \\
\end{supertabular}

\chapterend

% ==================================================================
% AVANT-PROPOS
% *****************************************************************************
% *****************************************************************************
%                                   INTRODUCTION
% *****************************************************************************
% *****************************************************************************
\part{Introduction g�n�rale}
\pagestyle{headings}

\malettrine{B}{lablablabkiblablou}  INTRO \dots\\*

% Le sujet de la th�se
L'objectif de cette th�se a  �t� de \dots\\*

Totalit� des m�thodes explor�es durant ma th�se sont pr�sentes dans une toolbox python appel�e brainpipe, libre d'acc�s et de droit. 


% #############################################################################
%                                CORPS DE L'INTRO
% #############################################################################
% #############################################################################
%                        PRESENTATION DE LA THEMATIQUE
% #############################################################################
\chapter{Pr�sentation de la th�matique}

En 1964, Dr. Grey Walter connecte des �lectrodes directement dans le cortex moteur d'un patient et lui demande de presser un bouton pour faire avancer un r�tro-projecteur. En m�me temps, il enregistre l'activit� neuronale de telle sorte que elle aussi, puisse le faire avancer. L� o� l'exp�rience devient remarquable, c'est que le r�tro-projecteur avance avant que le patient ne presse le bouton ! Tout l'appareil musculaire du sujet est court-circuit� et le contr�le se fait sans mouvement. Contr�ler par la \textit{pens�e}, un sujet de science fiction qui devient une r�alit�.\\
Cette anecdote d�crite par \cite{graimann_braincomputer_2009}, permet de placer la naissance des \icm (ICM) dans l'histoire. C'est le point d'entr�e qui a ensuite conduit une grande diversit� de chercheurs � se passionner pour ce sujet. \\

% -----------------------------------------------------------------------------
% -> D�finition d'une ICM
\section{\icm: d�finition et objectifs}

\subsection{Interactions naturelles avec l'environnement}
Pour interagir avec son environnement, l'individu se sert des voies de communications naturelles, \cad via son syst�me nerveux et musculaire. Le processus de communication d�bute par une intention qui active certaines r�gions dans le cerveau. Il en r�sulte un signal c�r�brale qui est ensuite envoy� par le syst�me nerveux p�riph�rique en directions des muscles. C'est ce processus simplifi� qui permet � une personne d'interagir avec ce qui l'entoure.

\subsection{ICM: une communication alternative}
Une ICM (ou BCI en anglais pour \textit{Brain Computer Interface}) est un autre syst�me de communication o� les voies naturelles sont cout-circuit�es. Au lieu de passer par le syst�me nerveux puis musculaire, le signal c�r�bral est directement intercept� au niveau du cerveau et va ensuite �tre transform� en commandes. Une ICM est donc un syst�me permettant de traduire une activit� neuronale en commande ext�rieure. Le terme \textit{traduire} est � prendre au sens linguistique \cad que les signaux c�r�braux forment un langage, compos� de r�gles, de motifs ou \textit{pattern}, que l'on va essay� de d�coder (via un ordinateur) pour les transformer en op�rations. D'o� le terme "\icm".

\subsection{Composantes d'une ICM}
\cite{pfurtscheller_rehabilitation_2008} et \citep{graimann_braincomputer_2009} introduisent quatre �l�ments qui composent une ICM:
\begin{enumerate}
	\item Enregistrer l'activit� directement depuis le cerveau. Cet enregistrement pourra �tre invasif ou non-invasif (cf. \ref{invasif_non-invasif})
	\item G�n�rer un retour ou \textit{feedback} pour l'utilisateur
	\item L'enregistrement et le \textit{feedback} doivent �tre en temps r�el
	\item Enfin, l'interface doit �tre contr�lable par l'utilisateur, de mani�re active, via un ensemble d'intentions. 
\end{enumerate}
A titre d'exemple et pour illustrer ce dernier point, un utilisateur pourrait par exemple d�cider de bouger un curseur de souris sur un �cran en imaginant des mouvements soit de la main gauche soit de la main droite.

\figScaleDesciption{0.7}{Pfurtscheller2008_SchemaICM}{Sch�ma d'une \icm \citep{pfurtscheller_rehabilitation_2008}}{L'activit� neuronale est enregistr�e (Signal acquisition) puis nettoy�e (Preprocessing). Ensuite, on extrait des motifs ou patterns qui caract�risent la commande que souhaite envoyer le sujet (Feature extraction). Enfin, la machine tente de reconna�tre ces motifs (Classification) et de les transformer en commande (Application interface) \citep{pfurtscheller_rehabilitation_2008}}

\subsection{Acquisition de l'activit� neuronale}
L'acquisition de l'activit� neuronale constitue la premi�re �tape d'une ICM. Deux param�tres permettent de d�crire ces diff�rentes techniques:
\begin{enumerate}
	\item L'accessibilit�: \cad si cette technique est plus ou moins facile � mettre en place. 
	\item La qualit�: \cad le rapport signal sur bruit (RSB)
\end{enumerate}
Si la technique d'enregistrement n'est pas facilement accessible, cela pourrait limiter le nombre de sujets potentiel pour l'�tude. La qualit� quant � elle va permettre d'avoir acc�s � des signaux c�r�braux de plus ou moins grande qualit� ce qui va influer sur la performance et sur les limitations d'une ICM. \\
Les diff�rents types d'acquisition peuvent �tre diff�renci�s par leur degr�s de p�n�tration dans le corps. Les enregistrements dits \textit{invasifs} vont n�cessiter une intervention chirurgicale mais donnent acc�s � des signaux de grande qualit�. Les enregistrements \textit{non-invasifs} sont globalement plus faciles � mettre en place car ne n�cessitant aucune chirurgie mais la qualit� du signal est moindre car l'activit� neuronale est enregistr�e en dehors de la bo�te cr�nienne. \\
Au sein de ces deux cat�gories, on trouvera un ensemble de m�thodes d'acquisition qui se diff�rencient par la taille des populations de neurones qu'elles enregistrent ou par le type de signal.

\subsubsection{Enregistrements invasifs}
On parlera d'intracr�nien pour les enregistrements � l'int�rieur de la bo�te cr�nienne puis d'intracortical pour des �lectrodes implant�es dans le cortex \citep{jerbi_task-related_2009, engel_invasive_2005}. �tant donn� que les techniques ci-dessous n�cessitent une intervention chirurgicale, \\

\textbf{\sua (SUA)} et \textbf{\mua (MUA)} : ces micro-�lectrodes d�pos�es directement au contact de neurones, disposent de la plus haute r�solution et enregistrent des d�charges neuronales. SUA et MUA se distinguent par la taille des populations enregistr�es.\\

\textbf{\seeg (SEEG)} : macro-�lectrodes enregistrant des populations plus larges que les micro-�lectrodes. contrairement � la SUA ou MUA o� l'on peut compter le nombre de fois qu'un ou plusieurs neurones d�chargent, la SEEG enregistre des potentiels �lectriques $(\mu V)$. \\

\textbf{\ecog (ECoG)} : grille flexible compos�e d'une matrice d'�lectrodes. Cette grille est ensuite d�pos�e � la surface corticale. Parce que cette m�thode n'est pas intracortical elle est dite \textit{semi-invasive}.

\subsubsection{Enregistrements non-invasifs}
Les m�thodes \textit{non-invasives} repr�sentent le but ultime de l'impl�mentation concr�te d'une \icm. En effet, ne n�cessitant aucune intervention chirurgicale, leur utilisation est tr�s r�pandue. Actuellement, bien qu'ayant une bonne r�solution temporelle permettant ainsi de capturer les ph�nom�nes courts, elles souffrent encore d'un manque de r�solution spatiale et d'un rapport signal sur bruit plus faible que les techniques invasives. \\

\textbf{\eeg (EEG)} : c'est la technique la plus utilis�e dans le domaine des ICM pour son aspect pratique et portatif. On dispose � la surface de la bo�te cr�nienne un ensemble d'�lectrodes qui enregistrent l'activit� �lectrique �manant de populations larges de neurones. On trouve m�me denouveaux syst�mes sans fils, prometteurs.\\

\textbf{\meg (MEG)} : autre m�thode \textit{non-invasive} mais nettement moins portable que l'EEG, la MEG enregistre les champs magn�tiques r�sultant de l'activit� �lectrique du cerveau. Ces champs magn�tiques peuvent �tre jusqu'� 10 milliards de fois plus faibles que le champs magn�tique terrestre. L'aspect inamovible de la MEG limite son utilisation pour les ICM mais on trouve quand m�me quelque �tudes ayant test� son utilisation \citep{mellinger_meg-based_2007, waldert_hand_2008}.

\figScaleDesciption{1}{Waldert_2009_recording}{M�thodes d'acquisition de l'activit� c�r�brale \citep{waldert_review_2009}}{M�thodes d'acquisition de l'activit� c�r�brale \citep{waldert_review_2009} class�es par invasivit�. La figure indique �galement la taille des populations de neurones enregistr�s ainsi que la r�solution spatiale intimement li�e � l'invasivit�.}

\subsection{Neuro-r�habilitation et handicap moteur}
Pr�lever directement l'activit� neuronale et donc, \textit{bypasser} les voies naturelles, permet de s'affranchir d'�ventuelles limitations physiologiques. C'est pourquoi les ICM repr�sentent un enjeu majeur pour la r�habilitation motrice ou handicap moteur ou encore pour la communication palliative.
Les applications concr�tes des \icm visent donc les personnes disposant de leurs capacit�s cognitives mais qui sont priv�es de facult�s motrices. \\
Par exemple, le \textit{syndrome d'enfermement} ou \textit{locked-in syndrome} qui surviennent majoritairement � la suite d'un accident vasculaire c�r�brale. Les personnes touch�es par ce syndrome sont pleinement consciente de leur corps, de l'environnement, ils peuvent ressentir les sensations de toucher et de douleur mais n'ont plus de facult�s motrices, hormis peut-�tre, les mouvements de paupi�res ou des yeux. Un autre exemple est celui de la la scl�rose lat�rale amyotrophique (ou SLA) qui est une d�g�n�rescence des neurones moteur. Progressivement, les personnes atteintes de SLA perdent l'usage des bras, des jambes, de la parole des muscles faciaux et enfin de la d�glutition. Toutefois, a l'instar du \textit{locked-in syndrome}, la conscience et les facult�s cognitives demeurent intactes. \\

SLA et \textit{locked-in syndrome} ne sont que deux exemples expliquant l'int�r�t social du d�veloppement des ICM. Redonner un peu de contr�le ou �tablir un canal de communication avec les familles des patients expliquent l'engouement qui existe depuis maintenant plus de 40 ans pour les ICM.

% Une description d�taill�e des ICM existantes est propos�e dans la section \ref{etat_de_lart_ICM}.


% -----------------------------------------------------------------------------
% Etat de l'art
\section{�tat de l'art des interfaces cerveau-machine}
\label{etat_de_lart_ICM}

Mettre des ICM plus r�centes voir faire un sch�ma r�capitulatif � mettre en annexe. Good idea

\citep{bekaert_les_2009}
\figScale{bekaert_2009_ICM}{Pipeline g�n�ral d'un \icm}

\subsection{Les diff�rents types d'\icm}
Okokok, ref vers ces diff�rents types
\subsubsection{ICM \textit{synchrones} et \textit{asynchrones}}
Intro blabla \\
\textbf{ICM \textit{synchrones}: } description\\
\textbf{ICM \textit{asynchrones}: } description

\subsubsection{ICM \textit{invasives} et \textit{non-invasives}}
\label{invasif_non-invasif}
Blabla intro \\
\textbf{ICM \textit{invasives} :} description \\
\textbf{ICM \textit{non-invasives} :} description \\
La partie qui va suivre introduit les diff�rents types d'enregistrement c�r�braux.

\subsection{ICM non-invasives}
P300 speller

\subsection{ICM invasives}
\citep{hochberg_reach_2012}
\figScale{hochberg_2012}{Contr�le d'un bras robotis�}

% -----------------------------------------------------------------------------
\section{Apprentissage machine: applications aux neurosciences}
okok

% -----------------------------------------------------------------------------
\section{Encodage et d�codage moteur: bases physiologiques}
Pomper les r�sultas de rodrigo + besserve + inclure sch�ma global du cerveau

% -----------------------------------------------------------------------------
\section{Intention, ex�cution et imagerie motrice}
La figure \ref{fig:hanakawa_2008}   repr�sente  une   trajectoire  d'un
processus markovien de saut, avec les notations associ�es.
\citep{hanakawa_motor_2008}
\figScale{hanakawa_2008}{Comparatif}

% -----------------------------------------------------------------------------
\section{Delayed task: protocole exp�rimental}
okok     % Pr�sentation de la th�matique
% #############################################################################
%                           OBJECTIFS DE THESE
% #############################################################################
\chapter{Objectifs de la th�se}

% -----------------------------------------------------------------------------
\section{D�codage c�r�brale � partir d'activit� intracr�nienne}
\begin{itemize}
\item Exemple d'un sch�ma d'implantation + IRM 
\item Bipolarisation: d�bruitage et augmentation de la sp�cificit� (article Karim)
\item Extraction de features (ici on pourrait mentionner que le deep learning pourrait marcher sur les donn�es brutes)
\end{itemize}

% -----------------------------------------------------------------------------
% Features
\section{Exploration et am�lioration des features}
\subsubsection{R�le physiologique du \pacEN}
\citep{hyafil_neural_2015}
\figScale{hyafil_2015}{M�canismes du couplage phase-amplitude}


% -----------------------------------------------------------------------------
\section{Comparatif des classifieurs}
Expliquer que, chaque classifier poss�de une m�thodologie propre permettant de r�pondre � des types de donn�es diff�rentes (en fonction des hypoth�ses de fonctionnement de chacun des classifiers)

% -----------------------------------------------------------------------------
\section{Exploration des r�gions non-motrices}
\citep{van_langhenhove_interfaces_2008}
\figScale{langhenhove_2008}{Localisation des aires sensorimotrices} % Objectifs de th�se
% #############################################################################
%                             METHODOLOGIE
% #############################################################################
\chapter{m�thodologie}

Cette partie m�thodologique sera divis�e en deux grandes sous parties visant � pr�senter :
\begin{enumerate}
	\item L'extraction des features: pr�sentation des m�thodes utilis�es dans le cadre de l'extraction d'attributs issus de l'activit� neuronale. De mani�re g�n�rale, nous avons �tudi�s des attributs spectraux comprenant:
	\begin{itemize}
		\item Phase et puissance spectrale
		\item Attributs de couplage
	\end{itemize}
	\item Le machine learning: pr�sentation des principaux algorithmes test�es dans le cadre du d�codage de l'activit� neuronale 
\end{enumerate}


\section{Extraction des features}
% -----------------------------------------------------------------------------
% -----------------------------------------------------------------------------
%                                   FEATURES
% -----------------------------------------------------------------------------
% -----------------------------------------------------------------------------
Comme nous l'avons d�crit pr�c�demment, l'objectif du d�codage de l'activit� neuronale est d'arriver � extraire des signaux c�r�braux une information suffisamment pertinente pour pouvoir discriminer diff�rents types de classes (exemple: mouvement vers la gauche Vs droite). \\
Tout les attributs test�s dans le cadre de cette th�se sont des attributs spectraux, donc issus de bandes de fr�quences. La plupart de ces outils partagent donc une partie m�thodologique commune � savoir, le filtrage. De plus, la plupart sont extraits en utilisant la transform�e d'Hilbert. Pour �viter une redondance � travers les attributs, nous allons tout d'abord introduire quelques pr�-requis.

\subsection{Pr�-requis}
% ********************************************
%               PRE-REQUIS
% ********************************************
\subsubsection{Filtrage}
L'int�gralit� des filtrages dans cette th�se ont �t� effectu�s avec la fonction \textit{eegfilt} (qui a ensuite �t� reproduite pour le passage � python). De plus, afin d'�viter tout ph�nom�ne de d�phasage, le fonction \textit{filtfilt} a �t� syst�matiquement utilis�e afin que le filtre soit appliqu� dans les deux sens. Si cette derni�re fonctionnalit� n'est pas forc�ment indispensable dans le cadre d'un calcul de puissance, elle est absolument n�cessaire pour un calcul de \pacFR.  \\
L'ordre du filtre pr�sent� au dessus d�pend de la fr�quence de filtrage. Il a syst�matiquement �t� calcul� en utilisant la m�thode d�crite par \cite{bahramisharif_propagating_2013}:
\begin{equation}
FiltOrder = N_{cycle} \times f_{s}/f_{oi}
\end{equation} \\
o� $f_{s}$ est la fr�quence d'�chantillonnage, $f_{oi}$ est la fr�quence d'int�r�t et $N_{cycle}$ est un nombre de cycles d�finit par $N_{cycle}=3$ pour les oscillations lentes et $N_{cycle}=6$ pour les oscillations rapides. 

\subsubsection{Transform�e d'Hilbert}
Transform�e permettant de passer un signal temporel $x(t)$ du domaine r�el au domaine complexe. Le signal peut ensuite s'�crire $x_{H}(t)=a(t)e^{j\phi(t)}$ o� $a(t)$ est l'amplitude et $\phi(t)$, la phase. Cette transformation est particuli�rement exploit�e car le module de $x_{H}(t)$ permet de r�cup�rer l'amplitude et la phase est obtenue en prenant l'angle de $x_{H}(t)$.

\subsubsection{Transform�e en ondelettes}
La transform�e en ondelettes \citep{tallon-baudry_oscillatory_1997, worrell_recording_2012} permet de d�composer un signal dans le domaine temps-fr�quence. La d�composition en ondelettes d'une fonction $f$ est d�finie par:
\begin{equation}
f(a, b)=\int_{-\infty}^{\infty} \mathrm{f}(x)\overline{\psi}_{a, b}\mathrm{d}x 
\end{equation} \\
O� $\psi$ est appel� ondelette m�re dont la d�finition g�n�rale est donn�e par $\psi_{a,b}=\frac{1}{\sqrt{a}}\Psi(\frac{x-b}{a})$ o� $a$ est le facteur de dilatation et $b$ le facteur de translation. Le choix de l'ondelette m�re s'est port� sur l'ondelette de Morlet qui est tr�s largement utilis�e � travers la litt�rature et d�finie par:
\begin{equation}
w(t,f_{0})=A \mathrm{e}^{-t^{2}/2\sigma_{t}^{2}} \mathrm{e}^{2i\pi f_{0}t}
\end{equation}\\
O� $\sigma_{f}=1/2\pi \sigma_{t}$ et $A=(\sigma_{t}\sqrt{\pi})^{-1/2}$. L'ondelette de Morlet est caract�ris�e par le ratio constant $r=f_{0}/\sigma_{f}$ que nous avons fix� �gale � $7$ comme sugg�r� par \cite{tallon-baudry_oscillatory_1997}.\\ 
Cette d�composition peut �tre compar�e � la transform�e courte de Fourier qui d�compose le signal en une somme de combinaisons lin�aire de sinus et de cosinus mais part du principe qu'il existe une r�gularit� dans le signal permettant une telle d�composition. La transform�e en ondelettes r�sout plusieurs limitations:
\begin{itemize}
\item Elle permet d'obtenir l'�nergie d'un signal dans le temps, ce qui permet une bien meilleure exploration des ph�nom�nes.
\item Le rapport constant $r$ permet d'obtenir des ondelettes dont la r�solution fr�quentielle varie    en fonction des fr�quences et permet une meilleure co�ncidence avec la d�finition des bandes physiologiques \citep{bertrand_time-frequency_1994}
\end{itemize}
\vspace{1\baselineskip}
Tout les attributs qui vont �tre maintenant pr�sent�s, utilisent les m�thodes d�crites ci-dessus.

\subsubsection{�valuation statistique � base de permutations}
Pour une distribution de permutations construite � partir de deux sous-ensembles $A$ et $B$ et comportant $N$ observations et pour une valeur $p$ pr�d�finie, on pourra conlure que:
\begin{itemize}
	\item $A>B$ si $A$ est parmi les $N-N\times p$ derniers �chantillons (\textit{"One-tailed test upper tail"})
	\item $A<B$ si $A$ est parmi les $N\times p$ premiers �chantillons (\textit{"One-tailed test lower tail"})
	\item $A\neg B$ si $A$ est soit inf�rieur aux $(N\times p)/2$ premiers �chantillons soit sup�rieur aux $(N-N\times p)/2$
\end{itemize}  
\figScaleX{0.6}{stat_permutations}{\textit{"One-tailed"} et \textit{"two-tailed"} test}
Gr�ce � cette m�thode d'�valuation statistique, nous pourrons par exemple conclure si l'on a une augmentation, une diminution ou une diff�rence statistique entre une valeur de puissance et la puissance contenue dans une p�riode de baseline. Derni�re pr�cision, on comprend ainsi que pour obtenir une valeur $p$ il faut que la taille de la distribution $N$ soit au moins de $1/p$.



% ********************************************
%                 PUISSANCE
% ********************************************
\subsection{Puissance spectrale}

\subsubsection{M�thodes explor�es}
Le calcul de la puissance spectrale a �t� approch� par deux m�thodologies et qui ont �t� utilis�s � des fins diff�rentes :
\begin{itemize}
	\item La transform�e d'Hilbert: souvent exploit� dans le cadre du d�codage ainsi que pour garder une uniformit� entre les attributs de phase et \pacFR bas�s eux aussi sur cette transform�e.
	\item La transform�e en ondelettes: principalement utilis�e pour la visualisation des cartes \tf � cause de l'adaptation des ondelettes aux bandes physiologiques.
\end{itemize}

\subsubsection{Normalisation}
On utilise la normalisation pour observer l'�mergence d'un ph�nom�ne par rapport � une p�riode d�finie comme baseline. A travers la litt�rature, quatre grands types de normalisation sont rencontr�s:
\begin{enumerate}
	\item Soustraction par la moyenne de la baseline
	\item Division par la moyenne de la baseline
	\item Soustraction puis division par la moyenne de la baseline
	\item Z-score: soustraction de la moyenne puis division par la d�viation de la baseline
\end{enumerate}
La normalisation z-score est certainement la plus fr�quemment rencontr�e � travers la litt�rature. Le choix du type de normalisation d�pend du type de donn�es utilis�es. Dans le cadre de nos donn�es, \textit{3.} �tait clairement la plus adapt�e pour la visualisation. En revanche, dans le cadre de la classification, nous obtenions syst�matiquement de meilleurs r�sultats sans normalisation.

\subsubsection{�valuation statistique}
La fiabilit� statistique de la puissance a �t� �valu�e en comparant chaque valeur de puissance � la puissance contenue dans une p�riode d�finie comme baseline. Pour ce faire, nous avons test� deux approches:
\begin{enumerate}
	\item Permutations : les valeurs de puissance et de baseline sont al�atoirement m�lang�es � travers les essais. Puis, on normalise cette puissance. En r�p�tant cet proc�dure $N$ fois, on obtient une distribution qui peut ensuite �tre utilis�e pour en d�duire la valeur $p$ de la v�ritable puissance (cf: \textit{pr�-requis})
	\item "Wilcoxon signed-rank test": ordonne les distances entre les paires de puissances (vraie valeur, baseline) \citep{demandt_reaching_2012, rickert_encoding_2005, waldert_hand_2008}
\end{enumerate}
\figScaleX{0.6}{ossandon_tf}{Exemple de repr�sentation temps-fr�quence de puissance normalis�es z-score \citep{ossandon_transient_2011}}



% ********************************************
%                     PHASE
% ********************************************
\subsection{Phase}
L'extraction de la phase se fait de la m\^{e}me mani�re que pour le \PACFR, en prenant l'angle de la transform�e d'Hilbert d'un signal filtr�. La significativit� peut �tre �valu�e en utilisant le test de Rayleigh \citep{jervis_fundamental_1983, tallon-baudry_oscillatory_1997}. Point de vue pratique, cela correspond � la fonction \textit{circ\_rtest} de la toolbox Matlab \textit{CircStat} \citep{berens_circstat_2009}




% ********************************************
%                   PAC
% ********************************************
\subsection{\PACEN}
Le calcul du \PACEN ne se limite pas uniquement � la m�thode. En r�alit�, pour obtenir une estimation fiable sur des donn�es r�elles, il est indispensable de suivre les trois �tapes suivantes:
\begin{enumerate}
	\item Estimation de la v�ritable valeur de PAC. Il existe plusieurs m�thodes.
	\item Calcul de "\textit{surrogates}": on va calculer des PAC d�structur�s. Idem, il existe de nombreuses m�thodes
	\item Correction du v�ritable PAC par les "\textit{surrogates}". Cette correction, qui est en faite une normalisation, aura pour but de soustraire � l'estimation du PAC de l'information consid�r�e comme bruit�e.
\end{enumerate}
Les sous-parties suivantes pr�senteront de mani�res succinctes les principales m�thodes rencontr�es dans la litt�rature, ainsi que diff�rents types de corrections applicables.

\subsubsection{M�thodologie du \pacEN}
Il existe une large vari�t� de m�thodes pour calculer le PAC, ce qui complique son exploration. Toutefois, il n'existe pas de consensus sur une m�thode plus polyvalente qu'une autre, chacune poss�dant ses points forts et limitations.
Pour aller un peu plus loin, et pr�senter quelques m�thodes, il est n�cessaire d'introduire quelques variable. Soit $x(t)$, une s�rie temporelle de donn�es de taille N. Pour cette s�rie temporelle, on souhaite savoir si la phase extraite dans une bande de fr�quence $f_{\phi}=[f_{\phi_{1}},f_{\phi_{2}}]$ est coupl�e avec l'amplitude contenue dans $f_{A}=[f_{A_{1}},f_{A_{2}}]$. Pour cela, on va tout d'abord extraire $x_{\phi}(t)$ et $x_{A}(t)$ les signaux filtr�s dans ces deux bandes. Enfin, la phase $\phi(t)$ est obtenue en prenant l'angle de la transform�e d'Hilbert de $x_{\phi}(t)$ tandis que l'amplitude $a(t)$ est obtenue en prenant le module de la transform�e d'Hilbert de $x_{A}(t)$. 

\begin{enumerate}

	% MEAN VECTOR LENGTH
	\item \mvl: \\
	Cette m�thode � �t� introduite par \cite{canolty_high_2006} et consiste � sommer, � travers le temps, le complexe form� de l'amplitude des hautes fr�quences avec la phase des basses fr�quences. L'�quation est donn�e par:
	\begin{equation}
	MVL = |\sum_{j=1}^{N} a_(j) \times e^{j\phi(j)}|
	\end{equation} \\
	
	% KULLBACK-LEIBLER DIVERGENCE
	\item \kld:\\
	A l'origine, la divergence de Kullback-Leibler (KLD), qui est issue de la th�orie de l'information, permet de mesurer les dissimilarit�s entre deux distributions de probabilit�s. Ainsi, pour pouvoir utiliser cette mesure dans le cadre du PAC, \cite{tort_measuring_2010} propose une solution �l�gante qui consiste � g�n�rer une distribution de densit� probabilit�s de l'amplitude (DPA) en fonction des valeurs de phase et d'ensuite utiliser le KLD pour comparer cette distribution � la densit� de probabilit� d'une distribution uniforme (DPU). Plus la DPA s'�loigne de la DPU, plus le couplage entre l'amplitude et la phase est consistant. \\
	Pour construire la DPA, l'astuce consiste � couper le cercle trigonom�trique en N tranches (dans l'article il est propos� de couper en 18 tranches de 20�). Puis, si on prend l'exemple de la tranche $[0,20�]$, on va chercher tout les instants temporels o� la phase prend des valeurs comprises entre $[0,20�]$ ($t, \phi(t) \subset [0,20�]$). On prend ensuite la moyenne de l'amplitude pour ces valeurs de $t$ et on r�p�te cette proc�dure pour chacune des tranches de phase. On obtient ainsi la densit� d'amplitudes en fonction des valeurs de phase. Il ne reste plus qu'� normaliser cette distribution par la somme des amplitudes � travers les tranches et on r�cup�re une distribution de densit� de probabilit�s. La figure \ref{fig:PAC_plot_Tort_2010} \citep{tort_measuring_2010} pr�sente un exemple de DPA en fonction de tranches de phase. \\

	\figScaleX{0.6}{PAC_plot_Tort_2010}{Densit� de probabilit� d'une distribution d'amplitudes en fonction de tranches de phases}
	
	Le calcul de la divergence de Kullback-Leibler est ensuite appliqu� pour mesurer les dissimilarit�s entre la DPA et la DPU et c'est cette mesure qui servira d'estimation du \pacFR:
	\begin{equation}
	D_{KL}(P, Q) = \displaystyle\sum_{j=1}^{N} P(j) \times \log{\frac{P(j)}{Q(j)}}
	\end{equation} \\
	o� $P(j)$ est la densit� de probabilit� de $a(t)$ en fonction de $\phi(t)$ et $Q(j)$ est la densit� de probabilit� d'une distribution uniforme. \\
	
	% HEIGHT-RATIO
	\item \hr \\
	La m�thode du \hr \citep{lakatos_oscillatory_2005} est extr\^{e}mement proche du \kld. En effet, l'amplitude sera bin�e de la m�me fa�on en fonction des tranches de phase. La mesure du PAC est ensuite donn�e par:
	\begin{equation}
	hr=(f_{max}-f_{min})/f_{max}
	\end{equation} \\
	o� $f_{max}$ et $f_{min}$ sont respectivement le maximum et le minimum de la de la densit� de probabilit� de l'amplitude en fonction des valeurs de phase. \\
	
	% NDPAC
	\item \ndpac \\
	Le \ndpac, qui n'est pas une des m�thodes les plus fr�quemment rencontr�es, pr�sente toutefois une avantage certain. En plus de fournir une estimation fiable du \pacFR, \cite{ozkurt_statistically_2012} d�montre l'existence d'un seuil � partir duquel on peut consid�rer l'estimation du PAC comme �tant statistiquement fiable. La beaut� de cette m�thode, c'est que ce seuil statistique, qui est une fonction de la valeur p d�sir�e, ne d�pend que de la taille de la s�rie temporelle. Ce qui rend son utilisation particuli�rement simple. \\
	Pour estimer le PAC, une des hypoth�ses ayant permis d'aboutir � ce seuil statistique est de devoir normaliser l'amplitude par un z-score d�not�e $\tilde{a}(t)$. L'estimation du PAC est quasiment identique au MVL puisque c'est en r�alit� le carr� de celle-ci. Enfin, pour une valeur p d�sir�e, l'article introduit le seuil statistique: \\
	\begin{equation}
	x_{lim}=N \times [erf^{-1}(1-p)]^{2}
	\end{equation} \\
	
	o� $erf^{-1}$ est la fonction d'erreur inverse. On d�duira que l'estimation PAC est significative si et seulement si cette valeur est deux fois sup�rieur � ce seuil. \\
	
	% AUTRES
	\item Autres m�thodes:
	Tout les algorithmes pr�sent�s ci-dessus ont �t� test�s, impl�ment�s et compar�s. En compl�ment, voici une liste non exhaustive d'autres m�thodes existantes:
	\begin{itemize}
		\item \textit{Phase Locking Value (PLV)} \citep{cohen_assessing_2008, penny_testing_2008}: d�tournement du PLV propos� par \cite{lachaux_measuring_1999} qui mesure la synchronie de phase entre deux �lectrodes. Cette m�thode va comparer la phase des basses fr�quences avec la phase de l'amplitude des hautes-fr�quences.
		\item \textit{Generalized Linear Model (GLM)} \citep{penny_testing_2008}: outil d�crit comme adapt� aux donn�es courtes et bruit�es.
		\item \textit{Generalized Morse Wavelets (GMW)} \citep{nakhnikian_novel_2016}: bas�e sur des ondelettes, semble particuli�rement utile dans le cadre de l'exploration des donn�es.
		\item \textit{Oscillatory Triggered Coupling (OTC)} \citep{dvorak_toward_2014, watrous_phase-amplitude_2015}: issue d'une d�tection de maximums des hautes fr�quences.
	\end{itemize}

\end{enumerate}

% PERMUTATIONS ET NORMALISATION
\subsubsection{Correction du \pacEN et �valuation statistique}
Nous avons vu dans la section pr�c�dente diff�rentes m�thodes permettant de calculer un \PACFR. Toutefois, celui-ci peut �tre largement am�lior�e en faisant une estimation du PAC contenu dans le bruit des donn�es. Une fois que cette estimation sera faite, on pourra retrancher ce PAC bruit� � la valeur initiale. Tout comme il existe plusieurs m�thodes de PAC, les �quipes de recherche proposent � tour de r�le de nouvelles m�thodes. Parmi elles, on peut citer:
\begin{itemize}
	\item \textit{Time-lag}: propos�e par \cite{canolty_high_2006}, on introduit un d�lai sur l'amplitude compris entre $[f_{s},N-f_{s}]$ o� $f_{s}$ est la fr�quence d'�chantillonnage et $N$ est le nombre de points de la s�rie temporelle
	\item \textit{Shuffling des couples [phase,amplitude]}: ici, on m�lange al�atoirement les essais de phase et d'amplitude \citep{tort_measuring_2010}
	\item \textit{Swapping temporel d'amplitudes (ou de phase)}: on m�lange al�atoirement les essais d'amplitude puis on recalcule le PAC avec la phase originale \citep{bahramisharif_propagating_2013, lachaux_measuring_1999, penny_testing_2008, yanagisawa_regulation_2012} 
\end{itemize}
Ces trois m�thodes produisent une distribution de \textit{surrogates}. On pourra ensuite appliquer un z-score � la v�ritable estimation en utilisant la moyenne et la d�viation de cette distribution. Enfin, l'�valuation statistique se fait �galement � partir de cette distribution  (cf: \textit{pr�-requis}) \\
A ma connaissance, il n'existe pas de comparatif entre ces corrections et je n'ai jamais rencontr� d'articles mentionnant que l'on ne puisse pas combiner les m�thodes de PAC avec les diff�rentes corrections. En revanche, ce qui est relat� c'est que le \textit{time-lag} n�cessite des donn�es longues d\^{u} � l'introduction de ce d�lai temporel. 

% COMPARATIF
\subsubsection{Comparatif des m�thodes}
\cite{penny_testing_2008} ont compar� plusieurs m�thodes dont le \textit{MVL}, \textit{PLV} et le \textit{GLM} et \cite{tort_measuring_2010} ont compl�t� cette �tude avec d'autres m�thodes comme le montre le tableau ci-dessous. Enfin, \cite{canolty_functional_2010} a fait une review qui comprend un descriptif tr�s instructif.
\begin{figure}[H]
	\hspace*{-1in}
	\includegraphics[scale=0.35]{./figures/PAC_methods_comparison}
	\caption{Comparatif de m�thodes d'�valuation de \pacFR}
\end{figure}

% VISUALISATION
\subsubsection{Repr�sentation du \pacEN}
Compar�e � la puissance, l'exploration du PAC peut s'av�rer plus complexe d\^{u} � sa dimensionnalit� plus grande. Il existe donc des outils et des m�thodes destin�es � simplifier cette exploration et � visualiser ces r�sultats. \\
Exemple concret, si on cherche � conna\^{i}tre les modulations de puissance contenue dans un signal, on peut repr�senter une carte \tf. Pour le PAC, id�alement on voudrait visualiser les phases, les amplitudes et le temps mais ces trois dimensions emp�che une repr�sentation simple. On peut donc avoir recours � diff�rents types de repr�sentations compl�mentaires:
\begin{itemize}
	\item Puissance phase-locked : cette repr�sentation permet de faire �merger l'existence d'un couplage, pour une phase donn�e, et d'observer sa dur�e. Pour cela, on aligne les phase en d�tectant le pic le plus proche de l'instant temporel �tudi�. On calcul les cartes \tf que l'on va ensuite moyenner apr�s les avoir recal�es de la m�me fa�on que les phases (\cad avec la m�me latence).
	\item Comodulogramme : pour une tranche temporelle d�finie, on repr�sente les valeur de PAC pour diff�rentes valeurs de phase et d'amplitude
\end{itemize}

\figScaleX{1}{hemptinne_2013}{\textbf{(A)} Exemple de cartes temps-fr�quence phase locked sur le $\beta$, \textbf{(B)} Exemple de comodulogramme}

La figure \ref{fig:hemptinne_2013} \citep{hemptinne_exaggerated_2013} met en �vidence que la repr�sentation des cartes temps-fr�quence phase-locked \textbf{(A)} est limit�e d'une part, par la phase sur laquelle on choisit de recaler et d'autre part cette m�thode est �galement limit� par l'instant o� l'on choisit de recaler. Pour la figure \textbf{(B)}, le calcul du PAC se faisant � travers la dimension temporelle, on a aucune id�e de l'�volution du couplage dans le temps. \\

% ERPAC
\subsubsection{\PACEN: r�solution temporel?}
Comment peut-on savoir si un ensemble de musiciens jouent ensemble, en rythme? L'approche traditionnelle consiste � dire que, en fonction de la prestation du groupe, on sera en mesure de dire si ils �taient en rythme ou non. Donc on focalise notre attention sur chaque instant du morceau et on analyse chaque note, chaque d�calage. Cela signifie aussi que toute notre attention a �t� mobilis�e par l'analyse du rythme et finalement, on passe � c�t� de la musique. Notre attention au d�tail nous a �cart� du morceau global. On pourrait dire que l'on a �cras� la dimension temporelle du morceau. Une autre approche consiste � assister � toute les r�p�titions du fameux groupe. Ce faisant, on est capable de dire si d'une mani�re g�n�rale les musiciens ont tendance � jouer ensemble. Ainsi, le jour d'une repr�sentation, toute notre attention peut rester uniquement sur le concert. On garde donc la dimension temporelle.\\ 
C'est par ce changement de positionnement face au probl�me de r�solution temporelle que \cite{voytek_method_2013} introduit le \erpac . L'approche traditionnelle du PAC n�cessitant de conna�tre un nombre de cycles afin d'en d�duire l'existence ou non du couplage, et donc perdre la dimension temps, l'article propose de calculer le PAC � travers les essais (ou r�p�titions). Pour un jeu de donn�es de M essais de longueur N, on extrait respectivement les phases et les amplitudes $\phi_{M}(t)$ et $a_{M}(t)$ puis, pour chaque point temporel, on calcul la corr�lation � travers les essais (corr�lation lin�aire-circulaire \citep{berens_circstat_2009} qui se fait entre l'amplitude et des sinus/cosinus de la phase). Il en r�sulte une valeur de corr�lation pour chaque instant et donc, de couplage. \\



\section{Apprentissage supervis�}
% -----------------------------------------------------------------------------
% -----------------------------------------------------------------------------
%                           APPRENTISSAGE SUPERVISE
% -----------------------------------------------------------------------------
% -----------------------------------------------------------------------------
Pr�sentation du concept
Training set et Testing set

\subsection{Labellisation et apprentissage}

\subsection{Classifieurs}
\begin{enumerate}

	% LINEAR DISCRIMINANT ANALYSIS
	\item \lda \\
	
	\citep{fisher_use_1936}, \citep{lotte_review_2007} \\
	
	
	\figScaleX{0.4}{LDA_Lotte_2007}{Principe du \lda}
	
	% SUPPORT VECTOR MACHINE
	\item \svm \\
	\citep{cortes_support-vector_1995, vapnik_nature_2000}
	\citep{lotte_review_2007}
	\figScaleX{0.4}{SVM_Lotte_2007}{Principe du \svm}
	
	% K-NEAREST NEIGHBOR
	\item \knn
	
	% NAIVE BAYES
	\item \nb
	
	% RANDOM FOREST
	\item \rf
\end{enumerate}

\begin{figure}[H]
	\hspace*{-0.7in}
	\includegraphics[scale=0.35]{./figures/scikit_classifier}
	\caption{Comparatif de classifieurs (scikit-learn)}
\end{figure}

\subsection{Cross-validation}
Pr�sentation et utilisation (contexte: s�paration Training et Testing set // optimisation des param�tres (classifieurs et \mf))

\begin{enumerate}
	\item \kfold
	
	\kfold , \skfold , \shsp, et \sshsp
	\item \loo

\end{enumerate}
%\verb+\+figScaleX\{0.6\}\{LDA\_Lotte\_2007\}\{Principe du \verb+\+lda\}



\subsubsection{�valuation de la performance de d�codage}
\begin{enumerate}
	\item \da
	\item \roc
\end{enumerate}

\subsubsection{�valuation statistique de la performance de d�codage}
\begin{enumerate}
\item Loie binomiale
\item Permutation: data driven + diff�rents types de permutations \citep{ojala_permutation_2010}
\begin{itemize}
\item Shuffle y
\item Full-shuffle
\item Intra-class shuffle y
\end{itemize}
\end{enumerate}

\subsection{Sur-apprentissage}

\subsubsection{Optimisation des param�tres de classification}
% -----------------------------------------------------------------------------
% -> Single et Multi-features
\section{Du single au multi-features}
Pr�sentation du concept
\subsection{Single-feature}

\subsection{Multi-features}

\begin{enumerate}

\item S�lection statistique
\begin{enumerate}
\item S�lection binomiale
\item s�lection permutations
\end{enumerate}
\item S�lection s�quentielle
\begin{enumerate}
\item Forward selection
\item Backward selection
\item exhaustive selection
\end{enumerate}

\end{enumerate}   % M�thodologie
% #############################################################################
%                             DONNEES EXPERIMENTALES
% #############################################################################
\chapter{donn�es exp�rimentales}

% -----------------------------------------------------------------------------
\section{Donn�es "Center-out"}

% -----------------------------------------------------------------------------
\section{Autres donn�es}      % Donn�es exp�rimentales


% #############################################################################
%                                   OUVERTURE
% #############################################################################
\chapter{ouverture}
Nos contributions portent sur : \dots \\*

Le \emph{premier chapitre} expose  la probl�matique de la th�se.

Le \emph{deuxi�me chapitre} pr�sente  en d�tail \dots \\*

etc.

Cette th�se  a fait l'objet de  divers travaux �crits : \dots


\adjustmtc
 
% ==================================================================
% CONTENU G�N�RAL
%\pagestyle{headings}
\part{�tude 1: niveau de chance et �valuation statistique des r�sultats de classification par apprentissage supervis�}
\label{seuil_chance}

\begin{chapintro}
%  \malettrine{P}{ourquoi}  cette �tude? Quelles questions?
%- Seuil de chance th�orique vs pratique?
%- Impact sur des m�thodes (\cv, \clf)
%- Validation sur des donn�es r�elles (Intra MEG)
Sensibilisation � l'importance du nombre d'essais par exemple
\end{chapintro}

%%% --------------------------
%%% R�sum� de l'article
%%% --------------------------
\section{pr�sentation de l'�tude}
\subsection{Contexte}
\subsection{Probl�matique}
\subsection{R�sultats majeurs}
pourquoi cette �tude? Quelles questions?
- Seuil de chance th�orique vs pratique?
- Impact sur des m�thodes (\cv, \clf)
- Validation sur des donn�es r�elles (Intra MEG)
-  d�di� aux �tudiants
- Fournit une toolbox pour reproduire les r�sultats

%%% --------------------------
%%% Article
%%% --------------------------
\section{article}
\includepdf[pages={2-12}]{Chap1/Combrisson-Jerbi-JNeuroscienceMethods_2015.pdf}

%%% --------------------------
%%% Compl�ments
%%% --------------------------
\section{compl�ments d'�tude}
compl�ments sur les diff�rents types de permutation
\citep{ojala_permutation_2010}













%%% --------------------------
%%% Conclusion de l'article??
%%% --------------------------
%\section*{conclusion du chapitre}
%\addcontentsline{toc}{section}{Conclusion}
%
%Ceci est la conclusion. Personnellement, je n'aime pas que la conclusion 
%soit num�rot�, mais je veux qu'elle apparaisse dans la table des mati�re, d'o� 
%la commande addcontentsline.

\part{�tude 2: encodage de l'intention et de l'ex�cution motrice}
\label{Etude2_encodage}


\chapter*{Introduction}
blabla Intro

\vspace{1\baselineskip}
Blabla �tude


%%% --------------------------
%%% Article
%%% --------------------------
\includepdf[pages={1-29}]{Chap2/Etude2_Combrisson-etal-Encoding-Intention-Execution.pdf}


%!TEX root = ../main.tex
\part{�tude 3 : D�codage des directions de mouvement pendant et avant l'ex�cution de mouvement de membres sup�rieurs}
\label{Etude3_decodage}


\chapter*{Introduction}
Mon introduction


%%% --------------------------
%%% Article
%%% --------------------------
\includepdf[pages={1-28}]{Chap3/Combrisson-etal_Decoding-Intention-Execution.pdf}


%!TEX root = ../main.tex
\part{�tude 4 : Tensorpac, logiciel Python de calcul de Phase-Amplitude Coupling}
\label{Etude4_tensorpac}
\pagestyle{headings}


\chapter*{Introduction}
Le \pacFR (PAC) est un marqueur qui mesure le degr�s de couplage entre la phase d'ondes lentes et l'amplitude d'ondes rapides. L'�valuation d'un couplage se fait de mani�re suivante :
\begin{itemize}    
    \item Extraction de la phase et de l'amplitude en utilisant soit des outils de filtrage suivi de la transform�e d'Hilbert, soit une transformation continue en ondelettes.
    \item Calcul du couplage entre ces deux signaux en utilisant une des m�thodologies existantes \citep{tort_measuring_2010,ozkurt_statistically_2012,canolty_high_2006}...
    \item Le PAC �tant une mesure sensible aux bruits, on construit une distribution de mesure de PAC pouvant arriver par chance.
    \item La v�ritable mesure de PAC est ensuite normalis�e par cette distribution de chance afin de minimiser le bruit.
\end{itemize}
Un nombre cons�quent de m�thodes ont �t� propos�es pour chacune de ces �tapes ce qui complique la comparaison et la reproductibilit�. De plus, toutes les publications introduisant de nouvelles m�thodes les pr�sentent en utilisant des vecteurs et ne fournissent pas l'adaptation matricielle ce qui ne prend pas en compte le format des donn�es (nombre de sujets, d'�lectrodes, d'essais...) et donc n'est pas du tout optimal d'un point de vue temps de calcul.\\
Dans ce contexte, nous avons mis en place une toolbox Python, \textit{Tensorpac}, d�di�e exclusivement au calcul du \pacFR. Dans cette toolbox les m�thodes sont impl�ment�es de fa�on modulaire ce qui signifie que l'utilisateur peut combiner les m�thodes existantes pour chacune des �tapes du calcul du PAC. D'autre part, \textit{Tensorpac} utilise des tenseurs permettant de g�n�raliser le calcul � partir de s�ries temporelles vers des donn�es multi-dimensionnelles. Cette impl�mentation en tenseurs est combin�e � du calcul en parall�le ce qui diminue encore le temps d'ex�cution et facilite l'envoie sur des serveurs de calcul. Ce paquet inclue �galement le calcul de comodulograme (soit en cherchant les couples (phase, amplitude) soit en fixant l'un des deux et en faisant varier la largeur de bande de l'autre), statistiques et la visualisation. Pour finir, \textit{Tensorpac} est distribu� sous une licence BSD et peut �tre t�l�charger sur Github \footnotemark[1] et nous fournissons �galement une documentation d�taill�e \footnotemark[2].

\footnotetext[1]{\url{https://github.com/EtienneCmb/tensorpac}}
\footnotetext[2]{\url{https://etiennecmb.github.io/tensorpac/}}


%%% --------------------------
%%% Article
%%% --------------------------
\includepdf[pages={1-18}]{Chap4/combrisson_2017_tensorpac.pdf}
\part{�tude 5: d�codage des �motions}
\label{Etude5_emotions}


\chapter*{Introduction}
blabla Intro

\vspace{1\baselineskip}
Blabla �tude


%%% --------------------------
%%% Article
%%% --------------------------
%\includepdf[pages={1-29}]{Chap2/Etude2_Combrisson-etal-Encoding-Intention-Execution.pdf}




% ==================================================================
% CONCLUSION
\part{Discussion, conclusion et perspectives}
% \addcontentsline{toc}{chapter}{Conclusion g�n�rale}
\pagestyle{headings}

\chapter*{Discussion}

% Paragraphe 1-Vue d'ensemble de ce travail de th�se: Il s'est d�clin� en deux composante: Une contribution empirique [articles 2 et 3] et des contributions m�thodologiques (� la fois th�oriques [article 1] mais aussi en temre de d�veloppement logiciel [articles 4-6].

Ce travail de th�se s'articule autour de deux grands axes, (\textit{1}) un apport empirique permettant d'am�liorer notre compr�hension du r�le des composantes d'amplitude, de phase et de couplage phase-amplitude lors d'une t�che motrice dirig�e (articles 2 et 3). Nous avons utilis� des algorithmes de classification pour identifier les marqueurs les plus pertinents, (\textit{2}) un apport m�thodologique que ce soit d'un point de vue th�orique pour mieux comprendre la notion de seuil de chance en \textit{machine-learning} (article 1) et pratique via l'impl�mentation de plusieurs logiciels d'analyse et de visualisation (articles 4-6). \\

% Paragraphe 2-R�sum� des r�sultats les plus saillants obtenus dans les �tudes empiriques (articles 2 et 3)

Les deux articles empiriques partagent une m�me m�thodologie : l'utilisation des algorithmes de \textit{machine-learning} pour identifier les marqueurs de puissance, de phase et de PAC susceptibles de d�coder les �tats moteurs (article 2 cf. \ref{Etude2_encodage}) ou les directions motrices (article 3 cf. \ref{Etude3_decodage}). Cette premi�re �tude a permis de mettre en �vidence une augmentation significative de couplage $\alpha/\gamma$ durant la phase de repos qui s'att�nue ensuite durant l'ex�cution. D'un point de vue d�codage, la composante de phase tr�s basse fr�quence (\textit{VLFC}, $<1.5hz$) est le marqueur ayant montr� les plus forts taux de d�codage � travers toutes les conditions classifi�es (\textit{Repos Vs Pr�p.}, \textit{Repos Vs Exec.}, \textit{Pr�p. Vs Exec.}, \textit{Repos Vs Pr�p. Vs Exec.}). Dans une moindre mesure, le PAC pr�sente lui aussi des d�codages significatifs pour diff�rencier ces �tats moteurs ce qui n'est pas le cas pour diff�rencier les directions (article 3 cf \ref{Etude3_decodage}). En effet, le d�codage \textit{Up Vs Down Vs Left Vs Right} est majoritairement possible via des marqueurs de puissance. Durant l'ex�cution, la puissance des oscillations $\gamma$ permet de clairement diff�rencier les modulations � travers les directions dans les r�gions motrices et pr�-motrices. Plus important encore, le m�me ph�nom�ne a lieu lorsque la puissance dans la bande $\alpha$ dans le cortex pr�-moteur et frontal est utilis�e. Pris ind�pendamment, ces deux marqueurs en utilisation unique permettent d'obtenir des d�codages significatifs ($~70\%$ pour l'ex�cution et $~50\%$ durant la pr�paration). Pour finir, nous avons combin� plusieurs algorithmes de \textit{feature selection} ce qui a permis d'atteindre des d�codages proches de $90\%$ pour diff�rencier les quatre directions � la fois durant l'ex�cution et la pr�paration motrice. \\

% Paragraphe 3-Mettre en valeur la nouveaut�/originalit� de ces r�sultats (� quel point c'est diff�rent de ce qu'avait fait les autres chercheurs jusqu'ici?)

PARAGRAPHE DE CE QUI EST VRAIMENT NOUVEAU. \\

% Paragrpahe 4-Regard critique sur les 2 �tudes en intra: Les limitations de ces donn�es (nombre de patients, le fait qu'il s'agit de patient, le fait que les electrodes se trouvent � des endroits diff�rents � travers les patients, le fait que le d�lain'�tait pas variable, le fait qu'on avait pas acc�s syst�matique � l'onset du mouvement, etc etc).

Ces deux �tudes ont un certain nombre de limitations. Tout d'abord, les donn�es intracr�niennes �tant relativement rares, nous n'avons acc�s qu'aux donn�es de six sujets, chacun ayant approximativement une centaine d'�lectrodes. Au total, ces 6OO points d'enregistrement offrent une couverture partielle des r�gions c�r�brales. De plus, ces sujets souffrent d'une forme d'�pilepsie pharmacor�sistante ce qui limite la possibilit� de g�n�raliser � des sujets sains. L'implantation des sites intracr�niens �tant li�e � la localisation du foyer �pileptog�ne, il en r�sulte une implantation propre � chaque sujet et donc, d'une part il est tr�s difficile de pouvoir g�n�raliser le comportement d'un site en particulier puisqu'il est possible qu'il soit le seul dans cette r�gion et d'autre part, l'implantation n'�tant pas sym�trique elle ne permet pas d'�tudier l'effet controlat�ral et ipsilat�ral. La t�che utilis� pour mener � bien ces deux �tudes a �galement des limitations. Tout d'abord, tout les \textit{timing} sont fixes et donc, en l'absence de d�lais variables, il est tout � fait envisageable que les sujets puissent anticiper le d�but d'une phase au fur et � mesure que la t�che avance. De plus, le design de la t�che ne permet pas de conclure clairement sur le type de processus d�cod�s. En effet, on est en droit de se demander si l'on d�code une v�ritable pr�paration/ex�cution de mouvements ou un processus attentionnel, visuomoteur, s�lection spatiale ou d'imagerie motrice. Pour finir, apr�s l�apparition d'un signal visuel, les donn�es ne permettaient pas d'avoir acc�s au d�but du mouvement. Ainsi, l'activit� neuronale contenue dans les tanches temporelles qui suivent directement l'apparition du signal visuel peut contenir des �v�nements diff�rents d� � la variabilit� du temps de r�action d'un essais � l'autre. Les outils de \textit{machine-learning} �tant justement entra�n�s � travers les essais, l'alignement sur le \textit{go-signal} pourrait alors d�grader la performance de classification. \\

% Paragraphe 5-Pour quoi est-ce que malgr�s ces limitations nous avons confiance dans les r�sultats obtenus?

Pour pouvoir g�n�raliser les �tudes 2 et 3 aux sujets sains, les �lectrodes proches des yeux ou contenant une activit� �pileptiforme sont syst�matiquement rejet�es. De plus, les analyses sont appliqu�es � travers les essais ce qui permet de minimiser l'activit� c�r�brale qui n'est pas directement reli�e � la t�che. M�me si les six sujets offrent une couverture partielle, ils ont �t� s�lectionn�s pour leur implantation pr�-frontale, frontale et pari�tale \cad l� o� l'on attend les meilleurs r�sultats de d�codage d'une t�che motrice. L'�tude de la lat�ralit� est une limitation propre � tout les enregistrements invasifs et devrait donc �tre trait� avec des enregistrements d'EEG de scalp ou de MEG. L'impact du d�lai fixe et de l'alignement sur le \textit{go-signal} de la t�che ont �t� minimis�s de fa�on diff�rentes dans chacune de ces deux �tudes. Dans la premi�re (cf. \ref{Etude2_encodage}) la fen�tre de d�codage consid�r�e pour diff�rencier les �tats moteurs se situent 500ms apr�s le \textit{go-signal} et donc, devrait limiter l'utilisation d'une activit� neuronale non-pertinente. Dans le second article (cf. \ref{Etude3_decodage}, le d�codage des directions se fait dans un esprit \textit{temps-r�el}. Un classifieur est syst�matiquement r�-entra�n� puis test� dans les diff�rentes fen�tres temporelles d�finies (67 au total), du d�but du repos � la fin de l'ex�cution. En cons�quence, seule 2/67 fen�tres qui suivent les deux indications visuelles peuvent potentiellement montrer des d�codages plus bas. \\

% Paragraphe 6-R�sumer l'apport de d�veloppement logiciel (re-mentionner les packages, et le nombre de lignes de code par package). Indiquer qu'ils peuvent �tre utilis�s pour des applications diverses et vari�es au-del� de ce pourquoi ces outils ont �t� d�velopp�s au d�part (les donn�es motrices en intra etc.). Indiquer en quelques lignes des d�veloppments futurs pr�vus pour certains de ces outils et le d�sir d'encourager d'autres personnes � contribuer activement au d�veloppment de ces outils das un esprit de partage/communautaire etc  (ce ne sont que des id�es bien sur)

Pour finir, tout les r�sultats d'analyses et de visualisation de cette th�se sont issus des logiciels Python open-source d�velopp�s � cet effet. \textit{Brainpipe} ($7899$ lignes) permet d'extraire des marqueurs de l'activit� neuronale (tel que l'amplitude, puissance, phase \dots) et de les classifier en utilisant \textit{Scikit-Learn}. \textit{Tensorpac} ($2230$ lignes), autre paquet Python open-source, permet de calculer le \pacFR avec une impl�mentation modulaire des m�thodes le tout reposant sur des calculs bas�s sur des tenseurs et mise en parall�le. Pour finir \textit{Visbrain} ($30771$ lignes) permet d'obtenir de visualiser de nombreux types de donn�es (pour du \textit{data-mining}, avec cerveau MNI semi-transparent, donn�es de sommeil \dots). Ces trois principaux paquets ont d'abord �t� d�velopp�s pour couvrir nos besoins. Puis, dans un d�sir de partager nos outils, ils ont �t� enti�rement reformat�s pour �tre utilisable par d'autre. \textit{Brainpipe} �tant le premier paquet d�velopp�, pourrait �tre tr�s largement am�lior� (surtout dans la gestion des dimensions des donn�es et de la m�moire RAM). \textit{Tensorpac} et \textit{Visbrain} n'ont pas ce soucis puisqu'ils ont �t� entam�s apr�s une connaissance plus mature du langage Python. Toutefois, de nombreuses autres m�thodologies pourraient �tre ajout�es � \textit{Tensorpac}, certaines �tant plus dur que d'autres � �tre impl�ment�es en tenseurs. Pour finir, \textit{Visbrain} � une longue feuille de route de pr�vue pour son d�veloppement futur, incluant de nombreuses refontes, am�liorations et surtout l'ajout de nouveaux modules de visualisations. Quoi qu'il en soit, avec leur qualit�s et leurs d�fauts, ces trois logiciels viennent avec un code extr�mement comment�, des datasets et exemples et des documentations. Mon plus grand souhait �tant que d'autres personnes participent � faire avancer ces projets tout en conservant l'id�e principale, un partage libre pour une science ouverte. 

\chapter*{Conclusion et perspectives futurs}

[1/2 page suffira]

2 id�es � d�velopper

(i) Un truc du genre "Taken together, this body of research advances our understanding of the role of oscillatory power, phase and PAC in goal-directed behavior, and provides efficient open-source packages for the scientific community to replicate and extend this research."

(ii) Proposer des pistes prometteuses pour l'avenir: Par exemple le fait de s'orienter d�finitivement vers des �tudes du syst�me moteur dans de grandes cohortes de sujets (via le data sharing et �tude multi-centriques) et via le open-science et le partage des outils d'analyses pou rrenforcer la reproductiblit� des r�sultats etc.

 
% ==================================================================
% ANNEXES
\appendix
\todo[color=red!40]{Mettre juste la page de références de brainpipe}

% *****************************************************************
%                   COMPARATIF PAC
% *****************************************************************
\section{Comparatif de methodes PAC \citep{tort_measuring_2010}}
\begin{figure}[H]
	\centering
	\includegraphics[scale=0.4,angle=-90]{./figures/PAC_methods_comparison}
	\caption{Comparatif de methodes PAC \citep{tort_measuring_2010}}
	\label{comp_pac}
\end{figure}

% *****************************************************************
%                   CLASSIFICATION PIPELINE
% *****************************************************************
\section{Pipeline standard de classification}
\begin{figure}[H]
	\centering
	\makebox[\textwidth][c]{\includegraphics[scale=0.9]{./figures/clf_pipeline}}
	\caption{Pipeline standard de classification}
	\label{clf_pip}
\end{figure}

% *****************************************************************
%                   COMPARATIF CLASSIFIEURS
% *****************************************************************
\section{Comparatif de classifieurs \citep{scikit-learn}}
\begin{figure}[H]
	\centering
	\makebox[\textwidth][c]{\includegraphics[scale=0.35,angle=-90]{./figures/classifier_comp}}
	\caption{Comparatif de classifieurs \citep{scikit-learn}}
	\label{comp_clf}
\end{figure}


% *****************************************************************
%                   SCHEMA IMPLANTATION
% *****************************************************************
\section{Exemple de sch\'{e}ma d'implantation}
\begin{figure}[H]
	\centering
	\makebox[\textwidth][c]{\includegraphics[scale=1]{./figures/schema-implantation_annexe}}
	\caption{Exemple de sch\'{e}ma d'implantation}
	\label{schema_implantation}
\end{figure}

 
% ==================================================================
% BIBLIOGRAPHIE
\bibliographystyle{apalike}
\bibliography{Manuscrit}
 
% ==================================================================
% COLOPHON
%\colophon{Ce document a �t� pr�par� � l'aide de l'�diteur de texte GNU
%  Emacs et du logiciel de composition typographique \LaTeXe.}
 
% ==================================================================
% COUVERTURE : RESUME ET MOTS-CL�S
\abstractpage


\end{document}
%%% Local Variables:
%%% mode: latex
%%% TeX-master: t
%%% End:
