\documentclass[sommairechap,stylejchiquet]{these_gi}



\begin{document}

% ==================================================================
% OPTIONS D'AFFICHAGE
% non-d�finitif (soumis aux rapporteurs) ou  d�finitif
\definitiftrue
% \definitiffalse

% ==================================================================
% RENSEIGNEMENTS SUR LA TH�SE
\titleFR{D�codage des intentions et des repr�sentations motrices chez l'homme: analyse multi-�chelle et application aux interfaces cerveau-machine}
\titleEN{Le titre en anglais}
\abstractFR{Le r�sum� en fran�ais ($\approx$ 1000 caract�res)}
\abstractEN{Le r�sum� en anglais ($\approx$ 1000 caract�res)}
\keywordsFR{Les mots-cl�s en fran�ais}
\keywordsEN{Les mots-cl�s en anglais}
 
\author{Etienne Combrisson}
\address{e.combrisson@gmail.com}
\universite{universit� claude bernard lyon 1}
\laboratoire{}
\specialite{sp�cialit� de la th�se}
\datesoutenance{09/2016}
\datesoumission{la date de soumission aux rapporteurs}
\jury{\begin{tabular}{llll}
    M\up{me} & \textsc{Erika Rat�} & Universit� � la Menthe & (Rapporteur) \\
    M. & \textsc{Jacques Ouille} & Universit� � la Fraise & (Rapporteur) \\
    M. & \textsc{Henri Zoto} & Laboratoire laborieux & (Rapporteur) \\
    M. & \textsc{Jean File} & Indienne & (Directeur) \\
       & etc. &  \\
  \end{tabular}    
}
{
% ==================================================================
% D�DICACE
\dedicate{� mes parents, Isabelle et Didier Pr�vot,\\ Merci}
 
% ==================================================================
% DEBUT DE LA PR�FACE
\beforepreface
 
% ==================================================================
% COUVERTURE : RESUME ET MOTS-CL�S
\abstractpage
 
% remerciements
\include{Remerc/remerc}
 
% table des mati�res g�n�rale
{
\hypersetup{linkcolor=black}
\tableofcontents
 
% affiche la liste des figures
\newpage
\listoffigures
}
% ==================================================================
\afterpreface

% ==================================================================
% NOTATIONS
\chapter*{Notations}
\addcontentsline{toc}{chapter}{Notations}

\pagestyle{plain}

%----------------------------------
% GENERAL
%----------------------------------
\Large G\'en\'eral \\% \bigskip
\normalsize
\begin{supertabular}{ll}
  ICM & Interface Cerveau-Machine \\
  BCI & Brain Computer Interface \\
\end{supertabular}

%----------------------------------
% ENREGISTREMENTS
%----------------------------------
\vspace{1cm}
\Large Enregistrements \\% \bigskip
\normalsize
\begin{supertabular}{ll}
  EEG & \eeg \\
  MEG & \meg \\
  SUA & \sua \\
  MUA & \mua \\
  SEEG & \seeg \\
  ECoG & \ecog \\
\end{supertabular}

%----------------------------------
% FEATURES
%----------------------------------
\vspace{1cm}
\Large Features \\% \bigskip
\normalsize
\begin{supertabular}{ll}
  PAC & Phase Amplitude Coupling \\
\end{supertabular}

%----------------------------------
% CLASSIFIEURS
%----------------------------------
\vspace{1cm}
\Large Classifieurs \\% \bigskip
\normalsize
\begin{supertabular}{ll}
  LDA & \lda \\
  SVM & \svm \\
  RF & \rf \\
  KNN & \knn \\
  NB & \nb \\
\end{supertabular}

\chapterend

% ==================================================================
% AVANT-PROPOS
\setcounter{secnumdepth}{4}
% *****************************************************************************
% *****************************************************************************
%                                   INTRODUCTION
% *****************************************************************************
% *****************************************************************************
\part{Introduction}
\pagestyle{headings}

\chapter*{Probl�matique}

L'objectif principal de cette th�se est d'utiliser les outils de d'apprentissage machine (ou \textit{machine-learning}) pour extraire des marqueurs de l'activit� neuronale, un marqueur �tant un motif qui appara�t de mani�re robuste et syst�matique lorsque la t�che se r�p�te. L'approche traditionnelle d'analyse, dite \textit{hypothesis-driven}, consiste � formuler une hypoth�se cibl�e en amont qui est ensuite confirm�e ou rejet�e par les r�sultats des analyses. L'approche de cette th�se s'inscrit dans une d�marche r�cente en neuroscience et consiste en une exploration dite \textit{data-driven}. Cette approche, qui est devenue envisageable gr�ce aux progr�s informatique, permet une investigation plus large et plus ensembliste dont la strat�gie est dict�e par les donn�es. \\
Cette m�thodologie d'apprentissage machine a souvent �t� utilis�e sur des donn�es c�r�brales dans le contexte des \icm (ICM). Dans un premier temps, la machine apprend � reconna�tre des motifs, ou \textit{pattern} associ�s � un �tat cognitif particulier puis, dans un second temps, les motifs reconnus sont transcrits en commandes. Ces approches sont particuli�rement prometteuses mais leur acuit� d�pend grandement du choix des attributs c�r�braux. En cons�quence, pour am�liorer ces approches il faut am�liorer le choix des marqueurs et donc, am�liorer notre connaissance des corr�lats neuronaux li�s aux processus moteur. Cette exploration de la pertinence de divers attributs oscillatoires propres aux intentions motrices repr�sentent justement ce que nous proposons d'entreprendre dans ce travail de th�se. Ainsi, les outils de \textit{machine-learning} ont �t� employ�s pour identifier les m�canismes les plus pertinents et permettant de pr�dire \textit{(a)} les �tats moteurs et \textit{(b)}, les directions de mouvement, que ce soit durant son ex�cution ou sa pr�paration. \\
L'approche \textit{data-driven} permet certes une investigation � grande �chelle mais requi�re en cons�quence des outils efficaces pour un temps d'ex�cution raisonnable et d'autre part, des outils adapt�s aux formats des donn�es neuroscientifiques. Dans ce cadre, une partie importante de cette th�se a �t� d�di� � la mise en place d'outils informatiques permettant d'analyser et de visualiser les r�sultats d'analyses. Ces outils logiciels d�velopp�s en Python constituent en soi une contribution m�thodologique significative de cette th�se. Ces paquets \textit{open-source} � la disposition de la communaut� scientifiques sont au nombre de trois et comptabilisent $52533$ lignes de code pures, plus de $90000$ en incluant leurs documentations respectives disponible en ligne. L'int�gralit� des analyses et des figures pr�sent�s durant cette th�se sont issus des logiciels que nous avons impl�ment�. \\
En introduction nous poserons le cadre li� au d�codage de l'activit� neuronale et aux m�canismes c�r�braux impliqu�s dans les processus moteurs. Puis nous pr�senterons les m�thodologies associ�es ainsi que les solutions informatiques mises en place. Puis nous pr�senterons un premier article m�thodologique en \textit{machine-learning} (Article 1 (\ref{seuil_chance})), deux articles sur le d�codage des intentions motrices � partir de donn�es intrac�r�brales (Article 2 (\ref{Etude2_encodage}) et 3 (\ref{Etude3_decodage})) et pour finir, trois articles qui pr�sentent les d�veloppements logiciels (Articles 4 (\ref{Etude4_tensorpac}), 5 (\ref{Etude5_visbrain}), 6 (\ref{Etude6_sleep})).

% #############################################################################
%                                CORPS DE L'INTRO
% #############################################################################
% #############################################################################
%                      Classification des signaux c�r�breaux
% #############################################################################
\chapter{D�codage de l'activit� c�r�brale}


% -----------------------------------------------------------------------------
% -----------------------------------------------------------------------------
%                            DEFINITION D UNE ICM
% -----------------------------------------------------------------------------
% -----------------------------------------------------------------------------
\section{Les \icm}

\subsection{D�finition et objectifs}
Cette premi�re section a pour but d'introduire et de d�finir le concept d'\icm. Nous verrons dans quel contexte elles sont apparues, les personnes � qui elles sont destin�es ainsi que les principaux �l�ments qui les composent. \\
Point de vue lexicale, on utilisera indiff�remment \textit{Interface Cerveau-Machine (ICM)}, \textit{Interface Cerveau-Ordinateur (ICO)} ou les termes anglais correspondant � savoir \textit{Brain Computer Interface (BCI)} et \textit{Brain Machine Interface (BMI)}.

% ********************************************
%           CONTEXTE D'APPARITION
% ********************************************
\subsubsection{Contexte d'apparition des ICM}
En 1964, Dr. Grey Walter connecte des �lectrodes directement dans le cortex moteur d'un patient et lui demande de presser un bouton pour faire avancer un r�tro-projecteur. En m�me temps, il enregistre l'activit� neuronale de telle sorte que elle aussi, puisse le faire avancer. L� o� l'exp�rience devient remarquable, c'est que le r�tro-projecteur avance avant que le patient ne presse le bouton ! Tout l'appareil musculaire du sujet est court-circuit� et le contr�le se fait sans mouvement. Contr�ler par la \textit{pens�e}, un sujet de science fiction qui devient une r�alit�. Cette anecdote d�crite par \cite{graimann_braincomputer_2009}, permet de placer la naissance de la possibilit� d'une \icm (ICM) dans l'histoire. C'est le point d'entr�e qui a ensuite conduit une grande diversit� de chercheurs � se passionner pour ce sujet. \\
Le terme \textit{Brain Computer Interface} fait son apparition, au d�but des ann�es 70, dans les publications de Jacques Vidal \citep{vidal_toward_1973, vidal1977real} o� il �tait question de contr�ler un curseur sur un �cran. \\
La progression des ICM et de l'int�r�t de la communaut� scientifique � v�ritablement commenc� dans les ann�es 2000. Trois facteurs sont � l'origine \citep{wolpaw_brain_2002}:
\begin{enumerate}
	\item Une am�lioration des connaissances des processus neuro-physiologiques et des techniques d'imagerie.
	\item L'arriv�e d'ordinateur bon march� et l'am�lioration constante de leur performances et des composants �lectroniques (processeurs, m�moire vive, logiciel...)
	\item Une prise de conscience soci�tale des besoins de personnes souffrant de probl�mes neuro-musculaires.
\end{enumerate}
A noter que, � l'heure actuelle, l'arriv�e de cartes graphiques bon march� est entrain de r�volutionner l'approche computationnelle des ICM permettant des calculs plus lourds en moins de temps (notamment pour le \textit{Deep Learning}). Outre la continuelle am�lioration des composants et de leur miniaturisation, la prochaine r�volution concernera certainement les ordinateurs quantiques, d�j� en phase de test dans les domaines de la g�n�tique et de la chimie. 

% ********************************************
%               INTERACTIONS NATURELLES
% ********************************************
\subsubsection{Interactions naturelles avec l'environnement}
Pour interagir avec son environnement, l'individu se sert des voies de communications naturelles, \cad via son syst�me nerveux et musculaire. Le processus de communication d�bute par une intention qui active certaines r�gions dans le cerveau. Il en r�sulte un signal c�r�brale qui est ensuite envoy� par le syst�me nerveux p�riph�rique en directions des muscles \citep{besserve_analyse_2007}. C'est ce processus simplifi� qui permet � une personne d'interagir avec ce qui l'entoure.

% ********************************************
%           COMMUNICATION ALTERNATIVE
% ********************************************
\subsubsection{Un canal de communication alternatif}
Il existe plusieurs maladies ou accidents qui entra�nent une d�g�n�rescence des performances motrices. Parmi elles, ont peut par exemple citer la Scl�rose Lat�rale Amyotrophique (ou SLA), les accidents vasculaires c�r�braux, certaines formes de scl�rose, les l�sions de la moelle �pini�re... Toutes ont en commun la possibilit� de probl�me moteur. Dans ce contexte, \cite{wolpaw_brain_2002} introduit trois options pour restaurer ces fonctions:
\begin{enumerate}
	\item Augmenter les capacit�s des facult�s motrices restantes. Autrement dit, donner un sens nouveau aux mouvements que l'individu est toujours en capacit� de faire. A titre d'exemple, le guitariste virtuose Jason Becker, reconnu pour sa v�locit� et dont tout le monde s'entendait sur son incroyable talent, f�t un jour frapp� par la SLA le conduisant au fur et � mesure � l'immobilit� totale. Il convenu alors d'un langage bas� sur les mouvements oculaires et, avec la complicit� de son p�re, continua de composer.  
	\item Contourner la l�sion. L'auteur donne � titre d'exemple, une l�sion de la moelle �pini�re que l'on peut contourner en utilisant l'activit� des muscles situ�s au dessus de la l�sion pour stimuler les muscles paralys�s.
	\item Enfin, la derni�re fa�on de restaurer des fonctions motrices qui prend tout son sens lorsque les deux pr�c�dentes ne sont pas possibles, c'est d'�tablir un nouveau canal de communication directe entre le cerveau et un ordinateur, et ce, ind�pendamment de l'activit� musculaire. D'o� le nom, \textit{\icm}.
\end{enumerate}
Une ICM est un autre syst�me de communication o� les voies naturelles sont cout-circuit�es. Au lieu de passer par le syst�me nerveux puis musculaire, le signal c�r�bral est directement intercept� au niveau du cerveau et va ensuite �tre transform� en commandes. Une ICM est donc un syst�me permettant de traduire une activit� neuronale en commande ext�rieure. Le terme \textit{traduire} est � prendre au sens linguistique \cad que les signaux c�r�braux forment un langage, compos� de r�gles, de motifs ou \textit{pattern}, que l'on va essay� de d�coder (via un ordinateur) pour les transformer en op�rations. D'o� le terme "\icm". \\

\cite{pfurtscheller_rehabilitation_2008} et \citep{graimann_braincomputer_2009} introduisent quatre �l�ments qui composent une ICM:
\begin{enumerate}
	\item Enregistrer l'activit� directement depuis le cerveau. Cet enregistrement pourra �tre invasif ou non-invasif (cf. \ref{invasif_non-invasif})
	\item G�n�rer un retour ou \textit{feedback} pour l'utilisateur
	\item L'enregistrement et le \textit{feedback} doivent �tre en temps r�el
	\item Enfin, l'interface doit �tre contr�lable par l'utilisateur, de mani�re active, via un ensemble d'intentions. 
\end{enumerate}
A titre d'exemple et pour illustrer ce dernier point, un utilisateur pourrait par exemple d�cider de bouger un curseur de souris sur un �cran en imaginant des mouvements soit de la main gauche soit de la main droite.


% ********************************************
%         COMPOSANTES D'UNE ICM
% ********************************************
\subsubsection{Principales composantes d'une ICM}
M�me si chaque \icm se destine � une utilisation particuli�re et contient des traitements qui lui sont propres, on peut globalement dire qu'une ICM s'articule autour de cinq grandes �tapes ordonn�es \citep{pfurtscheller_rehabilitation_2008,graimann_braincomputer_2009}:
\begin{enumerate}
	\item L'acquisition de l'activit� neuronale
	\item Les pr�-traitements
	\item L'extraction de marqueurs
	\item La classification
	\item La transformation en commande
\end{enumerate}
Ces �tapes sont interd�pendantes \cad que chacune s'appuie sur les r�sultats de l'�tape pr�c�dente. De plus, cette cascade de stades doit se faire en temps r�el pour que l'exp�rience utilisateur soit la plus fluide possible et qu'elle refl�te fid�lement ce qui se d�roule � chaque instant.

% -> Acquisition de l'activit� neuronale :
\paragraph{Acquisition de l'activit� neuronale}
L'acquisition de l'activit� neuronale constitue le point d'entr�e d'une ICM. Les diff�rentes techniques pour enregistrer sont plus ou moins accessibles (certaines sont portatives, d'autres n�cessitent un appareil tr�s lourd...). C'est, entre autre, l'accessibilit� qui va influer sur le nombre de sujets d'une �tude. Autre point tr�s important que nous d�crirons plus bas, la qualit� du signal (ou le rapport signal sur bruit (RSB)) qui aura un impact imm�diat sur les performances et sur les limitations d'une ICM. Enfin, on parlera d'enregistrements \textit{invasifs} (cf. \ref{sec_invasif_recordings}) quand ceux-ci n�cessiteront une implantation chirurgicale d'�lectrodes et \textit{non-invasif} (cf. \ref{sec_noninvasif_recordings}) pour les techniques d'acquisition se faisant en dehors de la bo�te cr�nienne.

% -> Pr�-traitements :
\paragraph{Pr�-traitements}
Les pr�-traitements regroupent un ensemble de techniques destin�es � nettoyer le signal pour faire ressortir, autant que possible, le signal utile par rapport au signal bruit�. Parmi ces traitements, on peut citer le nettoyage d'artefacts oculaires, cardiaques ou musculaires, le r�f�rencement (essentiellement pour l'EEG), la bipolarisation (pour la SEEG), le filtrage pour supprimer certaines composantes spectrales... Ces pr�-traitements sont propres � chaque technique d'enregistrement. Une description plus d�taill�e des pr�-traitements appliqu�s dans le cadre de donn�es SEEG est propos�e dans la section \ref{SEEG_preprocessing}.

% -> Extraction des marqueurs :
\paragraph{Extractions de marqueurs}
\label{subsec_marqueurs_ICM}
En imaginant que l'on r�p�te dix fois le m�me mouvement, il y aura dans l'activit� neuronale une partie similaire permettant de reproduire chacune de ces r�p�titions. Le signal entier sera tr�s probablement diff�rent � chaque fois, mais, � l'int�rieur de ce signal, on pourra trouver un "sous-signal" dont le contenu sera similaire � chacune de ces r�p�titions. \\
C'est le but de cette �tape d'extraction de marqueurs, la recherche de ce "sous-signal". Une fois l'activit� c�r�brale nettoy�e, on va chercher � extraire des marqueurs qui mat�rialisent l'�tat instantan� d'un sujet. Par exemple, si celui-ci bouge le bras vers la gauche ou vers la droite, on doit pouvoir extraire une information de ce signal qui encode chacun de ces �tats. En pratique, on peut distinguer deux types de marqueurs: les marqueurs \textit{locaux}, qui refl�tent l'activit� d'une "petite" population de neurones prises localement (cf. \ref{sec_marqueurs_AN}), et les marqueurs d'\textit{interaction} qui quantifie un degr�s de couplage � distance entre deux r�gions du cerveau. \\
Cette �tape est v�ritablement au c\oe ur du bon fonctionnement d'une ICM puisque, en fonction de la qualit� de ce marqueur, la machine sera plus moins encline � reconna�tre les diff�rentes commandes d'un sujet. \\    
En lieu et place du terme \textit{marqueur}, on pourra utiliser indiff�remment \textit{motif}, \textit{pattern}, \textit{feature} ou \textit{attribut}.  

% -> Classification :
\paragraph{Classification}
En reprenant l'exemple de la section pr�c�dente, supposons que l'on dispose d'un marqueur et on cherche � savoir si celui-ci appartient � une des deux classes de mouvement de bras, vers la gauche ou vers la droite. C'est le probl�me de classification. A partir d'un motif dont on ignore la provenance, on utilise un algorithme permettant de reconna�tre la classe dont est issue ce pattern. Parmi les algorithmes les plus fr�quemment rencontr�s, on peut citer le \lda, le \svm ou le \knn.\\
En pratique, cette reconnaissance de classes est mise en place en deux �tapes:
\begin{enumerate}
	\item L'entra�nement ( ou \train): durant une certaine p�riode, on va apprendre � une machine � reconna�tre des �v�nements. Pour cela, on utilise des donn�es dont on conna�t la provenance (c'est ce que l'on appelle la \textit{labellisation})
	\item Le test ( ou \test): une fois la machine entra�n�e � partir d'une s�rie de marqueurs lab�lis�s, on teste l'algorithme avec des nouvelles donn�es pour �valuer l'acuit� de la machine � identifier ces �v�nements.
\end{enumerate}
Quelque soit l'algorithme de classification, celui-ci doit s'adapter � chaque utilisateur suivant trois niveaux \citep{wolpaw_brain_2002}:
\begin{enumerate}
	\item En premier lieu, et de mani�re assez �vidente, il doit pouvoir s'adapter au marqueur du sujet
	\item Ensuite, l'algorithme doit s'adapter en temps-r�el aux variations spontan�es. En effet, l'exp�rience utilisateur va varier en fonction d'un certain nombre de param�tres comme le moment de la journ�e, la fatigue, la maladie, le taux d'hormones, la motivation/concentration/frustration... \citep{curran_learning_2003}
	\item Enfin, il doit permettre de prendre en compte l'adaptation du sujet. Durant l'exp�rience, l'utilisateur module son activit� et fournit des efforts pour s'adapter au fonctionnement de la machine. En contrepartie, le software doit prendre en compte cette am�lioration en fournissant des performances accrues.
\end{enumerate}
Cette �tape de classification, d�crite plus largement dans la section \ref{methodo_classification}, est tr�s fortement d�pendante de la qualit� des marqueurs extraits en amont. Autrement dit, plus ces \textit{features} refl�tent fid�lement un �tat, plus le travail de la machine � identifier ces �v�nements sera facilit�.

% -> Commande :
\paragraph{Transformation en commande}
Derni�re �tape du processus d'une ICM, lorsque l'algorithme de classification pense avoir identifier le type de marqueurs, on attribue une commande physique. Par exemple, si la machine reconna�t un mouvement de bras vers la droite, on pourrait attribuer une commande o� l'on d�place le curseur d'une souris sur un �cran dans la m�me direction. Ainsi, on procure � l'utilisateur un \textit{feedback} sur la transformation que la machine a r�ussit � faire � partir de l'activit� de son cerveau.\\
Cette transformation en commande est ensuite physiquement appliqu�e � un syst�me externe. L'efficacit� de l'ensemble de l'ICM est donc �valu�e en fonction de son acuit� � restituer, avec plus ou moins de fid�lit�, la commande d�sir�e par l'utilisateur.

\figScaleDesciption{0.7}{Pfurtscheller2008_SchemaICM}{Sch�ma d'une \icm \citep{pfurtscheller_rehabilitation_2008}}{L'activit� neuronale est enregistr�e (Signal acquisition) puis nettoy�e (Preprocessing). Ensuite, on extrait des motifs ou patterns qui caract�risent la commande que souhaite envoyer le sujet (Feature extraction). Enfin, la machine tente de reconna�tre ces motifs (Classification) et de les transformer en commande (Application interface). Cette boucle se termine en donnant un feedback � l'utilisateur sur l'�tat actuel de la machine. \citep{pfurtscheller_rehabilitation_2008}}

Les ICM partagent globalement ces cinq �tapes mais se diff�rencient donc le type d'enregistrement de l'activit� neuronale, par les pr�-traitements associ�s, par le type de marqueurs �tudi�, par l'algorithme de classification choisi et surtout, par l'application concr�te de cette \icm.

% ********************************************
%           Applications des ICMs
% ********************************************
\subsubsection{Applications des ICM: cliniques et non-cliniques}
La r�alisation concr�te des ICM s'est articul�e autour des applications cliniques, \cad destin�es � essayer d'am�liorer les conditions de vie de certaines personnes puis, dans un second temps, pour des applications ludiques et tout publique.

\paragraph{Applications cliniques}
A l'heure actuelle, il existe de nombreuses maladies soit dont on ignore l'origine, soit qui sont pour le moment incurables. Parmi ces pathologies, certaines �voluent en enfermant les patients dans des conditions de vies difficiles. C'est dans ce contexte clinique qu'apparaissent de nombreuses ICM qui ne sont donc pas des traitements, mais bien des solutions \textit{temporaires} pour aider certaines personnes � mieux vivre avec leur handicap.
\subparagraph{Pathologies et ICM :} Pr�lever directement l'activit� neuronale et donc, \textit{bypasser} les voies naturelles, permet de s'affranchir d'�ventuelles limitations physiologiques. C'est pourquoi les ICM repr�sentent un enjeu majeur pour la r�habilitation motrice ou handicap moteur ou encore pour la communication palliative. Les applications concr�tes des \icm visent donc les personnes disposant de leurs capacit�s cognitives mais qui sont priv�es de facult�s motrices. \\
Un accident vasculaire c�r�brale (AVC) peut engendrer un �tat d'enfermement (\textit{locked-in state (LIS)}). Les personnes dans cet �tat sont pleinement conscientes de leur corps, de l'environnement, ils peuvent ressentir les sensations de toucher et de douleur mais n'ont plus de facult�s motrices, hormis peut-�tre, les mouvements de paupi�res ou des yeux. Un autre exemple est celui de la scl�rose lat�rale amyotrophique (ou SLA) qui est une d�g�n�rescence des neurones moteur (motoneurones). Progressivement, les personnes atteintes de SLA perdent l'usage des bras, des jambes, de la parole des muscles faciaux et enfin de la d�glutition mais la conscience et les facult�s cognitives demeurent intactes. L'�volution de la maladie am�ne � deux possibilit�s \citep{chaudhary_brain-machine_2015}: accepter une d�pendance totale (respiration artificielle et nutrition) ou un d�c�s par insuffisance respiratoire. Dans le premier cas, la maladie entra�ne progressivement les patients en LIS ne leur laissant qu'un minimum de fonctions motrices. Lorsque le patient perd tout contr�le musculaire, ce qui finit souvent par les muscles des yeux, il rentre dans un �tat d'enfermement complet (\textit{completely locked-in state (CLIS)}). \\
SLA et \textit{locked-in syndrome} ne sont que deux exemples expliquant l'int�r�t social du d�veloppement des ICM. Redonner un peu de contr�le ou �tablir un canal de communication avec les familles des patients expliquent l'engouement qui existe depuis maintenant plus de 40 ans pour les ICM.

\figScaleDesciption{0.5}{SLA_physio2}{Paralysie caus�e par la SLA et accident vasculaire c�r�brale \citep{kubler2001brain}}{Paralysie caus�e par la SLA et accident vasculaire c�r�brale \citep{kubler2001brain}}

\textbf{Exemples d'applications:} les applications cliniques peuvent �tre divis�es en deux grandes familles: les ICM destin�es � r�tablir un lien de communication (\textit{communication palliative}) et les ICM pour redonner de la mobilit�. \\

\textbf{Communication palliative:} ces ICM pr�sentent en g�n�ral des lettres ou groupe de lettres qui doivent pouvoir �tre s�lectionn�s volontairement par l'utilisateur. Le \textit{P300-Speller} \citep{farwell_talking_1988, donchin_mental_2000}, utilisant l'onde P300, est sans aucun doute l'interface la plus connue et la plus exploit�e � ce jour pour la communication palliative. Le \textit{Hex-o-spell} \citep{blankertz_note_2007} est une autre interface contr�lable par l'utilisateur via l'imagerie motrice. \\

\textbf{Mobilit�, robotique et proth�se:} ces syst�mes sont destin�s � aider, am�liorer ou remplacer des facult�s motrices l�s�es. Ces applications dont la mobilit� est au centre peuvent afficher diff�rents degr�s de complexit�:
\begin{itemize}
	\item Curseur $1D$, $2D$ et $3D$: certainement la premi�re �tape, le but ici est de permettre au sujet de pouvoir contr�ler le curseur sur un moniteur. Ce curseur pourra par exemple servir de bouton \textit{yes/no} ou pour le d�placement \citep{wolpaw1991eeg, wu_neural_2003, trejo2006brain, kayagil2009binary, sung-phil_kim_point-and-click_2011, vadera_stereoelectroencephalography_2013} ou tout simplement pour de la navigation comme le \textit{web-browsing} \citep{mugler_design_2010}. Ces cat�gories de d�placement de curseur diff�rent par le nombre de degr�s de libert� qu'elles offrent.  
	\item Contr�le d'un fauteuil roulant: comme le nom l'indique, l'id�e ici est de permettre � un utilisateur de contr�ler son fauteuil roulant via son activit� neuronale seulement. En effet, si de nombreuses personnes handicap�es peuvent encore se servir de leur membres sup�rieurs, d'autres sont compl�tement d�pendantes. Ce type d'applications permettra donc � terme de redonner une lib�rt� de mouvement � ces personnes \citep{tanaka2005electroencephalogram, leeb_self-paced_2007, galan_brain-actuated_2008, philips_adaptive_2007, vanacker_context-based_2007, pires_visual_2008, rebsamen_brain_2009, lin_eeg_2010, diez_commanding_2013}.
	\item Proth�se: enfin, derni�re application li�e � la mobilit�, le contr�le de proth�se est un large d�fi. A titre d'exemple, le contr�le d'un bras robotis� doit permettre le contr�le de celui-ci dans l'espace ainsi que des mouvements de main, de coude... C'est un probl�me � haute dimensionnalit� et donc complexe. Toutefois, modulo un certain degr�s de r�ussite, certaine �quipe ont propos� de tel syst�mes \citep{fetz_real-time_1999, hochberg_reach_2012, yanagisawa_electrocorticographic_2012, sunny_robotic_2016}
\end{itemize}

\paragraph{Applications non-cliniques et r�cr�atives}
Les ICM ont �galement �t� exploit�es � d'autres fins que des applications cliniques:
\subparagraph{Jeux:} les ICM d�di�es aux jeux utilisent l'activit� neuronale pour contr�ler un vaisseau spatial, un objet ou un personnage dans un environnent virtuel \citep{lalor_brain_2004, nijholt_bci_2008, oude_bos_brainbasher_2008, coyle_eeg-based_2011}. Outre le caract�re r�cr�atif de ce type de dispositif, les ICM bas�es sur le jeux pourraient tr�s bien servir pour entra�ner la machine d'une mani�re plus ludique et moins fatigante.
\subparagraph{Art:} pour finir, les ICM peuvent avoir des applications artistiques comme pour la composition et la pratique musicale \citep{miranda_harnessing_2003, miranda_brain-computer_2006, hamadicharef_brain-computer_2010} ou pour peindre \citep{munsinger_brain_2010, zickler2013brain}

% ********************************************
%                 OPEN DATA
% ********************************************
\subsubsection{\textit{BCI competition} et \textit{open-data}}
Les \textit{BCI competitions} sont propos�es par l'�quipe du \textit{Berlin Brain-Computer Interface} (BBCI). L'id�e de ces comp�titions est, sur un m�me jeux de donn�es d�coup�es en \train et \test (cf. \ref{subsec_training_testing}) mettre les �quipes de recherche en ICM en comp�titions puis �lire celle qui arrivera au meilleur d�codage. A l'heure o� cette th�se est �crite, quatre de ces comp�titions ont eut lieu et chacune est associ�e � un article de pr�sentation (\textit{BCI competition I}: \cite{sajda_data_2003}, \textit{II}: \cite{blankertz_bci_2004}, \textit{III}: \cite{blankertz_bci_2006}, \textit{IV}: \cite{tangermann_review_2012}). La liste des \textit{datasets} en acc�s libres est disponible en annexe (cf. \ref{annexe_bci_competition}. Remarque: il semble que les donn�es de la premi�re comp�tition ne soient plus accessibles, c'est la raison pour laquelle elles n'apparaissent pas dans le tableau)\\

Ces comp�titions pr�sentent de nombreux avantages:
\begin{enumerate}
	\item Visibilit� des �quipes: ces comp�titions sont de plus en plus "virales" et sont relay�es par la suite par de nombreux articles (voir figure ci-dessous). 
	\item Permet aux �quipes de disposant pas de syst�mes d'acquisition de profiter de jeux de donn�es pour d�velopper des m�thodes. C'est �galement un moyen d'initier les �tudiants � un ensemble de techniques.
	\item Autre avantage, tout le monde se retrouve sur un pied d'�galit� en travaillant sur les m�mes donn�es. Ce qui repr�sente par la suite, une m�thode rigoureuse pour s�lectionner les meilleurs algorithmes.
\end{enumerate}

\figScaleDesciption{1}{bcicomp_visi}{\textit{BCI competition} et visibilit� \cite{tangermann_review_2012}}{(A gauche) �volution du nombre de citations par an des articles de pr�sentation pour les comp�titions \textit{I, II et III}, (A droite) �volution du nombre de citations par an pour l'�quipe ayant gagn� la deuxi�me comp�titions \citep{tangermann_review_2012}}

% ********************************************
%                CONClUSION
% ********************************************
\vspace{2\baselineskip}
\todo[inline,caption={},color=blue!10]{
	\textbf{Conclusion sur la pr�sentation des ICM} \\
	Les personnes atteintes de probl�mes moteurs sont certainement les premiers b�n�ficiaires de l'avanc�e des \icm. Ces syst�mes leur permettraient de r�tablir un canal de communication avec leur environnement. Pour cela, les ICM partent de l'enregistrement de l'activit� de leur cerveau puis, recherche dans cette activit� des motifs relatant un �tat. Si cet �tat est \textit{compris} par la machine, il sera ensuite transform� en commande. De nombreux syst�mes se sont d�velopp�s, principalement autour de la communication palliative ou de la mobilit�. L'efficacit� de ces syst�mes reposent en grande partie sur deux points: l'acuit� des algorithmes de classification, pour cette raison, ils peuvent �tre compar�s notamment gr�ce � des initiatives comme les \textit{BCI competitions}. Le deuxi�me point concerne l'enregistrement de l'activit� neuronale qui va tr�s largement conditionner la qualit� du signal et des marqueurs qui en seront extraits.    
}


% -----------------------------------------------------------------------------
% -----------------------------------------------------------------------------
%                         ACQUISITION DE L'AN
% -----------------------------------------------------------------------------
% -----------------------------------------------------------------------------
\subsection{Techniques d'acquisition de l'activit� neuronale}
\label{invasif_non-invasif}
Les types d'acquisition peuvent �tre class�s par leur degr�s de p�n�tration dans le corps. Les enregistrements dits \textit{invasifs} vont n�cessiter une intervention chirurgicale, donc avec risque potentiel d'infection, mais donnent acc�s � des signaux de grande qualit� permettant des contr�les complexes comme une proth�se ou un bras robotis� \citep{hochberg_reach_2012, taylor_direct_2002}. Les enregistrements \textit{non-invasifs} sont globalement plus faciles � mettre en place car ne n�cessitant aucune chirurgie mais la qualit� du signal est moindre car l'activit� neuronale est enregistr�e en dehors de la bo�te cr�nienne, donc filtr�e par l'os et la peau. \\
Au sein de ces deux cat�gories, on trouvera un ensemble de m�thodes d'acquisition qui se diff�rencient par la taille des populations de neurones qu'elles enregistrent ou par le type de signal (�lectrique, magn�tique, mesure indirecte...).

% ********************************************
%                  INVASIFS
% ********************************************
\subsubsection{Enregistrements invasifs}
\label{sec_invasif_recordings}
On parlera d'intracr�nien pour les enregistrements � l'int�rieur de la bo�te cr�nienne puis d'intracortical pour des �lectrodes implant�es dans le cortex \citep{engel_invasive_2005, jerbi_task-related_2009}.

% -> Micro�lectrodes :
\paragraph{Enregistrements unitaires}
\textit{\sua (SUA)} et \textit{\mua (MUA)} sont des micro-�lectrodes d�pos�es directement au contact de neurones et enregistrent des d�charges neuronales. SUA et MUA se distinguent par la taille des populations enregistr�es. Les d�charges sont �v�nements tr�s courts dans le temps, donc pour �tre en mesure de les capter, les syst�mes d'acquisition poss�dent des fr�quences d'�chantillonnage tr�s �lev�es (de l'ordre de 30khz). Le signal obtenu en filtrant en dessous de 300hz, est appel� \textit{\LFP} (LFP) et repr�sente l'activit� �lectrique d'une assembl�e de neurones prise dans un petit volume.\\
Dans le cadre des ICM, les micro-�lectrodes peuvent �tre implant�es directement dans le cortex moteur primaire et permettre un contr�le de BCI de la plus haute pr�cision et fid�lit�. Chez l'animal, des rats ont pu contr�ler en temps r�el un bras robotis� pour obtenir de l'eau \citep{chapin1999real} et des singes ont �galement pu contr�ler un bras robotis� ainsi qu'une pince au bout \citep{velliste2008cortical}. Auparavant, en 2006, \cite{hochberg_neuronal_2006} d�montre qu'un patient t�trapl�gique implant� avec 96 micro-�lectrodes, peut contr�ler un curseur sur un �cran d'ordinateur, d'ouvrir et fermer une main artificielle ainsi que d'effectuer des mouvements rudimentaires � l'aide d'un bras robotis�. En compl�ment, \cite{sung-phil_kim_point-and-click_2011} utilise les micro-�lectrodes pour contr�ler et cliquer � l'aide un pointeur sur un �cran 2D. Mais c'est en 2012 que le sujet contr�le compl�tement le bras robotis�, dans l'espace, lui permettant d'attraper et de boire son caf� \citep{hochberg_reach_2012}. M�me si le mouvement n'est pas aussi fluide et rapide qu'un mouvement r�el, ce f�t une avanc�e majeure et une  preuve du concept chez l'homme. \cite{collinger2013high} d�crivent �galement le contr�le d'un bras robotis� par un patient implant� avec 96-micro�lectrodes. Avec des taux de d�codage �lev�s, le sujet r�ussit � contr�ler un bras robotis� � 7-dimensions (3 degr�s de translation, 3 degr�s de rotation et un degr�s de saisie).

\figScaleDesciption{1}{hochberg_2006_2012}{ICM utilisant des micro-�lectrodes \citep{hochberg_neuronal_2006, hochberg_reach_2012}}{(\textit{A gauche}) Sujet t�trapl�gique implant� avec des micro-�l�ctrodes contr�lant un curseur sur un �cran \citep{hochberg_neuronal_2006}, (\textit{A droite}) \citep{hochberg_reach_2012}}


% -> St�r�o�lectroenc�halographie :
\paragraph{\seeg (SEEG)}
Macro-�lectrodes enregistrant des populations plus larges que les micro-�lectrodes. Contrairement � la SUA ou MUA o� l'on peut compter le nombre de fois qu'un ou plusieurs neurones d�chargent, la SEEG enregistre des potentiels �lectriques $(\mu V)$. La fr�quence d'�chantillonnage est plus faible que celle des micro-�lectrodes (de l'ordre de 1khz) et donne donc acc�s au signal LFP.\\
Les donn�es utilisant la \seeg sont globalement rares dans la litt�rature. Leur utilisation dans le cadre des ICM n'est pas fr�quente mais a quand m�me �t� explor� dans le cadre d'ICM d�di�e � la communication \citep{krusienski_control_2011-1,shih_signals_2012} ou au contr�le de curseur 2D \citep{vadera_stereoelectroencephalography_2013} \\
C'est le type d'enregistrement qui a �t� le plus exploit� durant cette th�se. Une section compl�te lui est donc accord�e (cf. \ref{sec_intra_data}).

% -> Electrocorticographie :
\paragraph{\ecog (ECoG)}
L'ECoG est une grille flexible compos�e d'une matrice d'�lectrodes. Celle-ci est ensuite d�pos�e � la surface corticale. Parce que cette m�thode n'est pas intracortical elle est dite \textit{semi-invasive}. En comparaison avec l'EEG (voir section suivante), l'ECoG pr�sente de nombreux atouts \citep{schalk_brain-computer_2011}. Tout d'abord, la r�solution spatiale est meilleure (de l'ordre du millim�tre contre quelques centim�tres pour l'EEG). Une meilleure amplitude dont le maximum peut-�tre jusqu'� 5 fois sup�rieure � celle de l'EEG (amplitude max pour l'EEG est d'environ 10-20$\mu V$). L'ECoG est �galement moins sensible aux artefacts (mouvements musculaires et oculaires). Enfin, elle permet d'enregistrer des ph�nom�nes plus large bande (entre 0 et 500Hz contre 0-40Hz pour l'EEG). \\
L'utilisation de l'ECoG dans le cadre des ICM est fr�quente \citep{schalk_brain-computer_2011, shih_brain-computer_2012}, que ce soit pour le d�codage appliqu� aux membres sup�rieurs comme les mouvements de doigts \citep{scherer2009classification, acharya2010electrocorticographic}, de trajectoire de main \citep{schalk2007decoding, gunduz2009mapping} ou de bras \citep{pistohl_prediction_2008}. Il a �galement �t� d�montr� qu'il est possible de d�coder des mouvements fins de saisie avec la main, comme des saisies pr�cises avec le bout des doigts versus des saisies avec la main enti�re \citep{pistohl_decoding_2012}, ou encore des mouvements de saisie, de pinc�e, d'ouverture de la main, de flexion et d'extension du coude ou de bras robotis� \citep{muller-putz_control_2008, yanagisawa_electrocorticographic_2012}. De plus, l'ECoG a �galement �t� exploit� pour le contr�le d'un curseur 1 ou 2D \citep{leuthardt_braincomputer_2004, wilson2006ecog, felton2007electrocorticographically, milekovic_online_2012} ainsi que pour la communication palliative \citep{brunner2011rapid, krusienski2011control}.

\figScaleDesciption{1}{shalk_and_yanagisawa_ecog}{Micro et macro-�lectrodes pour l'\ecog \citep{schalk_brain-computer_2011, yanagisawa_electrocorticographic_2012}}{\textit{A gauche} - (\textbf{A-B}) Micro-�lectrode, (\textbf{C}) Repr�sentation sch�matique d'une �lectrode, (\textbf{D}) Grille d'�lectrodes d�pos�e directement au contact du cortex, (\textbf{E}) Placement chirurgical de la grille d'�lectrodes \citep{schalk_brain-computer_2011}. \textit{A Droite} - D�codage de mouvements de la main et contr�le d'un bras robotis� \citep{yanagisawa_electrocorticographic_2012}}


% ********************************************
%               NON-INVASIF
% ********************************************
\subsubsection{Enregistrements non-invasifs}
\label{sec_noninvasif_recordings}

% -> Champs �lectrique et magn�tiques
\paragraph{Introduction � la production des champs �lectriques et magn�tiques}
Avant de pr�senter les diff�rentes techniques d'enregistrement non-invasives, il m'est apparu int�ressant de pr�senter succinctement la fa�on dont une population de neurones peut produire des champs �lectriques et magn�tiques. Cela permettra d'une part de comprendre un peu mieux les m�canismes sous-jacents � l'EEG et � la MEG, leur limitations et surtout, leur compl�mentarit�. \\
L'explication des production de champs �lectromagn�tique s'appuiera sur la figure ci-dessous, issue de \cite{sato_principles_1991} ainsi que sur l'article de \cite{garnero_magnetoencephalographie_1998}.

\figScaleX{0.6}{sato_champsEM}{Production des champs �lectriques et magn�tiques. Recueil EEG et MEG, limitations et compl�mentarit� \citep{sato_principles_1991}}

\textbf{a} et \textbf{b} sont deux capteurs EEG dispos�s au contact du scalp. \textbf{M} est un capteur MEG. \textbf{A} et \textbf{B} sont deux dip�les mat�rialisant la somme des courants issus d'une macro-colonne de neurones. Lorsqu'un neurone est excit�, il y a lib�ration d'ions at niveau de la membrane des synapses. Ces flux ioniques vont engendr�s des courants dits \textit{primaires} ou \textit{sources} et qui seront � l'origine des signaux EEG et MEG. Ces courants primaires vont � leur tour engendr�s des courants dits \textit{secondaires} ou \textit{volumiques}, circulant dans tout le volume de la t�te (\textit{extracellular current}). Un signal MEG r�sultera principalement des champs magn�tiques produits par les courants sources, contrairement au signaux EEG qui sont majoritairement issus des courants extracellulaires et volumiques. \textbf{A} est un dip�le tangentielle (donc parall�le � la surface du cr�ne) dispos� dans un sillon. Le champs magn�tique \textbf{C} produit par ce dip�le, qui n'est pas filtr� par les m�ninges, l'os et la peau, est bien capt� par le capteur MEG. La colonne \textbf{B}, qui correspond aux activations situ�es dans les parties courbes du cortex (gyri) est un dip�le radiale donc perpendiculaire au scalp. Celui-ci produit un champs magn�tique \textbf{D}, parall�le � la bobine \textbf{M}, qui sera donc pas ou peu d�tect� par la MEG. En revanche, cette source radiale entra�ne un fort potentiel �lectrique. C'est donc un premier point de compl�mentarit� entre la MEG et l'EEG. \textbf{A et B} engendrent des champs �lectriques qui seront fortement dispers�s, att�nu�s et distordus par la dure-m�re, l'os et la peau. Ce ph�nom�ne de dispersion explique pourquoi il est beaucoup plus difficile de reconstruire les sources (probl�me inverse) en EEG plut�t qu'en MEG. Enfin, les propri�t�s physiques du champs magn�tique font qu'il d�cro�t davantage avec la distance que les champs �lectrique, ce qui limite l'�tude des sources profondes mais c'est un deuxi�me argument justifiant leur compl�mentarit� pour observer l'ensemble des ph�nom�nes.

% -> Electroenc�phalographie :
\paragraph{\eeg (EEG)}
C'est la technique la plus utilis�e dans le domaine des ICM pour son aspect pratique, portatif et peu dispendieux. On dispose � la surface de la bo�te cr�nienne un ensemble d'�lectrodes qui enregistrent, sous forme de potentiel �lectrique, l'activit� r�sultant d'une population relativement large de neurones. Une �lectrode (souvent sur le front ou le nez) est consid�r�e comme r�f�rence. Le potentiel de cette �lectrode de r�f�rence est ensuite soustrait � toutes les autres pour obtenir une tension. On pourra utiliser un gel entre l'�lectrode et la bo�te cr�nienne pour adapter l'imp�dance. L'EEG dispose d'une excellente r�solution temporelle mais sa r�solution spatiale est moins bonne (en partie d� au fait que les m�ninges, l'os puis la peau filtre le signal). \\
Les premiers enregistrements non-invasifs utilisant l'EEG chez l'homme, ont �t� report� par \cite{berger1929ueber}. Depuis, ils n'ont cess� de s'am�liorer et � l'heure actuelle on peut trouver des syst�mes sans fils et peu dispendieux permettant le contr�le d'une BMI \citep{lin_eeg_2010, liao_gaming_2012, liu_implementation_2012}. La transmission sans fil occasionne une perte de signal suppl�mentaire, donc la plupart des �tudes utilisant l'EEG continu avec les bons vieux c�bles et amplificateur. Il existe plusieurs sous-cat�gories d'ICM utilisant l'EEG. Celles-ci sont bas�es sur diff�rents types de marqueurs (cf. \ref{sec_marqueurs_AN}).

\figScaleDesciption{1}{casque_eeg}{Exemple de casque EEG}{Exemple de casque EEG (figure extraite et adapt�e de \url{http://www-psych.nmsu.edu/~jkroger/lab/principles.html}}

% -> Magn�toenc�phalographie :
\paragraph{\meg (MEG)}
Nettement moins portable que l'EEG, la MEG enregistre les champs magn�tiques r�sultant d'une production de courants intracellulaires. Ces champs magn�tiques peuvent �tre jusqu'� 10 milliards de fois plus faibles que le champs magn�tique terrestre ce qui explique l'utilisation d'un blindage pour limiter les artefacts (un blindage en $\mu$-m�tal peut att�nuer de $10^{3}$ � $10^{4}$ l'influence des champs externes). Cet appareil, qui peut �tre compos� de 100 � 300 capteurs, dispose �galement d'une excellente r�solution temporelle et d'une r�solution spatiale sup�rieure � celle de l'EEG. \\
L'aspect inamovible de la MEG limite son utilisation pour les ICM mais on trouve quand m�me quelque �tudes ayant test� son utilisation \citep{mellinger_meg-based_2007, waldert_hand_2008}.

% -> Imagerie par R�sonance Magn�tique fonctionnelle :
\paragraph{\IRMf (IRMf)}
L'IRMf permet de mesurer l'oxyg�nation d'aires actives gr�ce � l'apport en sang. Cette technique se base sur des diff�rences de susceptibilit� magn�tique du fer entre le sang oxyg�n� (diamagn�tique) et d�soxyg�n� (paramagn�tique). Le signal BOLD (\BOLD) mesure les variations locales du temps de relaxation caus�es par les modifications h�modynamiques. Si la r�solution spatiale de l'IRMf est excellente puisqu'elle est de l'ordre du millim�tre, la r�solution temporelle, quant � elle, est assez faible (de l'ordre de la seconde) ce qui limite la captation de ph�nom�nes temporellement courts. \\
L'utilisation de l'IRMf pour les ICM a �t� exploit� \cite{sitaram_fmri_2007}. Cette �tude, qui proposent plusieurs ICM-IRMf existantes, explique qu'une des principales limitations de cette technique d'enregistrement est son co�t et la complexit� li� � son usage. Enfin, \cite{weiskopf_principles_2004} explique qu'une autre limitation est qu'il peut s'�couler entre 3 et 6 secondes pour observer les changements h�modynamiques.  

% -> Imagerie par infrarouge :
\paragraph{\fNIRS (fNIRS)}
Technique d'imagerie relativement r�cente, puisque la premi�re description du principe fut d�crite par \citep{jobsis1977noninvasive}, celle-ci exploite la lumi�re infra-rouge (longueur d'ondes entre 650 et 1000nm) pour mesurer les variations de concentration de l'h�moglobine oxyg�n�e (HbO) et l'h�moglobine d�soxyg�n�e (HbR). La principale limitation de cette technique est qu'elle ne permet pas l'�tude de structures profondes (profondeur de 3cm maximum � partir du sommet du cr�ne). \\
Cette technique est de plus en plus rencontr�e dans la litt�rature ICM d� � son moindre co�t et � sa portabilit�. La premi�re ICM utilisant la fNIRS a �t� d�crite par \cite{coyle_suitability_2004}. Dans cette �tude, les auteurs utilisent l'imagerie motrice (compression d'une balle en caoutchouc) et d�terminant si l'activit� du sujet et en activit� ou au repos. Depuis, bien d'autres �tudes ont suivi, dont beaucoup se focalisent sur l'imagerie motrice \citep{sitaram_temporal_2007, nagaoka2010development, fazli2012enhanced, mihara2013near, zimmermann_detection_2013, kaiser2014cortical}. \cite{naseer_fnirs_2015} propose une review r�cente des principales ICM-fNIRS.

\figScaleDesciption{0.8}{Waldert_2009_recording}{M�thodes d'acquisition de l'activit� c�r�brale \citep{waldert_review_2009}}{M�thodes d'acquisition de l'activit� c�r�brale \citep{waldert_review_2009} class�es par invasivit�. La figure indique �galement la taille des populations de neurones enregistr�s ainsi que la r�solution spatiale intimement li�e � l'invasivit�.}

% ********************************************
%                CONClUSION
% ********************************************
\vspace{2\baselineskip}
\todo[inline,caption={},color=blue!10]{
	\textbf{Conclusion sur les techniques d'enregistrements} \\
	Les m�thodes non-invasives ne n�cessitent aucune intervention chirurgicale et peuvent donc �tre appliqu�es sur n'importe quel individu. De plus, elles sont particuli�rement accessibles, \cad que leur mise en place est relativement simple. \\
	Toutefois, ces m�thodes de mesures souffrent encore de compromis. Si l'EEG et la MEG ont une excellente r�solution temporelle, la r�solution spatiale est peu pr�cise. A contrario, l'IRMf ou la PET offrent une excellente r�solution spatiale mais la dimension temporelle est l�s�e. Les m�thodes invasives ne souffrent pas de ces compromis, puisqu'elles offrent � la fois une excellente r�solution temporelle et spatiale, au d�triment d'une chirurgie invasive. Enfin, les m�thodes non-invasives sont beaucoup plus sensibles aux artefacts (musculaires et oculaires) et ont un rapport signal sur bruit (RSB) inf�rieur aux enregistrements invasifs.\\
	Malgr� tout, les techniques d'enregistrement non-invasives jouissent d'un engouement certain de la part de la communaut� scientifique, notamment gr�ce au crit�re d'accessibilit�. L'EEG a tr�s certainement un bel avenir devant lui, en attendant que des chercheurs r�ussissent � mettre des roulettes � la MEG ou � l'IRM. Sans nul doute, les techniques non-invasives devraient �tre au c\oe ur des Interfaces Cerveau-Machine du futur.
}

% -----------------------------------------------------------------------------
% -----------------------------------------------------------------------------
%                         DIFFERENTS TYPES D'ICM
% -----------------------------------------------------------------------------
% -----------------------------------------------------------------------------
\subsection{ICM synchrones/asynchrones et invasives/non-invasives}
On peut distinguer diff�rents types d'ICM, sur la base de deux crit�res \citep{donoghue_connecting_2002, lebedev_brainmachine_2006, besserve_analyse_2007, bekaert_les_2009}:
\begin{itemize}
	\item \textbf{La synchronisation}: ce crit�re va d�finir le fonctionnement interne de l'ICM \cad qu'il va fixer la fa�on dont un utilisateur va pouvoir interagir avec elle. Soit de fa�on volontaire en modifiant son activit� neuronale, ce sont les \textit{ICM asynchrones}, soit de fa�on impos�e par l'utilisation de stimuli externes qui permettront de piloter l'interface, ce sont les \textit{ICM synchrones}.
	\item \textbf{Le type d'enregistrement}: on distinguera les \textit{ICM invasives} des \textit{ICM non-invasives} de part l'utilisation de technique d'enregistrement n�cessitant ou non, une chirurgie pour implanter des d'�lectrodes (cf. \ref{invasif_non-invasif}).
\end{itemize}

% ********************************************
%         Synchrones et Asynchrones
% ********************************************
\subsubsection{Synchronisation: ICM synchrones et asynchrones}
\begin{enumerate}
	\item \textit{ICM synchrones} ou \textit{exog�nes}: exploitent la r�ponse du cerveau � des stimuli externe (visuels, auditifs...). Par exemple, un damier compos� de cases blanches et noires entra�nera une forte variation dans les potentiels visuels. Cette diff�rence de potentiel peut ensuite �tre d�tect�e puis transform�e en commande. Ce type d'ICM pr�sente l'avantage de n�cessiter que tr�s peu d'entra�nement. En revanche, la r�ponse �tant d�pendante du stimulus, le comportement est bool�en ce qui limite dans les possibilit�s pour un contr�le progressif et continu.
	\item \textit{ICM asynchrones} ou \textit{endog�ne}: ici, gr�ce � un \textit{feedback}, l'utilisateur change volontairement son activit� neuronale pour influer sur le comportement de l'ICM. En pratique, on pourra par exemple se servir de l'imagerie motrice pour avoir un contr�le continu et progressif d'un curseur de souris. Toutefois, les ICM asynchrones n�cessitent une longue p�riode d'apprentissage avant de pouvoir reconna�tre les \textit{patterns} propres � chaque sujet.
\end{enumerate}

% -----------------------------------------------------------------------------
% -----------------------------------------------------------------------------
%                        MARQUEURS DE L'AN
% -----------------------------------------------------------------------------
% -----------------------------------------------------------------------------
\subsection{Signaux physiologiques pour le contr�le d'une ICM}
\label{sec_marqueurs_AN}
Les signaux peuvent class�s en deux cat�gories \citep{wolpaw_brain_2002, pfurtscheller_rehabilitation_2008}:
\begin{itemize}
	\item \textbf{Les r�ponses �voqu�s} ou \textit{exog�nes}: sont produites, sans que le sujet en ai conscience, suite � un stimuli externe. 
	\item \textbf{Les r�ponses spontan�s} ou \textit{endog�nes}: celles-ci peuvent �tre volontairement modifi�es par l'utilisateur.
\end{itemize}

% ********************************************
%         Signaux �voqu�s
% ********************************************
\subsubsection{R�ponses �voqu�s}
Comme d�crit plus haut, les signaux �voqu�s sont provoqu�s par un stimuli externe et engendrent une r�ponse sp�cifique, \cad localement dans le cerveau et � des instants pr�cis. Ces signaux ainsi provoqu�s prendront des valeurs diff�rentes en fonction de l'intention du sujet. Et c'est gr�ce � ces variations de valeurs que le sujet pourra contr�ler la BMI. Le sujet est donc d�pendant des stimulus qu'on lui envoie. La premi�re cons�quence, c'est que les ICM utilisant les signaux �voqu�s ne n�cessitent pas d'apprentissage particulier. En revanche, elles peuvent entra�ner une grande fatigue pouvant alt�rer les performances de l'ICM \citep{wolpaw_brain_2002, curran_learning_2003}.Enfin, les stimulus externes peuvent �tre de nature diff�rents, que ce soit visuel, auditif ou tactile. Ce type peut s'adapter en fonction de la condition physique du sujet. \\
Ces r�ponses sont donc li�es � un �v�nement (\textit{event-related}). La plus connue et la plus ancienne est le \textit{Potentiel �voqu�} (PE) recueillis en \eeg (et puisqu'il est li� � l'apparition d'un �v�nement, on parlera d'\textit{Event-Realted Potential} (ERP)) et \textit{Champs magn�tique �voqu�} en \meg. Ces ondes sont obtenues en moyennant un grand nombre d'essais par rapport � l'apparition du stimuli, afin d'�craser les variabilit� inter-signal (bruits) et renforcer l'�mergence des ph�nom�nes communs aux essais. Parmi ces ph�nom�nes, deux sont particuli�rement exploit�s dans le cadre des ICM, les \textit{\SSEP} et l'onde \textit{P300}.

% -> SSEP
\paragraph{\SSEP (SSEP)}
Les SSEP sont une r�ponse naturelle du cerveau � des stimulus envoy�s � une certaine fr�quence. Bien souvent, le stimuli est de nature visuelle, on parlera donc de \textit{Steady-State Visual Evoked Potentials} (SSVEP). G�n�ralement, les stimulus visuels sont envoy�s � des fr�quences comprises entre 3.5hz et 75hz et g�n�rent, dans le cortex visuel, des r�ponses aux m�mes fr�quences (et harmoniques) \citep{wolpaw_brain_2002, beverina_user_2003}. \\
\cite{sutter_special_1992} d�crit une BCI bas�e sur les SSVEP (SSVEP-BCI) o� l'utilisateur fait face � un �cran compos� de lettres et de symboles dispos�s dans une matrice 8x8. Chaque groupe de lettre est flash� � des vitesses diff�rentes pour �tablir le profil standard du sujet. On demande ensuite � l'individu de choisir une lettre. En flashant de nouveau les �l�ments de la matrices, les r�ponses dans le cortex visuel vont diff�r�es du profil standard et c'est �a qui permettra d'�tablir le choix de l'utilisateur. Depuis, les SSVEP ont �t� utilis�es dans de nombreux autres syst�mes, comme la s�lection binaire \citep{allison2008towards}, pour �peler \citep{cecotti2010self}, le contr�le continu d'un curseur $1D$ ou $2D$ \citep{trejo2006brain}, les proth�ses \citep{muller-putz_control_2008} ou encore pour le jeux \citep{lalor_steady-state_2005}. \\
Ces SSVEP-BCI n�cessitent que l'individu ai, d'une part, un restant de contr�le occulo-moteur pour orienter son regard et d'autre part, un syst�me visuelle fonctionnelle. Enfin, puisque des images sont flash�es � des vitesses assez importantes, ces ICM ne correspondent pas aux personnes �pileptiques.

\figScaleDesciption{0.5}{PSD_SSVEP}{Spectre de puissance d'un signal EEG contenant des SSVEP \citep{lalor_steady-state_2005}}{Spectre de puissance d'un signal EEG contenant des SSVEP � 17hz (ligne pleine) et � 20hz (ligne pointill�es). Dans cette �tude, les auteurs utilisent les SSVEP pour une prise de d�cision binaire permettant un contr�le d'un jeux virtuel en $3D$ \citep{lalor_steady-state_2005}}

% -> P300
\paragraph{Onde P300}
Comme d�crit ci-dessus, les ERP sont une r�ponse exog�ne � un stimulus ext�rieur pouvant �tre de nature visuelle, auditive ou tactile. Apr�s moyennage � travers les essais, on peut constater l'apparition d'une s�rie de pic de tension � alternance positive (\textit{P}) et n�gative (\textit{N}) dont la localisation temporelle est \textit{fixe} (ou contenue dans un intervalle) et connue \citep{wolpaw_brain_2002, beverina_user_2003}. Ces ondes sont �tudi�es au dessus du cortex pari�tal. Par convention, leur nom d�rive de l'instant temporel d'apparition. Ainsi, on parlera des ondes N100 (ou N1), de la P200 (ou P2) et enfin de la P300. \\
Dans le cadre des ICM, on va surtout s'int�resser � cette derni�re, la P300 (bien que la N1 est en souvent �galement prise en compte). A l'instar des SSVEP, des objets sont al�atoirement mis en surbrillance sur un �cran et on demande au sujet de choisir un de ces �l�ment et de compter le nombre de fois que celui-ci est flash�. Enfin, lorsqu'une P300 est d�tect�e dans l'activit� neuronale, en remontant 300ms plus t�t, il est possible de retrouver l'�l�ment mis en surbrillance � cet instant. Une autre petite beaut� de la P300, c'est que son amplitude est d'autant plus forte que le sujet � r�ussit � d�nombrer le nombre d'apparition.

\figScaleDesciption{0.8}{ERP_and_P300}{ERP et onde P300 \citep{kubler2001brain}}{(A gauche) ERP g�n�r� par un stimuli auditif et apparition des diff�rentes ondes, (A droite) Onde P300. On constate une augmentation d'amplitude lorsque l'utilisateur se focalise sur la lettre P (\textit{target letter (P)} et \textit{target row/column}), compar� au profil standard (\textit{standards}) \citep{kubler2001brain}}

La premi�re ICM utilisant la P300 a �t� introduite par \cite{farwell_talking_1988, donchin_mental_2000}. Ce syst�me appel� le \textit{P300-speller}, permet � l'utilisateur d'�peler des mots. Le sujet est devant un �cran form� d'une matrice de lettres al�atoirement flash�es. Cette BCI est toujours d'actualit� et de nombreuses �tudes continuent de la d�velopper \citep{vaughan_wadsworth_2006, hoffmann2008efficient}. L'utilit� de la P300 pour contr�ler un fauteuil roulant a �galement �t� explor� \citep{vanacker_context-based_2007, pires_visual_2008}. Enfin, \cite{mugler_design_2010} d�crivent une ICM test�e sur des sujets sains et atteints de SLA, permettant de contr�ler un navigateur internet.

\vspace{1\baselineskip}
Les syst�mes pr�sent�s utilisent des stimulus visuels. Toutefois, d'autres ICM reposent sur des stimulus auditifs \citep{sellers2006brain, sellers2006p300, furdea2009auditory, schreuder2010new} ou tactiles \citep{muller2006steady, brouwer2010tactile}.

% ********************************************
%         Signaux spontan�s
% ********************************************
\subsubsection{Signaux spontan�s}
Les signaux spontan�s correspondent � un ensemble de signaux c�r�braux que l'utilisateur peut apprendre � moduler. Cet apprentissage, qui peut s'av�rer assez long, va permettre de contr�ler une \icm. \\
Parmi ces signaux, on rencontre les \textit{potentiels corticaux lents} ainsi que les \textit{rythmes sensorimoteurs}. Ces derniers sont largement plus pr�sents dans la litt�rature ICM. Avant tout, pour apprendre � moduler sont activit� neuronale, l'utilisateur devra utiliser une strat�gie mentale 

% -> Apprentissage autonome et imagerie motrice
\paragraph{Strat�gies mentales pour le contr�le d'une ICM}
\label{subsubsec_strategie_mentale}
Pour assimiler le contr�le des signaux spontan�s, il est n�cessaire que utilisateur passe par une phase d'apprentissage pouvant �tre soit autonome, soit guid�e via l'utilisation de l'imagerie motrice \citep{curran_learning_2003}.
\begin{enumerate}
	\item Apprentissage autonome (\textit{operant conditioning}, \textit{implicit learning} ou \textit{operant self-control}): le choix de la strat�gie mentale est laiss� � l'utilisateur. C'est � lui de trouver celle qui lui convient le mieux. Dans ce cas, le syst�me doit imp�rativement fournir un feedback au sujet afin que celui-ci comprenne comment la machine.
	\item Imagerie motrice: pour cette strat�gie mentale, on demandera donc � l'utilisateur d'imaginer des mouvements qui lui seront impos�s en amont (cf. \ref{sec_imagerie_motrice})
\end{enumerate}
Ces strat�gies mentales conditionnent le mode d'interaction de l'utilisateur avec la machine. Il est fr�quent que, dans les premiers stades de l'entra�nement, les sujets soumis � l'apprentissage autonome utilisent naturellement l'imagerie motrice \citep{wolpaw1991eeg, birbaumer_spelling_1999}. \\
L'\textit{operant conditioning} demande un long apprentissage (de plusieurs semaines � ann�es) mais les syst�mes l'utilisant rapportent de bonnes performances et une grande stabilit� \citep{fetz_operant_1969, wolpaw1991eeg, birbaumer_spelling_1999, wolpaw_control_2004-1, birbaumer_breaking_2006, vaughan_wadsworth_2006, wolpaw_brain-computer_2007}. Certaines t�ches motrices peuvent ne pas convenir aux personnes ayant un d�ficit moteur de longue date ou depuis la naissance tout comme les t�ches visuelles pour les personnes aveugles de naissance \citep{curran_learning_2003}. Donc, autre avantage de l'apprentissage autonome, il permet de prendre en compte la pr�f�rence ou le confort d'utilisation qui pourrait jouer un r�le dans les performances de l'ICM \citep{pfurtscheller_brain_2000}. 


\paragraph{Imagerie motrice}
\label{sec_imagerie_motrice}
L'imagerie motrice (IM) est d�finie par la repr�sentation d'une action qui n'est pas suivie de son ex�cution. Dans les sections pr�c�dentes, nous avons vu que l'IM permettait de contr�ler une \icm. Les substrats neuronaux mis en jeux lors de l'imagination d'un mouvement sont sensiblement les m�mes que lors de l'ex�cution de ce mouvement, ce qui permet donc aux personnes � mobilit� r�duite de solliciter les aires motrices sans la possibilit� d'accomplir le mouvement. \\
Cette section a pour objectif de raffiner l'imagerie motrice, \cad introduire les diff�rents types d'imagerie, diff�rentes applications et enfin, leur utilisation pour les ICM.

\begin{enumerate}
	\item \textbf{Types d'IM: } les diff�rents types d'imagerie reposent sur l'exploitation de modalit�s sensorielles, \cad sur les sens \citep{kosslyn1990imagery}. On distingue donc l'imagerie visuelle, tactile, auditive et olfactive. L'imagerie visuelle se d�compose en deux sous-divisions: interne (ou � la premi�re personne) et externe (ou � la troisi�me personne) \citep{ruby_effect_2001, jackson_neural_2006, lorey_embodied_2009}. Dans le premier cas, on est directement acteur de l'action, donc par exemple, on pourra imaginer effectuer un mouvement de bras de mani�re similaire � une r�elle ex�cution. En imagerie visuelle externe, on est spectateur d'un mouvement pouvant �tre effectu� par une autre personne, ou par soi-m�me. A ces diff�rents types d'imagerie, s'ajoute l'imagerie kinesth�sique bas�e sur des informations proprioceptives. Bien que � priori proches, imagerie visuelle et kinesth�sique sollicitent des substrats neuronaux diff�rents \citep{solodkin_fine_2004, guillot_brain_2009}. \\
	Si une personne imagine avoir une balle dans la main, c'est de l'imagerie visuelle interne. Si elle imagine que cette balle est tenue par une tierce personne, c'est de l'imagerie externe. Enfin, si la personne imagine les sensation que peut procurer les propri�t�s de cette balle (comme sa texture, sa mall�abilit�, son poids ou sa taille) c'est de l'imagerie kinesth�sique. De nombreuses �tudes ont compar�s les diff�rents types d'imagerie permettant de mettre en valeur les r�seaux associ�s \citep{jiang_neural_2015, seiler_biological_2015}.
	
	\item \textbf{Utilisation de l'IM: } dans le paragraphe pr�c�dent nous avons vu que l'IM peut �tre utilis�e comme strat�gie mentale pour le contr�le d'une ICM (cf. \ref{subsubsec_strategie_mentale}). L'IM peut �galement �tre utilis�e pour faciliter et am�liorer un apprentissage, notamment pour la pratique sportive de haut niveau \citep{driskell_does_1994, guillot_functional_2008, schuster_best_2011, di_rienzo_online_2016}, pour la r�habilitation motrice \citep{jackson_potential_2001, sharma_motor_2006, de_vries_recovery_2011, malouin_towards_2013, di_rienzo_impact_2014}.
\end{enumerate}
Dans le cadre des \icm, \cite{neuper_imagery_2005} ont �tudi� le d�codage de diff�rentes modalit�s d'imagerie en comparaison avec le repos. L'int�r�t de cette �tude et de renseigner sur le type d'IM a privil�gier pour le contr�le d'une ICM. Pour cela, les sujets effectuent diff�rentes t�ches avec une balle: soit ils ex�cutent des mouvements (\textbf{ME}) (pression continue), soit ils observent une main anim�e (\textbf{OOM}), soit ils imaginent les sensations procur�es par cette balle (\textbf{MIK}) (imagerie kinesth�sique) soit ils imaginent une main effectuant des mouvements sous forme de film (\textbf{MIV}) (imagerie visuelle). \cite{neuper_imagery_2005} montre alors que le d�codage avec la condition MIK (67\%) est nettement sup�rieur � celui atteint avec la condition MIV (58\%). De plus, le d�codage MIK se focalise essentiellement autour des aires sensorimotrices, tout comme pour l'ex�cution, alors qu'il n'y a pas de pattern r�ellement �mergent pour le MIV. En conclusion, les auteurs conseil plut�t l'utilisation de l'imagerie kinesth�sique � l'imagerie visuelle comme strat�gie mentale pour le contr�le d'une ICM.


\figScaleDesciption{0.6}{hanakawa_2008}{Comparaison des aires actives lors d'un mouvement imagin� ou ex�cut� \citep{hanakawa_motor_2008}}{Comparaison des aires actives lors d'un mouvement imagin� ou ex�cut� \citep{hanakawa_motor_2008}}


% -> Onde lente
\paragraph{Potentiel corticaux lents}
Les potentiels corticaux lents (ou \textit{\SCPS} (SCP)) sont des variations lentes du potentiel cortical qui ont lieu entre 500ms et 10s \citep{birbaumer1990slow, birbaumer1997slow, birbaumer_thought-translation_2003, birbaumer_slow_1999, kleber_direct_2005}. Un utilisateur peut apprendre � contr�ler l'amplitude de ces signaux, notamment gr�ce � un bio-feedback o� l'utilisateur voit son activit� se moduler en temps r�el sur un �cran. les SCP peuvent prendre des valeurs positives, g�n�ralement lors d'un mouvement ou tout autre fonction impliquant une activation corticale, ou n�gative lors d'une r�duction de l'activit� corticale \citep{rockstroh1989slow, birbaumer1997slow, wolpaw_brain_2002}. \\
Les ICM bas�es sur les SCP vont n�cessiter un seuil et, en fonction de ce seuil, le sujet module son activit� lui permettant un contr�le binaire \citep{birbaumer_spelling_1999}. Toutefois, l'apprentissage peut s'av�rer extr�mement long. Le dispositif de traduction de pens�es \citep{kubler_thought_1999} (\textit{Thought Translation Device} (TTD)) , est un appareil destin� � entra�n� des sujets puis � tester leur apprentissage pour �peler des mots \citep{perelmouter1999language, birbaumer_thought-translation_2003}. Le software est d�velopp�s en C++ et utilise des outils de $BCI2000$, plate-forme de d�veloppement du groupe central de Wadsworth \citep{wolpaw_wadsworth_2003, schalk_bci2000_2004}. Le TTD a �t� d�velopp� pour les sujets compl�tement paralys�s (LIS et CLIS). Tout d'abord, les sujets s'entra�nent de mani�re autonome � moduler leur SCP notemment gr�ce � un retour visuel et auditif et un renforcement positif (visage souriant et musique lorsque le contr�le est r�ussit). Apr�s cette phase d'apprentissage machine, les sujets sont ensuite test�s pour s�lectionner des lettres ou des mots \citep{birbaumer_thought_2000, birbaumer_thought-translation_2003}. \\
Les SCP ont �galement �t� test�es � des fins cliniques. \cite{rockstroh1993cortical, kotchoubey1998control} ont montr� qu'apr�s un an et demi d'apprentissage autonome � moduler positivement et n�gativement les SCP, des sujets atteints d'�pilepsies pharmacor�sistantes ont vu leur crise diminuer de 50\% en moyenne (certains sujets n'avaient plus aucune crise tandis que d'autres n'ont eut aucun changement).

\figScaleDesciption{0.6}{SCP}{Exemple d'apprentissage pour contr�ler les \SCPS \citep{kubler2001brain}}{Feedback visuel des SCP renvoy�es par le TTD - A droite, un curseur peut se d�placer entre les deux objectifs (en haut et en bas). Les n�gativit� corticales bougent le curseur vers le haut (ligne du haut en pointill�) � l'inverse, les positivit�s corticales permettent de faire descendre le curseur (ligne pleine du bas). Toute modulation de $7 \mu V$ est consid�r�e comme une r�ussite \citep{kubler2001brain}}

% -> Bereitschaftspotential
\paragraph{Le Bereitschaftspotential}
Le Bereitschaftspotential (BP), ou \textit{readiness potential}, est un potentiel moteur focalis� principalement dans la pr�-aire motrice suppl�mentaire (pr�-SMA) et SMA \citep{kornhuber1965hirnpotentialanderungen, ball1999role, brunia2000motor}. Le BP comprend deux composantes dont la premi�re, la composante pr�coce (ou \textit{early BP}), commence environ deux secondes avant le d�but du mouvement qui est ensuite suivie d'une pente n�gative tardive (ou \textit{late BP}), environ 400ms avant le d�but du mouvement \citep{shibasaki_what_2006}. Il a �t� montr� que le BP d�pend de param�tres de mouvements tels que l'�tat de pr�paration ou encore la r�p�tition et pr�cision de mouvement \citep{birbaumer1990slow}. Enfin, le BP est d'avantage pr�sent lors de mouvements auto-initi�s, compar�s � des mouvements imagin�s ou bas�s sur un \textit{cue} externe \citep{deiber1999mesial, jenkins2000self, jankelowitz_movement-related_2002}.

% -> Rythmes sensorymoteur
\paragraph{Rythmes sensorimoteurs (RSM)}
Les RSM (ou \textit{Sensorimotor rhythms} (SMR)) correspondent � l'amplitude de signaux, au dessus du cortex sensorimoteur, dans des bandes de fr�quences sp�cifiques. Les plus fr�quents sont les rythmes $\mu$ 8-13hz et $\beta$ 13-30hz. Un utilisateur peut apprendre � moduler l'amplitude de son activit� c�r�brale dans ces bandes pour contr�ler une \icm notamment par le biais de l'imagerie motrice (cf. \ref{sec_imagerie_motrice}). On appelle \textit{\ERS} lorsque l'amplitude augmente et \textit{\ERD} lorsqu'elle diminue \citep{pfurtscheller_event-related_1999, pfurtscheller_rehabilitation_2008}. \\
Ces rythmes ont tr�s largement �t� exploit�s pour contr�ler un BMI, que ce soit pour d�placer un curseur dans une, deux ou trois dimensions \citep{wolpaw_control_2004, mcfarland2008emulation, kayagil2009binary, mcfarland2010electroencephalographic, doud2011continuous}, pour �peler \citep{neuper2006motor, vaughan_wadsworth_2006}, pour contr�ler une proth�se \citep{pfurtscheller_brain_2000, muller2005eeg, mcfarland2008brain} ou un fauteuil roulant \citep{tanaka2005electroencephalogram, galan2008brain}

\figScaleDesciption{1}{kubler_mu_and_imagerie}{Utilisation des ERD et du $\mu$ pour contr�ler une ICM \citep{kubler2001brain}}{(A gauche) Utilisation de l'imagerie motrice pour contr�ler la direction d'un curseur et trac� de l'activit� neuronale pour les �lectrodes C3 (h�misph�re gauche) et C4 (h�misph�re droit) en utilisant l'activit� neuronale contenue dans la bande $\alpha$ - L'utilisateur imagine des mouvements des mains droite ou gauche pour d�placer le curseur. L'imagination d'un mouvement de main gauche entra�ne une ERD dans l'h�misph�re droit (C4) mais l'activit� est maintenue sur C3 (A droite) Modulation du $\mu$ au dessus de C3-C4 pour contr�ler un curseur vers le haut (augmentation du $\mu$, ligne pointill�e) ou vers le bas (baisse du $\mu$, ligne pleine) \citep{kubler2001brain}}

\paragraph{Autres marqueurs}
Les marqueurs pr�sent�s ci-dessus sont ceux que l'on retrouve le plus largement � travers la litt�rature. Toutefois, d'autres marqueurs de l'activit� neuronale sont �tudi�s essentiellement d'un point de vue neuro-scientifique, \cad pour am�liorer la compr�hension des ph�nom�nes physiologiques.

\subparagraph{D�composition phase-amplitude:} tout signal temporel r�el peut �tre d�compos� en un signal d'amplitude (ou enveloppe car elle va suivre les maxima du signal) et un signal de phase qui indique la situation instantan�e d'un cycle (pic, creux, passage � z�ro \dots). La transform�e d'Hilbert permet de d�composer ainsi n'importe quel signal (\ref{methodo_hilbert}).
\figScaleDesciption{1}{phamp_decomp}{D�composition d'un signal en phase et amplitude}{(En gris) Un signal original, (En rouge) L'amplitude du signal, (En bleu) la phase instantan�e}

\subparagraph{Phase:} si l'amplitude a tr�s largement �t� �tudi�e, que ce soit dans le cadre des ICM, du d�codage et de son implication physiologique, la phase reste � l'heure actuelle l'objet d'un nombre plus restreint d'�tudes. Toutefois, quelques �tudes ont montr� l'implication de la phase basse-fr�quence dans l'encodage neuronal de mouvements \citep{hammer_role_2013, hammer_predominance_2016}.

\subparagraph{Couplage inter-fr�quences locaux ou � distance:} ou \textit{Cross-Frequency Coupling} (CFC) regroupe un ensemble de marqueurs qui, contrairement � ceux pr�sent�s ci-dessus, �tudient une forme de corr�lation entre deux signaux pouvant �tre soit locaux (comme provenant d'une m�me �lectrode) ou � distance pour �valuer des synchronisations d'aires c�r�brales. Ces \textit{features} sont �tudi�s dans des bandes de fr�quences particuli�res et vont donc n�cessiter une filtrage en amont. Ensuite, on pourra extraire de ces signaux filtr�s les informations de phase et d'amplitude afin d'�tudier diff�rentes formes de couplage:
\begin{description}
	\item[Couplage � distance phase-phase:] le \textit{Phase-Locking Value} (PLV) \citep{lachaux_measuring_1999, lachaux_studying_2000} est un des outils permettant de d�terminer la synchronisation de phase. Le couplage inter-fr�quence phase-phase semble jouer un r�le dans la communication inter-structures \citep{fries_mechanism_2005, gregoriou_high-frequency_2009, siegel_phase-dependent_2009, elswijk_corticospinal_2010}
	\item[Couplage � distance amplitude-amplitude:] bien que le r�le physiologique du couplage inter-fr�quences amplitude-amplitude soit encore incertain, de tel couplages ont �t� d�cris dans la litt�rature \citep{friston_another_1997, shirvalkar_bidirectional_2010, siegel_phase-dependent_2009}
	\item[Couplage local ou � distance phase-amplitude:] ou \textit{\PACEN} (PAC), fait intervenir des signaux pris dans deux bandes de fr�quence et permet d'�valuer la fa�on dont ces signaux �voluent l'un avec l'autre. Plus pr�cis�ment, on consid�re un signal dans les basses fr�quences (BF), typiquement dans les bandes delta, th�ta ou alpha et on extrait la phase de ce signal. Ensuite, on consid�re l'amplitude d'un signal haute fr�quence (HF), souvent dans la bande gamma. Le PAC renseigne si la phase des BF et l'amplitude des HF �voluent de mani�re synchrones. A noter ici que il n'est pas question d'inf�rer une relation de cause/cons�quence entre la phase des BF et l'amplitude des HF. Autrement dit, le PAC ne permet pas de conclure que l'un vient moduler l'autre. Plusieurs outils m�thodologiques permettant une mesure du PAC ont �t� propos�s (cf. \ref{methodo_PAC}) \\
	Le r�le physiologique du PAC est encore discut� \citep{canolty_functional_2010, hyafil_neural_2015} tout comme son impl�mentation m�thodologique \citep{aru_untangling_2015}, mais il a �t� observ� dans des t�ches vari�es \citep{bruns2004task, voytek_shifts_2010, soto_investigation_2012}, dans la maladie de Parkinson \citep{hemptinne_exaggerated_2013}, dans la prise de risques \citep{lee_correlation_2013} ou derni�rement dans l'encodage m�moriel \citep{lega_slow-theta--gamma_2016}. Enfin, \cite{yanagisawa_regulation_2012} ont pu montrer l'existence d'un couplage alpha-gamma dans le cortex sensorimoteur durant une p�riode de repos. Ensuite, ce couplage diminue avec l'ex�cution de t�ches motrice (mouvements de saisie, de pinc�e et d'ouverture de main). Enfin, cette �tude a montr� que ce couplage alpha-gamma, dans cette r�gion sensorimotrice, ne permettait pas de d�coder ces diff�rents types de mouvements. A noter que g�n�ralement l'amplitude est prise dans la bande gamma mais \cite{cohen_oscillatory_2008} ont d�montr� l'existence d'un couplage entre la phase du delta et th�ta avec l'amplitude de l'alpha et du gamma dans la prise de d�cision.
\end{description}

%\figScale{hyafil_2015}{M�canismes du couplage phase-amplitude}


% ********************************************
%                ICM hybrides
% ********************************************
%\subsection{ICM hybrides}
%Pour finir, les syst�mes d'ICM pr�sent�s ci-dessus sont uniquement bas�s sur l'activit� neuronale. Plus r�cemment, il a �t� propos� d'int�grer d'autres signaux physiologiques (comme l'activit� musculaire) pour am�liorer l'acuit� des syst�mes. Ces interfaces utilisant � la fois l'activit� neuronale et autres signaux physiologiques sont appel�es ICM hybrides (voir \cite{muller-putz_towards_2015} pour une \textit{review} r�cente de ces syst�mes hybrides).
%
%\vspace{1\baselineskip}
%\todo[inline,caption={},color=blue!20]{
%	\textbf{Conclusion sur les signaux physiologiques} \\
%	D'un point de vue clinique \citep{chaudhary_brain-machine_2015}, pour les personnes pr�sentant des d�ficits moteurs (comme la SLA menant au \textit{locked-in state} (LIS) puis au \textit{completly locked-in state} (CLIS)), les SMR et P300-BCI donnent de meilleurs r�sultats en absolue mais les SCP-BCI semblent plus stables et moins d�pendantes des fonctions cognitives motrices et sensorimotrices \citep{birbaumer_breaking_2006}, ce qui repr�sente un avantage pour ces personnes atteintes en LIS ou CLIS. De plus, \cite{kubler_braincomputer_2008} ont pu montrer une corr�lation entre l'�volution de la SLA (et donc de la d�croissance des capacit�s motrices) et les performances des EEG-BCI. Pour compl�ter ce dernier point, \cite{de_massari_brain_2013} ont utilis� une SCP-BCI pour essay� d'�tablir une communication (deux choix \textit{yes} et \textit{no}) avec des sujets en LIS (et un en CLIS). Toutefois, les r�sultats ne sont pas probants, contrairement � \cite{gallegos-ayala_brain_2014} qui ont pr�f�r� utiliser la r�ponse h�modynamique pour un fNIRS-BCI et qui ont pu obtenir des d�codages plus prometteurs.
%}

%	- tDCS permet l'apprentissage moteur chez les sujets sains \citep{cuypers_is_2013, reis_time_2015} ou chez les stokes \citep{cuypers_is_2013, lefebvre2013dual}

% ********************************************
%         Invasives et Non-invasives
% ********************************************
\subsubsection{Enregistrement: ICM invasives et non-invasives}
\begin{enumerate}
	\item \textit{ICM invasives} ou \textit{directes}: bas�es sur un enregistrement invasif de l'activit� neuronal, la qualit� du signal �tant excellente les possibilit�s d'exploitations et d'am�liorations futures sont grandes. En revanche, cela va avec les risques que comporte la chirurgie.
	\item \textit{ICM non-invasives} ou \textit{indirectes}: de la m�me mani�re, ces ICM utilisent des enregistrements non-invasifs. Le terme \textit{indirecte} signifie que l'on enregistre pas directement des d�charges de neurones mais un ph�nom�ne li� � une population de neurones (consommation d'oxyg�ne pour \IRMf, champs �lectrique et magn�tique pour l'EEG et la MEG). Ces ICM sont les plus r�pandues et repr�sentent un enjeu majeure de part leur accessibilit�.
\end{enumerate}


%\textit{P300 speller} : blablabla\\
%
%\textit{Hex-o-spell} : ICM indirecte asynchrone d�velopp�e par \cite{blankertz_berlin_2006, blankertz_note_2007}, elle utilise l'imagerie motrice pour s�lectionner des caract�res. Le sujet imagine un mouvement de la main droite pour faire tourner la roue puis, stoppe la rotation et s�lectionne un groupe de lettres en imaginant un mouvement de pied droit. Ce dispositif a permis � un sujet d'atteindre la vitesse de 7.6 caract�res/min
%\figScaleX{0.9}{hex_o_spell_color}{Hex-o-spell \citep{blankertz_note_2007}}


% -----------------------------------------------------------------------------
% -----------------------------------------------------------------------------
%                      DATA MINING EN NEUROSCIENCES
% -----------------------------------------------------------------------------
% -----------------------------------------------------------------------------
\section{\textit{Data-mining} en neurosciences}

\subsection{Exploration des donn�es}
\subsection{Outils de validation}

% -----------------------------------------------------------------------------
% -----------------------------------------------------------------------------
%                      CONCLUSION DU CHAPITRE 1
% -----------------------------------------------------------------------------
% -----------------------------------------------------------------------------
%\vspace{3\baselineskip}
%\todo[inline,caption={},color=red!40]{
%	\textbf{Conclusion du chapitre 1} \\
%	Depuis leur naissance au d�but des ann�es 60, les \icm fascinent, intriguent et tout le monde s'entend sur leur potentiel et leur futures applications, notamment pour les personnes qui rencontrent des limitations motrices. Plusieurs grandes familles d'ICM sont alors apparues, chacune rencontrant des avantages et des limitations. On parlera ainsi d'ICM invasives et non-invasives si elles requi�rent, ou non, une intervention chirurgicale. Autres sous-cat�gories, les ICM synchrones et asynchrones, fonctionnant gr�ce � des substrats neuronaux diff�rents. La premi�re s'appuie sur des r�ponses c�r�brales engendr�es par des stimuli externes tandis que la seconde repose sur un contr�le volontaire de son activit� c�r�brale. \\
%	Les ICM sont un carrefour scientifique m�lant informatique, math�matique et neurosciences. Les progr�s de ces dispositifs r�sident principalement dans l'am�lioration et la miniaturisation des composants �lectroniques, dans l'acuit� des m�thodes mais �galement dans la compr�hension des processus neurophysiologiques mis en jeu.
%}
      % Interface Cerveau-Machine
% #############################################################################
%                        PRESENTATION DE LA THEMATIQUE
% #############################################################################
\chapter{ICM et neurophysiologie}

% -----------------------------------------------------------------------------
% -----------------------------------------------------------------------------
%                            BASES PHYSIOLOGIQUES
% -----------------------------------------------------------------------------
% -----------------------------------------------------------------------------
\section{Bases physiologiques}
\figScaleX{0.6}{Grainmann_2002_Brain}{Blabla \citep{graimann_braincomputer_2009}}

\section{Apprentissage machine: applications aux neurosciences}
okok

% -----------------------------------------------------------------------------
\section{Encodage et d�codage moteur: bases physiologiques}
Pomper les r�sultas de rodrigo + besserve + inclure sch�ma global du cerveau

% -----------------------------------------------------------------------------
% -----------------------------------------------------------------------------
%                        MARQUEURS DE L'AN
% -----------------------------------------------------------------------------
% -----------------------------------------------------------------------------
\section{Marqueurs de l'activit� neuronale}
\label{sec_marqueurs_AN}

\subsection{Activit� \textit{spike}}
\subsection{P300}
\subsection{Ondes lentes}
\subsection{Rythmes sensorimoteurs}
\subsection{SSVEP}
% -----------------------------------------------------------------------------
% -----------------------------------------------------------------------------
%                         IMAGERIE MOTRICE
% -----------------------------------------------------------------------------
% -----------------------------------------------------------------------------
\section{Imagerie motrice}

\subsection{Les diff�rents types d'imagerie}

\subsection{Imagerie, intention et ex�cution motrice}
\citep{hanakawa_motor_2008}
\figScale{hanakawa_2008}{Comparatif}

\subsection{Utilisation de l'imagerie motrice pour les ICM}

\subsection{Autre utilisation de l'imagerie motrice}

        % Neuro
% #############################################################################
%                           OBJECTIFS DE THESE
% #############################################################################
\chapter{Objectifs de la th�se}

Durant cette th�se, nous avons principalement utilis� les donn�es intracr�niennes issues d'une t�che motrice (cf. \ref{chap_data_xp}) afin d'�tudier les axes suivants :
\begin{description}
	\item[Utilisation des outils de \textit{machine learning} :] les m�thodes d'apprentissage machine ont �t� utilis�es comme un outil de validation pour explorer les diff�rences entre des �tats moteur.
	\item[Comparatif d'�tats moteur :] dans un premier temps, nous avons chercher � raffiner la connaissance des substrats neuronaux propres � un �tat de repos, de pr�paration ou d'ex�cution motrice. Puis, dans un second temps, nous avons �tudi� le d�codage directionnel durant ces phases de pr�paration ou d'ex�cution motrice.
	\item[Exploration et am�lioration des marqueurs de l'activit� c�r�brale :] pour comprendre ce qui caract�rise ces �tats, nous avons extrait de l'activit� neuronale une vari�t� relativement large de marqueurs spectraux (principalement la puissance, la phase et le \pacFR). De plus, nous avons fait varier de nombreux param�tres li�s � ces attributs (taille et emplacement des fen�tres temporelles et fr�quentielles, comparatif de diff�rentes m�thodologies propres � chaque \textit{feature}\dots)
	\item[Exploration des r�gions non-motrices :] les donn�es intracr�niennes nous ont �galement permis d'explorer si des r�gions non-motrices peuvent discriminer certains �tats ou si leur association avec des r�gions motrices peut constituer un gain de performances. 
	\item[Optimisation des param�tres de \textit{machine learning} :] de nombreux param�tres li�s � la classification ont �t� pris en compte afin d'�valuer leur influence sur le d�codage (choix et optimisation de l'algorithme de classification et de validation crois�e, strat�gie de \textit{multi-features}, �valuation statistique\dots)
	\item[Impl�mentation et mise � disposition d'un ensemble de m�thodes :] enfin, l'essentiel des m�thodes et des outils pr�sent�s et utilis�s durant cette th�se ont �t� cod� en Python puis mis � la disposition de tous sur un compte Github. Ces outils comprennent l'extraction de marqueurs, leur classification ainsi qu'un ensemble de fonction pour visualiser les r�sultats. 
\end{description}

\vspace{1\baselineskip}
Tous ces diff�rents param�tres ont permis d'avoir une compr�hension assez fine des impacts m�thodologiques mais ont aussi forc� chaque �tude � �tre hautement dimensionnelle. Cette th�se permettra peut-�tre aux lecteurs, et je l'esp�re, de commencer son exploration avec une nombre plus restreint de dimensions.    % Objectifs de th�se
\input{Intro/04_methodo}      % M�thodologie
%!TEX root = ../main.tex
% #############################################################################
%                                   LOGICIELS
% #############################################################################
\chapter{D�veloppements informatiques}

A l'heure actuelle, il existe une vaste diversit� de \textit{toolboxs}/logiciels permettant d'analyser puis de visualiser des donn�es neuro-scientifiques, que ce soit en \textit{Matlab}, en \textit{Python} ou tout autre langage. Parmi ces solutions, on pourrait citer \textit{Brainstorm}, \textit{FieldTrip}, \textit{MNE python}, \textit{Nipipe}, \textit{ELAN}... Toutes sont d�velopp�es depuis des ann�es, par des �quipes hautement qualifi�es et expertes et jouissent d'une excellente r�putation. Durant cette th�se, nous avons souhait� proposer des solutions informatiques pour les raisons suivantes :
\begin{description}
	\item[Ma�trise et compr�hension des outils :] bien qu'il soit tout � fait envisageable d'analyser ligne par ligne ces \textit{toolboxs}, le code peut parfois �tre assez dense et difficile � comprendre. Coder soi-m�me ses outils est une merveilleuse m�thode pour les d�mystifier et surtout, pour les utiliser correctement c'est-�-dire conna�tre les avantages et les limites de chacun.
	\item[Adaptation, am�lioration et ind�pendance :] lorsque l'on choisit une \textit{toolbox} on est limit� aux possibilit�s et � la qualit� d'impl�mentation de celle-ci. Il se peut que des besoins tr�s sp�cifiques ne soient donc pas couverts (pour des donn�es intracr�nienne comme dans le cadre de cette th�se, il existe peu de solution). Pour ces raisons, le d�veloppement d'outils personnels permet une meilleure couverture des besoins sp�cifiques et assure une ind�pendance fa�e aux limites d'une bo�te � outils.
	\item[Acquisition de comp�tences :] l'inconv�nient majeur de l'impl�mentation d'outils est le temps, un temps qui est forc�ment pris sur autre chose. En revanche, c'est une somme de comp�tences non-n�gligeables.
	\item[Identit�, communication et communaut� :] si les outils d�velopp�s sont de qualit� et forment un ensemble coh�rent, une communaut� d'utilisateurs peut se mettre en place ce qui peut contribuer � faire conna�tre une �quipe ou un laboratoire.
\end{description}

% -----------------------------------------------------------------------------
% -----------------------------------------------------------------------------
%                            PYTHON
% -----------------------------------------------------------------------------
% -----------------------------------------------------------------------------
\section{Choix du langage: Python}
Dans leur premi�re version, les solutions informatiques ont �t� d�velopp�es en \textit{Matlab}. \textit{Python} s'est impos� plus tard notamment gr�ce � son confort d'�criture et sa qualit� syntaxique, l'abondance de documentions, d'utilisateurs et de modules. De plus, c'est une langage portable, pouvant �tre install� sur toute machine et tout syst�me et surtout, \textit{Open Source} distribuable � souhait. A noter que le langage \textit{Julia} (\url{https://julialang.org/}) a �galement �t� test�. Ce langage se veut particuli�rement prometteur puisqu'il promet une syntaxe �l�gante � l'instar de \textit{Python} et des performances se rapprochant du \textit{C}, devan�ant ainsi \textit{Python} dans sa version non-optimis�e. Il a toutefois �t� �cart�, non � cause de ses performances mais parce que c'est un langage encore r�cent et comportant un nombre plus r�duit de modules que \textit{Python} et une communaut� plus petite.

% -----------------------------------------------------------------------------
% -----------------------------------------------------------------------------
%                Paquets d�velopp�s durant cette th�se
% -----------------------------------------------------------------------------
% -----------------------------------------------------------------------------
\section{Paquets d�velopp�s durant cette th�se}
Trois paquets ont �t� d�velopp�s pour proposer une solution coh�rente de l'extraction des donn�es � la visualisation des r�sultats :
\begin{description}
	\item[ipywksp :] workspace s'int�grant dans les notebooks \textit{Jupyter}.
	\item[brainpipe :] paquet permettant d'analyser des donn�es (pr�-traitements, extraction de features, classification et visualisation 2D)
	\item[visbrain :] ensemble de modules destin�s � des visualisations complexes et de hautes performances.
\end{description}

% _______________ IPYWKSP _______________
\subsection{ipywksp}
Ce paquet est destin� aux utilisateurs venant de \textit{Matlab} et souhaitant retrouver un workspace semblable. \textbf{ipywksp} m�le plusieurs langages (\textit{Python}, \textit{HTML} et \textit{Javascript}) et permet de visualiser le type, le contenue et la taille des variables, de les sauvegarder/charger et de les visualiser. Enfin, ce paquet

\subsubsection{Installation et utilisation}
Dans un terminal, lancer : \\
\textbf{git clone https://github.com/EtienneCmb/ipywksp.git} \\
\textbf{python setup.py install} \\

Pour utiliser le workspace, lancer dans un notebook \textit{Jupyter} :
\begin{python}
# Chargement du module :
from ipywksp import workspace

# Ouverture du workspace avec un th�me noir et s'affichant automatiquement au survol de la souris :
workspace(theme="dark", autoHide=True)
\end{python}

\figScaleDesciption{.8}{theme_dark}{\textit{ipywksp} : Exemple de workspace pour \textit{Jupyter}}{\textit{ipywksp} : Exemple de workspace pour \textit{Jupyter}}


% _______________ BRAINPIPE _______________
\subsection{Brainpipe}
Ce paquet est destin� � l'analyse de donn�es de tout type m�me si il est particuli�rement adapt� aux donn�es intracr�nienne. Il permet d'extraire un ensemble d'attributs, de les classifier, d'effectuer des analyses statistiques et de visualiser les r�sultats sur des graphes simples. Tout les r�sultats obtenus durant cette th�se ont �t� obtenu avec ce module et donc, toutes les m�thodes y sont impl�ment�es.

% -------------------> Fonctionnalit�s
\subsubsection{Fonctionnalit�s}

% Study
\paragraph{Study : }
Ce sous-module permet de g�rer plusieurs �tudes et plusieurs jeux de donn�es, de g�rer un tr�s grands nombres de fichiers, de cr�er une arborescence de dossiers propre, une meilleure gestion des chemins d'acc�s ce qui est un atout majeur pour des collaborations.
\begin{python}
	# Importation des librairies :
	import numpy as np
	from brainpipe.system import study

	# Cr�ation de deux �tudes :
	st = study('MEG')
	st.add('~/Python/')
	st = study('EEG')
	st.add('~/Python/')

	# Cr�ation et sauvegarde d'une variable dans le sous-dossier database :
	x = np.random.rand(1000)
	st.save('database', 'test.npy', x)
\end{python}

% Pr�-traitements :
\paragraph{Pr�-traitements :}
Ensemble d'outils pour pr�-traiter les donn�es, \cad des outils de filtrage performants, la bipolarisation et la recherche des structures anatomiques associ�es � des coordonn�es MNI/Talairach.
\begin{python}
	# Chargement des librairies :
	from brainpipe.preprocessing import bipolarization, xyz2phy
	from brainpipe.system import study

	# Chargement de l'�tude en cours :
	st = study('CenterOut')

	# Chargement des donn�es � pr�-traiter :
	data, channels, xyz = st.load('database', 'centerout_data.npz')
	# O� :
	# - data : les donn�es intracr�niennes
	# - channels : nom des channels monopolaires
	# - xyz : les coordonn�es MNI des channels

	# Bipolarisation et recherche des structures anatomiques :
	data_bip, channels_bip, xyz_bip = bipolarization(data, channels, xyz=xyz)
	phy = xyz2phy().get(xyz_bip, channels_bip)
\end{python}

% Features :
\paragraph{Attributs :}
Brainpipe int�gre une collection relativement importante de features calculables :
\begin{itemize}
	\item Signal filtr�
	\item Amplitude
	\item Puissance (hilbert, wavelet ou PSD)
	\item Phase Amplitude Coupling (nombreuses m�thodologies / possibilit� de g�n�rer des signaux synth�tiques coupl�s)
	\item Phase-Locking Factor (PLF)
	\item Cartes temps-frequences
	\item Phase-Locked Power (puissance align�es sur un cue)
	\item Event-Related Phase Amplitude Coupling (ERPAC)
	\item Phase pr�f�rentielle
	\item Phase Locking Value (PLV, soit � travers le temps, soit � travers les trials)
	\item Entropie spectrale
\end{itemize}
A noter que certains attributs int�grent un fen�trage et le calcul dans des bandes de fr�quences. De plus, tous ont �t� impl�ment�s de fa�on matricielle et peuvent �tre calcul�s en parall�le pour un temps de calcul le plus r�duit possible. Enfin, tous comprennent de nombreuses configuration possibles, int�grent le calcul de significativit� et les outils de visualisation.

\begin{python}
	# Chargement des librairies :
	from brainpipe.feature import *
	import numpy as np
	import matplotlib.pyplot as plt

	sf = 1024  # Fr�quence d'�chantillonage

	# On g�n�re des donn�es contenant un couplage entre 10 et 100hz :
	data = cfcRndSignals(sf=sf, fPha=10, fAmp=100, ndatasets=10, noise=2, chi=.5)[0].T
	npts = data.shape[0]

	# On g�n�re des vecteurs phase et amplitude :
	pVec, aVec, pha, amp = cfcVec()

	# Calcul du PAC :
	pacO = pac(sf, npts, pha_f=pha, amp_f=amp, Id='133')
	xPac = pacO.get(data, data, matricial=True)[0]
	xPac = np.squeeze(xPac)  # Suppression des dimensions inutiles

	# Plot du PAC avec les fonctions int�gr�es :
	fig = plt.figure()
	pacO.plot2D(fig, xPac.mean(-1), vmin=0, vmax=16, xvec=pVec, yvec=aVec, xlabel='Fr�quence de phase (hz)',
	            ylabel='Fr�quence amplitude (hz)', title='Example de PAC', cmap='viridis', cblabel='Couplage PAC')
	plt.show()
\end{python}

\figScaleDesciption{.4}{exemple_PAC}{Exemple de calcul PAC avec brainpipe}{Exemple de calcul PAC avec brainpipe}

% Classification :
\paragraph{Classification :}
L'essentiel de la classification est assur� par \textit{scikit-learn} \citep{scikit-learn}. Toutefois, brainpipe offre certaines fonctionnalit�s d'ordre pratiques qui, de mani�re non-exhaustive, peuvent �tre r�sum�es � :
\begin{itemize}
	\item Possibilit� de d�finir une cross-validation et diff�rents classifieurs et surtout, de pouvoir comparer leur performances de mani�re plus compactes.
	\item Adaptation de la classification aux donn�es neuro-scientifiques, notamment en offrant un calcul en parall�le plus efficace car mieux adapt� � nos petites donn�es (en comparaison aux �normes banques de donn�es d'images). De plus, de nombreuses �tudes utilisent des classification particuli�re telles que le \textit{Leave-p-Subject-Out}, pr�sente dans brainpipe tout comme les cross-validation de type \textit{10-times...}.
	\item G�n�ralisation temporelle \citep{king_characterizing_2014}
	\item Calcul de la significativit� des d�codages plus synth�tique (loi binomiale ou permutations)
\end{itemize}
Pour les neuro-scientifiques, brainpipe est un bon point d'entr� au monde de la classification puisqu'il permet de rapidement classifier nos donn�es en un minimum de lignes et de mani�re lisible. Toutefois, pour une utilisation plus fine, un programme en \textit{scikit-learn} pure reste moins limitatif.

% Statistiques :
\paragraph{Statistiques :}
En plus des statistiques calcul�s pour chaque attribut et pour la classification, brainpipe met � disposition un ensemble d'outils d'analyses statistiques. Calcul et gestion de permutations, correction multiple (Bonferroni, False Discovery Rate, Maximum statistic) ainsi que des outils pour les donn�es circulaires (comme des donn�es de phase).

% Visualisation :
\paragraph{Visualisation :}
Enfin, des fonctions pour visualiser des donn�es ont �galement �t� ajout�es. Celles-ci permettent de cr�er des graphes 2D esth�tiques et hautement configurables (plot de lignes avec d�viation, ajout de valeur \textit{p}, de lignes verticales/horizontales, plot d'image, de contours...).

\begin{python}
	# Chargement des librairies :
	from brainpipe.visual import BorderPlot
	import numpy as np

	# D�finition d'un vecteur temps et de 10 sinuso�dales:
	sf = 1024
	_, t = np.mgrid[0:10, 0:1000] / sf
	x = np.sin(2*np.pi*5*t) + .5 * np.random.rand(*t.shape)

	# Plot du signal et de sa d�viation :
	BorderPlot(t[0, :], x.T, kind='std', xlabel='Temps', ylabel='Amplitude',
	           title='Exemple de visualisation')
	plt.show()
\end{python}
\figScaleDesciption{.7}{exemple_visu}{Exemple de plot d'un signal et de sa d�viation}{Exemple de plot d'un signal et de sa d�viation}

% -------------------> T�l�chargement, installation et documentation
\subsubsection{Installation et documentation}
Pour installer brainpipe, lancer dans un terminal : \\
\textbf{git clone https://github.com/EtienneCmb/brainpipe.git} \\
\textbf{python setup.py install} \\
Pour finir, une documentation compl�te est disponible en ligne \url{https://etiennecmb.github.io/brainpipe}

% _______________ VISBRAIN _______________
\subsection{Visbrain}
Visbrain est un paquet destin� � la visualisation de donn�es neuro-scientifiques. Sa particularit� r�side dans le fait qu'il se base sur Vispy \cite{campagnola_vispy_2013} qui lui m�me utilise \textit{OpenGL}. Les calculs sont envoy�s sur la carte graphique ce qui, en cons�quence, offre de tr�s hautes performances en terme de fluidit� et de temps de calcul. De plus, les interactions entre l'utilisateur et les diff�rents modules se font via des interfaces graphiques (\textit{Graphical User Interface, GUI}) construites � partir de \textit{PyQt}.

% -------------------> Pr�sentation
\subsubsection{Pr�sentation des modules}

% Brain :
\paragraph{Brain :}
\textit{Brain} est une GUI avec un cerveau MNI dans lequel il est possible d'ins�rer des objets :
\begin{itemize}
	\item \textbf{Sources} : dispositions de sources mat�rialis�es par des sph�res de couleur
	\item \textbf{Connectivit�} : possibilit� d'afficher des liens de connectivit� entre ces sources
	\item \textbf{Structures} : ajout de structures 3D internes soit bas�es sur les aires de Brodmann soit sur l'AAL (\textit{Automated Anatomical Labeling})
	\item \textbf{Autres} : tout autre objet � trois dimensions peut-�tre rajout� par l'utilisateur.
\end{itemize}
Il n'y a aucune limite sur le nombre d'objets pouvant �tre ajout�s et ils peuvent tous �tre contr�l�s ind�pendamment (couleur, transparence, taille, forme...). De plus, certains de ces objets peuvent interagir ensemble. Par exemple, l'activit� des sources peuvent �tre projet�es sur le surface du cerveau ou sur des structures internes. \\
Dernier point important, toutes les interactions possibles depuis l'interface graphique (et par raccourcis) sont �galement possibles en ligne de commande (Voir les \href{http://etiennecmb.github.io/visbrain/brain.html\#user-functions}{User functions} dans la documentation). Cette fonctionnalit� est particuli�rement utile pour produire un grand nombre de figures puisque tout peut �tre automatis�.

\figScaleDesciption{1}{visbrain_brain}{Exemples des principales fonctionnalit�s de \textbf{Brain}}{Exemples des principales fonctionnalit�s de \textbf{Brain}, (A) Sites intracr�niens et connectivit�, (B) Sources MEG et projection de leur puissance b�ta sur le thalamus, (C) Nombre de sources contribuant � chaque point de l'h�misph�re droit, (D) Projection de l'activit� corticale de plusieurs sources intracr�niennes, (E) Exemple de sc�ne complexe m�lant diff�rents objets poss�dant chacun leur configuration}

% Sleep :
\paragraph{Sleep :}
\textit{Sleep} est un module particuli�rement performant pour visualiser, analyser et �diter des donn�es de sommeil. Il a �t� d�velopp� en collaboration avec \href{https://raphaelvallat.github.io/}{Raphael Vallat}. \\
Parmi les fonctionnalit�s principales, on peut citer :
\begin{itemize}
	\item Chargement de fichiers *.edf, *.eeg (Brainvision, ELAN) et *.trc
	\item Visualisation temporelle des donn�es polysomnographiques (avec possibilit� d'afficher/cacher les channels, contr�le des unit�s temporelles, de la taille de fen�tre, de l'amplitude...), en spectrogramme ou sous forme topographique
	\item Chargement/visualisation/�dition/sauvegarde de l'hypnogramme
	\item Impl�mentation de d�tection de spindles, K-complexes, slow waves, rapid eye movements (REM), contraction musculaire ou encore de pic. Chaque d�tection peut �tre lanc�e sur les channels souhait�s et des rep�res visuels sont ajout�s � l'hypnogramme
	\item Outils de traitement de signal (suppression des composantes lin�aire et de moyenne, bipolarisation/re-r�f�rencement, outil de filtrage pour afficher le signal filtr�, l'amplitude, la puissance ou encore la phase)
	\item Nombreux raccourcis pour une interaction la plus efficace possible. 
	\item Une fonction d'\textit{Auto scoring} est actuellement en cours de d�veloppement.
\end{itemize}

\figScaleDesciption{1}{Sleep_013}{Exemples des principales fonctionnalit�s de \textbf{Sleep}}{Exemples des principales fonctionnalit�s de \textbf{Sleep}, (A) Possibilit� de choisir les channels � afficher et contr�le ind�pendant des amplitudes, (B) Repr�sentation temporelle des donn�es polysomnographique, (C) Sprectogramme d'un channel, (D) Visualisation de l'hypnogramme et possibilit� de l'�diter, (E) D�tections de spindle et de REM sur deux channels, (F) Exemple de repr�sentation topographique (topoplot)}

% Ndviz :
\paragraph{Ndviz}
Le module \textit{Ndviz} a �t� con�u pour fouiller et explorer des donn�es multi-dimensionnelles. Un des soucis majeurs des �tudiants qui ne sont pas familiers avec la programmation est de se faire une image de ce que signifie une matrice et surtout, arriver � g�rer les dimensions. Par exemple, des donn�es organis�es en $(n_channels, n_points, n_essais)$ offrent un certain nombre de visualisation possible � travers les dimensions : essais par essais par channel, la moyenne des essais par channel voir la moyenne � travers certains channels et essais... De plus, pour des donn�es que l'on ne conna�t pas, il peut �tre difficile de rechercher des artefacts, des activit�s �pileptiques... \textit{Ndviz} essaye de r�pondre � ses diff�rentes probl�matiques en offrant diff�rentes fonctionnalit�s :
\begin{itemize}
	\item Dans tout \textit{Ndviz} il est possible de s�lectionner les dimensions � inspecter ce qui permet de se familiariser avec les matrices.
	\item Possibilit� de visualiser plusieurs milliers de signaux dispos�s en grille en m�me temps. Cette fonctionnalit� est issue d'un exemple de \textit{Vipy} originalement cod� par \href{https://github.com/rossant}{Cyrille Rossant}. Par exemple, pour des donn�es organis�es en $(nchannels, npoints, nessais)$, il serait possible d'afficher une grille de $nchannels$ lignes et $nessais$ colonnes et o� sur chaque point de cette grille serait dispos� un signal de $npoints$ temporels. Cette fonctionnalit� permet donc de visualiser des donn�es comportant au maximum trois dimensions. 
	\item De plus, en s�lectionnant trois dimensions on peut aussi visualiser les donn�es sous forme d'image (avec la couleur en guise de troisi�me dimension) ce qui pourrait par exemple �tre utile pour inspecter un grand nombre de carte temps-fr�quence. Enfin, en s�lectionnant deux dimensions, les donn�es peuvent �tre repr�sent�es sous forme lin�aire, en nuage de points, en histogramme ou spectrogramme.
\end{itemize}

\figScaleDesciption{1}{ndviz}{Exemples des principales fonctionnalit�s de \textbf{Ndviz}}{Exemples des principales fonctionnalit�s de \textbf{Ndviz}, (A) Visualisation de 40000 signaux dispos�s dans une grille de (200, 200). Chaque signal fait plusieurs milliers de points et il est possible de zoomer sur chaque signal, (B) Repr�sentation lin�aire d'un signal, (C) Exemple de repr�sentation sous forme d'image, (D-E) Calcul du spectrogramme et d'un histogramme d'un signal, (F) Repr�sentation en nuage de points. Cette derni�re repr�sentations pourrait �tre utilis�e pour inspecter des features}

% Figure :
\paragraph{Figure}
Ce dernier module est le plus simple et certainement le plus utile. Il permet de faire des mises en page complexes de figures qui peuvent ensuite �tre export�es en haute d�finition et pr�te � �tre int�gr�e dans un papier. Il peut charger des images, les couper, les disposer en grille, ajouter des colorbars (soit pour chaque figure soit des colorbar communes � plus images), contr�ler la couleur de l'arri�re plan, ajouter des titres � tous les axes... 

\figScaleDesciption{1}{visbrain_figure}{Exemple de mise en page avec le module \textbf{Figure}}{Exemple de mise en page avec le module \textbf{Figure}}

% -------------------> T�l�chargement, installation et documentation
\subsubsection{Installation et documentation}
La proc�dure d'installation est plus complexe car elle poss�de plus de d�pendances. Elle a donc �t� d�crite plus largement dans la documentation \url{http://etiennecmb.github.io/visbrain/}. A noter que cette documentation d�crit et illustre les fonctionnalit�s de chaque module et des exemples complets sont �galement mis � disposition \url{https://github.com/EtienneCmb/visbrain/tree/master/examples}
  % Logiciels
% #############################################################################
%                             DONNEES EXPERIMENTALES
% #############################################################################
\chapter{Donn�es exp�rimentales}
\label{chap_data_xp}
Durant cette th�se, l'exploration s'est faite chez l'Homme par le biais, principalement, de donn�es de type \seeg (SEEG). Ces donn�es rares de tr�s grande qualit� (cf. \ref{seeg_av-in}) ont �t� acquise avant le d�but de la th�se, ce qui a permis de rentrer dans le vif du sujet tr�s rapidement, apr�s une p�riode d'acclimatation aux diff�rents traitements, propres � ce type d'enregistrement. D'autres types tel que l'EEG, la MEG ou les micro-�lectrodes ont �galement �t� approch�s mais de mani�re ponctuelle, comme ce f�t le cas dans l'�tude $1$ (cf. \ref{seuil_chance}) ou dans diff�rentes collaborations. Toutefois, �tant donn� que le temps consacr� � ces donn�es ne repr�sente qu'une faible portion du travail total, nous allons ici nous concentrer uniquement sur l'intra.

\vspace{1\baselineskip}
Pour commencer, nous verrons ce que l'analyse de la \seeg a de particulier (richesse des donn�es, qualit�, les traitements associ�s, les avantages et les limitations). Enfin, nous verrons concr�tement les enregistrements qui ont �t� utilis�s dans le cadre de cette th�se.

% -----------------------------------------------------------------------------
% -----------------------------------------------------------------------------
%                     DONNEES INTRA
% -----------------------------------------------------------------------------
% -----------------------------------------------------------------------------
\section{Donn�es intracr�niennes}
\label{sec_intra_data}

% ********************************************
%         ACQUISITION
% ********************************************
\subsection{Acquisition}
La premi�re question que l'on est en droit de se poser, c'est comment est-il possible de travailler, chez l'Homme, avec des enregistrements qui n�cessitent une implantation invasive, \cad dans le cortex? Certaines personnes pr�sentent des formes agressives d'�pilepsies, pouvant s'av�rer pharmacor�sistantes. En fonction de la localisation du foyer �pileptog�ne, les m�faits engendr�s par les d�charges �pileptiques peuvent �tre vari�s. Dans ce cas, il est n�cessaire de localiser ce foyer avec, de pr�f�rence, des techniques non-invasives telles que l'EEG ou la MEG. Mais si ces derni�res ne permettent pas une localisation pr�cise le patient sera implant� avec des macro-�lectrodes comme la SEEG pour tenter de localiser puis d'enlever ce foyer par intervention chirurgicale. Cette implantation a un second objectif, d�terminer quel est le r�le fonctionnel de la structure l�s�e (r�le moteur, langage, vision...). C'est dans ce contexte que les chercheurs proposent au patient de participer � une �tude scientifique.

% ********************************************
%       AVANTAGES ET LIMITATIONS
% ********************************************
\subsection{Avantages et limitations}
\label{seeg_av-in}
Le paragraphe pr�c�dent met en exergue la raret� de ces donn�es. De plus, ce type d'acquisition enregistre l'activit� c�r�brale d'une population relativement restreinte de neurones. En cons�quence, on peut esp�rer que ce petit groupe s'active dans des processus pr�cis et ainsi, �tudier des ph�nom�nes fins. Enfin, le rapport signal sur bruit (RSB) de la SEEG est excellent, ce qui doit permettre l'�tude de processus, m�me en essais unique l� o� d'autres types d'enregistrements auront besoin d'une large banque d'essais avant de pouvoir constater l'�mergence d'un ph�nom�ne. \\
La SEEG pr�sente toutefois quelques limitations que l'on peut nuancer. Le probl�me majeur est certainement la g�n�ralisation d'un ph�nom�ne ou la reproductibilit� � travers les sujets. La pathologie est propre � chaque patient, donc son implantation aussi. Ce qui signifie qu'il n'y a aucune chance que plusieurs sujets pr�sentent rigoureusement la m�me implantation. Pour contourner cette limitation, ou pourra utiliser:
\begin{itemize}
	\item Des r�gions d'int�r�t (ROI): on va regrouper les �lectrodes des diff�rents sujets par "proximit�" en faisant l'hypoth�se que celles-ci s'activent de fa�on similaire face � un processus. Ces ROI pourront �tre par exemple les gyrus ou les aires de Brodmann. Bien s�r, ce que l'on gagne on g�n�ralisation, on le perd en pr�cision.
	\item Projection corticale: autour de chaque �lectrode, on d�finit une sph�re d'int�r�t (g�n�ralement 10 millim�tres de rayon) puis on prend l'intersection de ces sph�res avec la surface du cerveau. Cette technique permet une visualisation des activit�s proches de la surface � travers les sujets mais on perd de la lisibilit� sur ce qui se passe en profondeur.
\end{itemize}
Une utilisation combin�e des ROI et de la projection corticale permet de palier, au moins partiellement, au probl�me de reproductibilit� inter-sujets. \\
Autre limitation, on ne dispose que d'une couverture partielle puisque le neuro-chirurgien implante une quantit� limit�e d'�lectrodes. Ce dernier point est trait� en augmentant le nombre de sujets. Enfin, la derni�re limitation que l'on soul�vera ici, concerne le fait de travailler sur un cerveau "malade" emp�chant donc une g�n�ralisation � des sujets sains. On limite ce probl�me par un ensemble de pr�traitements \citep{jerbi_task-related_2009} d�cris dans le prochain paragraphe. \\
Un dernier point que l'on peut argumenter � la fois comme avantage ou limitation, c'est de ne pas pouvoir contr�ler l'implantation pour �tudier un ph�nom�ne pr�cis. Par exemple, si l'on analyse l'encodage moteur, on s'attendrait � concentrer les efforts sur le cortex moteur primaire ou pr�-moteur. Or l'implantation SEEG peut tr�s bien contenir du frontal, du pari�tal, du temporal. L� o� finalement on peut consid�rer �a comme un avantage, c'est que l'on a acc�s � un ensemble de structures jug�es non-primordiales mais dont l'ajout pourrait permettre la compr�hension d'un processus de mani�re plus globale.

\figScaleDesciption{0.6}{seeg_comparison}{Comparatif de r�solution spatiale et temporelle pour diff�rentes techniques d'imagerie \citep{lachaux_intracranial_2003}}{Comparatif de r�solution spatiale et temporelle pour diff�rentes techniques d'imagerie \citep{lachaux_intracranial_2003}. La SEEG offre � la fois une r�solution spatiale �quivalente � la PET ou fMRI et une r�solution temporelle proche de celle de la MEG et de l'EEG ce qui en fait une technique de choix, sans compromis de r�solution.}

% ********************************************
%         COMPOSANTES D'UNE ICM
% ********************************************
\subsection{Inspection visuelle}
BrainTV \citep{lachaux_braintv_2007,jerbi_chapter_2009}

% ********************************************
%         PRE-TRAITEMENTS
% ********************************************
\subsection{Pr�traitements}
\label{SEEG_preprocessing}
Le premier pr�traitement appliqu� a �t� une r�jection des sites bruit�s ou pr�sentant une activit� pathologique, \cad des d�charges �pileptiques. C'est par cette inspection manuelle que l'on augmente le potentiel de g�n�ralisation aux sujets sains. \\
Autre pr�traitement, les donn�es peuvent �tre bipolaris�es comme c'est le cas dans de nombreuses �tudes \citep{bastin_direct_2016, ossandon_transient_2011, jerbi_task-related_2009}. La bipolarisation part du principe que deux sites proches enregistrent des activit�s neuronales diff�rentes mais que toute source de bruit, ou influence de sources lointaines, se retrouvera sur ces deux sites. La technique de bipolarisation consiste donc � soustraire les activit�s neuronales de sites proches ce qui a pour effet de supprimer la partie commune, le bruit. Par exemple prenons une �lectrode contenant les sites $k9$, $k10$, $k11$ et $k12$. Apr�s bipolarisation, on consid�rera les sites mat�rialis�s par $k10-k9$, $k11-k10$ et $k12-k11$. Les b�n�fices de la bipolarisation peuvent �tre r�sum�s par:
\begin{enumerate}
	\item Limitation des influences des sources lointaines et de la tension secteur (50hz)
	\item Augmentation de la sp�cificit� qui, pour un site bipolaris�, est estim�e � 3mm \citep{kahane_bancaud_2006, lachaux_intracranial_2003, jerbi_task-related_2009}
\end{enumerate} 

% -----------------------------------------------------------------------------
% -----------------------------------------------------------------------------
%                            DONNEES CENTER OUT
% -----------------------------------------------------------------------------
% -----------------------------------------------------------------------------
\section{Donn�es d'�tude}
Trois jeux de donn�es intracr�niaux ont �t� explor�:
\begin{enumerate}
	\item \co: �tude de l'encodage et du d�codage des actions et des intentions motrices lors de mouvements de main
	\item \textit{Occulo}: �tude des intentions et d�cision de mouvements oculaires
	\item \textit{Emotions}: �tude de l'encodage des �motions
\end{enumerate}

% ********************************************
%         DONNEES CENTER-OUT
% ********************************************
\subsection{Donn�es \co}
\label{subsec_co_data}
C'est le jeu de donn�es qui a �t� le plus largement exploit�. En effet, celui-ci a servis � �tudier l'encodage (cf. \ref{Etude2_encodage}) et le d�codage (cf. \ref{Etude3_decodage}) des actions motrices chez l'homme. 

% -> Implantation
\subsubsection{Descriptif des donn�es}
Six sujets (six femmes), implant�s au d�partement de l'�pilepsie de l'h�pital de Grenoble ont donn� leur consentement �crit pour passer l'exp�rience, sous la supervision du personnel m�dical. Le tableau \ref{subject_table} r�sume les d�tails clinique des diff�rents sujets. \\

\begin{figure}[H]
	\begin{center}
		\begin{tabular}{|c|c|c|c|c|}
		  \hline
		   & Dominance & Age & Genre & Zone �pileptique \\
		   \hline
		   P1 & D & 19 & F & Frontal (RH) \\
		   \hline
		   P2 & D & 23 & F & Frontal (LH) \\
		   \hline
		   P3 & D & 18 & F & Frontal (RH) \\
		   \hline
		   P4 & D & 18 & F & Frontal (RH) \\
		   \hline
		   P5 & D & 31 & F & Insula (RH) \\
		   \hline
		   P6 & D & 24 & F & Frontal (LH) \\
		   \hline
		  \multicolumn{5}{|l|}{Moyenne: $22.17 \pm 4.6$}\\
		  \hline
		\end{tabular}
	\end{center}
	\caption{D�tails cliniques des sujets ayant particip� � la t�che \co}
\end{figure}
\label{subject_table}

% -> Mat�riel d'acquisition
\subsubsection{Mat�riel d'acquisition}
De 12 � 15 multi-�lectrodes ont �t� implant�es dans diff�rentes structures. Chaque multi-�lectrode poss�de entre 10 et 15 sites mesurant 0.8mm et s�par�s de 1.5mm. La localisation anatomique des �lectrodes s'est faite en utilisant le sch�ma d'implantation (exemple en annexe \ref{schema_implantation}) et l'atlas proportionnel de Talairach et Tournoux \citep{talairach_referentially_1993}. La visualisation de la pr�-implantation s'est faite par $IRM-3D$ et un $CT-scan$ a �t� utilis� pour la post-implantation. Enfin, un $IRM$ a �galement servis pour visualiser les �lectrodes implant�es dans la mati�re blanche. Les coordonn�es Talairach ont �t� d�duites du $CT-scan$ puis ont �t� transform�es en MNI afin de pouvoir les superposer dans un cerveau standard (cf. figure ci-dessous).\\
Le syst�me Micromed a �t� utilis� pour visionner l'acquisition de l'activit� neuronale. Une �lectrode prise dans la mati�re blanche a �t� prise comme r�f�rence et un filtrage \textit{passband} entre $[0.1, 200hz]$ a �galement �t� effectu� \textit{online}. La fr�quence d'�chantillonnage est de $1024hz$.

\figScaleX{1}{data_Cover}{Implantation intracr�niale et couverture corticale de six sujets �pileptiques ayant pass�s la t�che \co}


% -> Descriptif de la t�che
\subsubsection{Descriptif de la t�che}
La t�che est compos�e de trois phases:
\begin{enumerate}
	\item Phase de repos: on demande au sujet de rester immobile pendant une dur�e de une seconde
	\item Phase de pr�paration motrice ($\textbf{CUE 1}$): une direction est impos�e � l'�cran (haut/bas/gauche/droite). On demande au sujet de se pr�parer pendant 1.5s � bouger la souris dans la direction impos�e.
	\item Phase d'ex�cution motrice ($\textbf{CUE 2}$): le sujet ex�cute le mouvement en bougeant la souris du centre � l'extr�mit� de l'�cran indiqu�e (environ 1.5s) puis de revient vers le centre.
\end{enumerate}
\figScaleX{1}{data_Task}{Descriptif de la t�che \co}

Cette t�che � conduit � deux �tudes essentielles:
\begin{itemize}
	\item L'�tude de l'encodage des intentions motrices (cf. \ref{Etude2_encodage}): on �tudiera le d�codage du repos vs pr�paration, repos vs ex�cution et de la pr�paration vs ex�cution
	\item L'�tude du d�codage des directions de mouvement (cf. \ref{Etude3_decodage}) que ce soit pendant la pr�paration ou pendant l'ex�cution motrice.
\end{itemize}

% ********************************************
%         AUTRE DONNEES
% ********************************************
\subsection{Autres donn�es}
\todo[inline,caption={},color=red!20]{
  \begin{itemize}
	\item Occulo: Donn�es occulo mais un seul sujet donc pas super cool. De plus, les r�sultats sont un peu vieux et m�riteraient de se pencher dessus une fois pour toute.
	\item Emotions: Ce serait pas mal que j'ai quelques r�sultats sur les �motions, histoire de montrer aussi un peu la diversit� des donn�es utilis�es.
  \end{itemize}
}

\section{Delayed task: protocole exp�rimental}
okok


         % Donn�es exp�rimentales

%% #############################################################################
%                                   OUVERTURE
% #############################################################################
\chapter{Ouverture}
Nos contributions portent sur : \dots \\*

Le \emph{premier chapitre} expose  la probl�matique de la th�se.

Le \emph{deuxi�me chapitre} pr�sente  en d�tail \dots \\*

etc.

Cette th�se  a fait l'objet de  divers travaux �crits : \dots         % Donn�es exp�rimentales


\adjustmtc

% ==================================================================
% CONTENU G�N�RAL
% \pagestyle{headings}
% \part{�tude 1: niveau de chance et �valuation statistique des r�sultats de classification par apprentissage supervis�}
\label{seuil_chance}

\begin{chapintro}
%  \malettrine{P}{ourquoi}  cette �tude? Quelles questions?
%- Seuil de chance th�orique vs pratique?
%- Impact sur des m�thodes (\cv, \clf)
%- Validation sur des donn�es r�elles (Intra MEG)
Sensibilisation � l'importance du nombre d'essais par exemple
\end{chapintro}

%%% --------------------------
%%% R�sum� de l'article
%%% --------------------------
\section{pr�sentation de l'�tude}
\subsection{Contexte}
\subsection{Probl�matique}
\subsection{R�sultats majeurs}
pourquoi cette �tude? Quelles questions?
- Seuil de chance th�orique vs pratique?
- Impact sur des m�thodes (\cv, \clf)
- Validation sur des donn�es r�elles (Intra MEG)
-  d�di� aux �tudiants
- Fournit une toolbox pour reproduire les r�sultats

%%% --------------------------
%%% Article
%%% --------------------------
\section{article}
\includepdf[pages={2-12}]{Chap1/Combrisson-Jerbi-JNeuroscienceMethods_2015.pdf}

%%% --------------------------
%%% Compl�ments
%%% --------------------------
\section{compl�ments d'�tude}
compl�ments sur les diff�rents types de permutation
\citep{ojala_permutation_2010}













%%% --------------------------
%%% Conclusion de l'article??
%%% --------------------------
%\section*{conclusion du chapitre}
%\addcontentsline{toc}{section}{Conclusion}
%
%Ceci est la conclusion. Personnellement, je n'aime pas que la conclusion 
%soit num�rot�, mais je veux qu'elle apparaisse dans la table des mati�re, d'o� 
%la commande addcontentsline.

% \part{�tude 2: encodage de l'intention et de l'ex�cution motrice}
\label{Etude2_encodage}


\chapter*{Introduction}
blabla Intro

\vspace{1\baselineskip}
Blabla �tude


%%% --------------------------
%%% Article
%%% --------------------------
\includepdf[pages={1-29}]{Chap2/Etude2_Combrisson-etal-Encoding-Intention-Execution.pdf}


% %!TEX root = ../main.tex
\part{�tude 3 : D�codage des directions de mouvement pendant et avant l'ex�cution de mouvement de membres sup�rieurs}
\label{Etude3_decodage}


\chapter*{Introduction}
Mon introduction


%%% --------------------------
%%% Article
%%% --------------------------
\includepdf[pages={1-28}]{Chap3/Combrisson-etal_Decoding-Intention-Execution.pdf}


%%%%%%%%%%%%% %!TEX root = ../main.tex
\part{�tude 4 : Tensorpac, logiciel Python de calcul de Phase-Amplitude Coupling}
\label{Etude4_tensorpac}
\pagestyle{headings}


\chapter*{Introduction}
Le \pacFR (PAC) est un marqueur qui mesure le degr�s de couplage entre la phase d'ondes lentes et l'amplitude d'ondes rapides. L'�valuation d'un couplage se fait de mani�re suivante :
\begin{itemize}    
    \item Extraction de la phase et de l'amplitude en utilisant soit des outils de filtrage suivi de la transform�e d'Hilbert, soit une transformation continue en ondelettes.
    \item Calcul du couplage entre ces deux signaux en utilisant une des m�thodologies existantes \citep{tort_measuring_2010,ozkurt_statistically_2012,canolty_high_2006}...
    \item Le PAC �tant une mesure sensible aux bruits, on construit une distribution de mesure de PAC pouvant arriver par chance.
    \item La v�ritable mesure de PAC est ensuite normalis�e par cette distribution de chance afin de minimiser le bruit.
\end{itemize}
Un nombre cons�quent de m�thodes ont �t� propos�es pour chacune de ces �tapes ce qui complique la comparaison et la reproductibilit�. De plus, toutes les publications introduisant de nouvelles m�thodes les pr�sentent en utilisant des vecteurs et ne fournissent pas l'adaptation matricielle ce qui ne prend pas en compte le format des donn�es (nombre de sujets, d'�lectrodes, d'essais...) et donc n'est pas du tout optimal d'un point de vue temps de calcul.\\
Dans ce contexte, nous avons mis en place une toolbox Python, \textit{Tensorpac}, d�di�e exclusivement au calcul du \pacFR. Dans cette toolbox les m�thodes sont impl�ment�es de fa�on modulaire ce qui signifie que l'utilisateur peut combiner les m�thodes existantes pour chacune des �tapes du calcul du PAC. D'autre part, \textit{Tensorpac} utilise des tenseurs permettant de g�n�raliser le calcul � partir de s�ries temporelles vers des donn�es multi-dimensionnelles. Cette impl�mentation en tenseurs est combin�e � du calcul en parall�le ce qui diminue encore le temps d'ex�cution et facilite l'envoie sur des serveurs de calcul. Ce paquet inclue �galement le calcul de comodulograme (soit en cherchant les couples (phase, amplitude) soit en fixant l'un des deux et en faisant varier la largeur de bande de l'autre), statistiques et la visualisation. Pour finir, \textit{Tensorpac} est distribu� sous une licence BSD et peut �tre t�l�charger sur Github \footnotemark[1] et nous fournissons �galement une documentation d�taill�e \footnotemark[2].

\footnotetext[1]{\url{https://github.com/EtienneCmb/tensorpac}}
\footnotetext[2]{\url{https://etiennecmb.github.io/tensorpac/}}


%%% --------------------------
%%% Article
%%% --------------------------
\includepdf[pages={1-18}]{Chap4/combrisson_2017_tensorpac.pdf}
%%%%%%%%%%%%% \part{�tude 5: d�codage des �motions}
\label{Etude5_emotions}


\chapter*{Introduction}
blabla Intro

\vspace{1\baselineskip}
Blabla �tude


%%% --------------------------
%%% Article
%%% --------------------------
%\includepdf[pages={1-29}]{Chap2/Etude2_Combrisson-etal-Encoding-Intention-Execution.pdf}




% ==================================================================
% CONCLUSION
\makeatletter
\renewcommand*{\toclevel@chapter}{-1}
\part{Discussion, conclusion et perspectives}
% \addcontentsline{toc}{chapter}{Conclusion g�n�rale}
\pagestyle{headings}

\chapter*{Discussion}

% Paragraphe 1-Vue d'ensemble de ce travail de th�se: Il s'est d�clin� en deux composante: Une contribution empirique [articles 2 et 3] et des contributions m�thodologiques (� la fois th�oriques [article 1] mais aussi en temre de d�veloppement logiciel [articles 4-6].

Ce travail de th�se s'articule autour de deux grands axes, (\textit{1}) un apport empirique permettant d'am�liorer notre compr�hension du r�le des composantes d'amplitude, de phase et de couplage phase-amplitude lors d'une t�che motrice dirig�e (articles 2 et 3). Nous avons utilis� des algorithmes de classification pour identifier les marqueurs les plus pertinents, (\textit{2}) un apport m�thodologique que ce soit d'un point de vue th�orique pour mieux comprendre la notion de seuil de chance en \textit{machine-learning} (article 1) et pratique via l'impl�mentation de plusieurs logiciels d'analyse et de visualisation (articles 4-6). \\

% Paragraphe 2-R�sum� des r�sultats les plus saillants obtenus dans les �tudes empiriques (articles 2 et 3)

Les deux articles empiriques partagent une m�me m�thodologie : l'utilisation des algorithmes de \textit{machine-learning} pour identifier les marqueurs de puissance, de phase et de PAC susceptibles de d�coder les �tats moteurs (article 2 cf. \ref{Etude2_encodage}) ou les directions motrices (article 3 cf. \ref{Etude3_decodage}). Cette premi�re �tude a permis de mettre en �vidence une augmentation significative de couplage $\alpha/\gamma$ durant la phase de repos qui s'att�nue ensuite durant l'ex�cution. D'un point de vue d�codage, la composante de phase tr�s basse fr�quence (\textit{VLFC}, $<1.5hz$) est le marqueur ayant montr� les plus forts taux de d�codage � travers toutes les conditions classifi�es (\textit{Repos Vs Pr�p.}, \textit{Repos Vs Exec.}, \textit{Pr�p. Vs Exec.}, \textit{Repos Vs Pr�p. Vs Exec.}). Dans une moindre mesure, le PAC pr�sente lui aussi des d�codages significatifs pour diff�rencier ces �tats moteurs ce qui n'est pas le cas pour diff�rencier les directions (article 3 cf \ref{Etude3_decodage}). En effet, le d�codage \textit{Up Vs Down Vs Left Vs Right} est majoritairement possible via des marqueurs de puissance. Durant l'ex�cution, la puissance des oscillations $\gamma$ permet de clairement diff�rencier les modulations � travers les directions dans les r�gions motrices et pr�-motrices. Plus important encore, le m�me ph�nom�ne a lieu lorsque la puissance dans la bande $\alpha$ dans le cortex pr�-moteur et frontal est utilis�e. Pris ind�pendamment, ces deux marqueurs en utilisation unique permettent d'obtenir des d�codages significatifs ($~70\%$ pour l'ex�cution et $~50\%$ durant la pr�paration). Pour finir, nous avons combin� plusieurs algorithmes de \textit{feature selection} ce qui a permis d'atteindre des d�codages proches de $90\%$ pour diff�rencier les quatre directions � la fois durant l'ex�cution et la pr�paration motrice. \\

% Paragraphe 3-Mettre en valeur la nouveaut�/originalit� de ces r�sultats (� quel point c'est diff�rent de ce qu'avait fait les autres chercheurs jusqu'ici?)

PARAGRAPHE DE CE QUI EST VRAIMENT NOUVEAU. \\

% Paragrpahe 4-Regard critique sur les 2 �tudes en intra: Les limitations de ces donn�es (nombre de patients, le fait qu'il s'agit de patient, le fait que les electrodes se trouvent � des endroits diff�rents � travers les patients, le fait que le d�lain'�tait pas variable, le fait qu'on avait pas acc�s syst�matique � l'onset du mouvement, etc etc).

Ces deux �tudes ont un certain nombre de limitations. Tout d'abord, les donn�es intracr�niennes �tant relativement rares, nous n'avons acc�s qu'aux donn�es de six sujets, chacun ayant approximativement une centaine d'�lectrodes. Au total, ces 6OO points d'enregistrement offrent une couverture partielle des r�gions c�r�brales. De plus, ces sujets souffrent d'une forme d'�pilepsie pharmacor�sistante ce qui limite la possibilit� de g�n�raliser � des sujets sains. L'implantation des sites intracr�niens �tant li�e � la localisation du foyer �pileptog�ne, il en r�sulte une implantation propre � chaque sujet et donc, d'une part il est tr�s difficile de pouvoir g�n�raliser le comportement d'un site en particulier puisqu'il est possible qu'il soit le seul dans cette r�gion et d'autre part, l'implantation n'�tant pas sym�trique elle ne permet pas d'�tudier l'effet controlat�ral et ipsilat�ral. La t�che utilis� pour mener � bien ces deux �tudes a �galement des limitations. Tout d'abord, tout les \textit{timing} sont fixes et donc, en l'absence de d�lais variables, il est tout � fait envisageable que les sujets puissent anticiper le d�but d'une phase au fur et � mesure que la t�che avance. De plus, le design de la t�che ne permet pas de conclure clairement sur le type de processus d�cod�s. En effet, on est en droit de se demander si l'on d�code une v�ritable pr�paration/ex�cution de mouvements ou un processus attentionnel, visuomoteur, s�lection spatiale ou d'imagerie motrice. Pour finir, apr�s l�apparition d'un signal visuel, les donn�es ne permettaient pas d'avoir acc�s au d�but du mouvement. Ainsi, l'activit� neuronale contenue dans les tanches temporelles qui suivent directement l'apparition du signal visuel peut contenir des �v�nements diff�rents d� � la variabilit� du temps de r�action d'un essais � l'autre. Les outils de \textit{machine-learning} �tant justement entra�n�s � travers les essais, l'alignement sur le \textit{go-signal} pourrait alors d�grader la performance de classification. \\

% Paragraphe 5-Pour quoi est-ce que malgr�s ces limitations nous avons confiance dans les r�sultats obtenus?

Pour pouvoir g�n�raliser les �tudes 2 et 3 aux sujets sains, les �lectrodes proches des yeux ou contenant une activit� �pileptiforme sont syst�matiquement rejet�es. De plus, les analyses sont appliqu�es � travers les essais ce qui permet de minimiser l'activit� c�r�brale qui n'est pas directement reli�e � la t�che. M�me si les six sujets offrent une couverture partielle, ils ont �t� s�lectionn�s pour leur implantation pr�-frontale, frontale et pari�tale \cad l� o� l'on attend les meilleurs r�sultats de d�codage d'une t�che motrice. L'�tude de la lat�ralit� est une limitation propre � tout les enregistrements invasifs et devrait donc �tre trait� avec des enregistrements d'EEG de scalp ou de MEG. L'impact du d�lai fixe et de l'alignement sur le \textit{go-signal} de la t�che ont �t� minimis�s de fa�on diff�rentes dans chacune de ces deux �tudes. Dans la premi�re (cf. \ref{Etude2_encodage}) la fen�tre de d�codage consid�r�e pour diff�rencier les �tats moteurs se situent 500ms apr�s le \textit{go-signal} et donc, devrait limiter l'utilisation d'une activit� neuronale non-pertinente. Dans le second article (cf. \ref{Etude3_decodage}, le d�codage des directions se fait dans un esprit \textit{temps-r�el}. Un classifieur est syst�matiquement r�-entra�n� puis test� dans les diff�rentes fen�tres temporelles d�finies (67 au total), du d�but du repos � la fin de l'ex�cution. En cons�quence, seule 2/67 fen�tres qui suivent les deux indications visuelles peuvent potentiellement montrer des d�codages plus bas. \\

% Paragraphe 6-R�sumer l'apport de d�veloppement logiciel (re-mentionner les packages, et le nombre de lignes de code par package). Indiquer qu'ils peuvent �tre utilis�s pour des applications diverses et vari�es au-del� de ce pourquoi ces outils ont �t� d�velopp�s au d�part (les donn�es motrices en intra etc.). Indiquer en quelques lignes des d�veloppments futurs pr�vus pour certains de ces outils et le d�sir d'encourager d'autres personnes � contribuer activement au d�veloppment de ces outils das un esprit de partage/communautaire etc  (ce ne sont que des id�es bien sur)

Pour finir, tout les r�sultats d'analyses et de visualisation de cette th�se sont issus des logiciels Python open-source d�velopp�s � cet effet. \textit{Brainpipe} ($7899$ lignes) permet d'extraire des marqueurs de l'activit� neuronale (tel que l'amplitude, puissance, phase \dots) et de les classifier en utilisant \textit{Scikit-Learn}. \textit{Tensorpac} ($2230$ lignes), autre paquet Python open-source, permet de calculer le \pacFR avec une impl�mentation modulaire des m�thodes le tout reposant sur des calculs bas�s sur des tenseurs et mise en parall�le. Pour finir \textit{Visbrain} ($30771$ lignes) permet d'obtenir de visualiser de nombreux types de donn�es (pour du \textit{data-mining}, avec cerveau MNI semi-transparent, donn�es de sommeil \dots). Ces trois principaux paquets ont d'abord �t� d�velopp�s pour couvrir nos besoins. Puis, dans un d�sir de partager nos outils, ils ont �t� enti�rement reformat�s pour �tre utilisable par d'autre. \textit{Brainpipe} �tant le premier paquet d�velopp�, pourrait �tre tr�s largement am�lior� (surtout dans la gestion des dimensions des donn�es et de la m�moire RAM). \textit{Tensorpac} et \textit{Visbrain} n'ont pas ce soucis puisqu'ils ont �t� entam�s apr�s une connaissance plus mature du langage Python. Toutefois, de nombreuses autres m�thodologies pourraient �tre ajout�es � \textit{Tensorpac}, certaines �tant plus dur que d'autres � �tre impl�ment�es en tenseurs. Pour finir, \textit{Visbrain} � une longue feuille de route de pr�vue pour son d�veloppement futur, incluant de nombreuses refontes, am�liorations et surtout l'ajout de nouveaux modules de visualisations. Quoi qu'il en soit, avec leur qualit�s et leurs d�fauts, ces trois logiciels viennent avec un code extr�mement comment�, des datasets et exemples et des documentations. Mon plus grand souhait �tant que d'autres personnes participent � faire avancer ces projets tout en conservant l'id�e principale, un partage libre pour une science ouverte. 

\chapter*{Conclusion et perspectives futurs}

[1/2 page suffira]

2 id�es � d�velopper

(i) Un truc du genre "Taken together, this body of research advances our understanding of the role of oscillatory power, phase and PAC in goal-directed behavior, and provides efficient open-source packages for the scientific community to replicate and extend this research."

(ii) Proposer des pistes prometteuses pour l'avenir: Par exemple le fait de s'orienter d�finitivement vers des �tudes du syst�me moteur dans de grandes cohortes de sujets (via le data sharing et �tude multi-centriques) et via le open-science et le partage des outils d'analyses pou rrenforcer la reproductiblit� des r�sultats etc.


% ==================================================================
% ANNEXES
\let\cleardoublepage\clearpage
{
\hypersetup{linkcolor=black}
\appendix
\renewcommand*{\toclevel@section}{0}
\todo[color=red!40]{Mettre juste la page de références de brainpipe}

% *****************************************************************
%                   COMPARATIF PAC
% *****************************************************************
\section{Comparatif de methodes PAC \citep{tort_measuring_2010}}
\begin{figure}[H]
	\centering
	\includegraphics[scale=0.4,angle=-90]{./figures/PAC_methods_comparison}
	\caption{Comparatif de methodes PAC \citep{tort_measuring_2010}}
	\label{comp_pac}
\end{figure}

% *****************************************************************
%                   CLASSIFICATION PIPELINE
% *****************************************************************
\section{Pipeline standard de classification}
\begin{figure}[H]
	\centering
	\makebox[\textwidth][c]{\includegraphics[scale=0.9]{./figures/clf_pipeline}}
	\caption{Pipeline standard de classification}
	\label{clf_pip}
\end{figure}

% *****************************************************************
%                   COMPARATIF CLASSIFIEURS
% *****************************************************************
\section{Comparatif de classifieurs \citep{scikit-learn}}
\begin{figure}[H]
	\centering
	\makebox[\textwidth][c]{\includegraphics[scale=0.35,angle=-90]{./figures/classifier_comp}}
	\caption{Comparatif de classifieurs \citep{scikit-learn}}
	\label{comp_clf}
\end{figure}


% *****************************************************************
%                   SCHEMA IMPLANTATION
% *****************************************************************
\section{Exemple de sch\'{e}ma d'implantation}
\begin{figure}[H]
	\centering
	\makebox[\textwidth][c]{\includegraphics[scale=1]{./figures/schema-implantation_annexe}}
	\caption{Exemple de sch\'{e}ma d'implantation}
	\label{schema_implantation}
\end{figure}

}
 
% ==================================================================
% BIBLIOGRAPHIE
\makeatletter
\renewcommand*{\toclevel@chapter}{-1}
\backmatter
\bibliographystyle{apalike}
\bibliography{Manuscrit}
 
% ==================================================================
% COLOPHON
%\colophon{Ce document a �t� pr�par� � l'aide de l'�diteur de texte GNU
%  Emacs et du logiciel de composition typographique \LaTeXe.}


\end{document}
%%% Local Variables:
%%% mode: latex
%%% TeX-master: t
%%% End:
