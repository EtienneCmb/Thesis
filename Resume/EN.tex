\abstractFR{La puissance des oscillations neuronales joue un r�le important dans l'ex�cution motrice. Toutefois, le r�le sp�cifique des composantes de phase, d'amplitude ou de couplage phase-amplitude (PAC) durant la pr�paration et l'ex�cution motrice dirig�e est encore partiellement compris. La premi�re partie de cette th�se traite de cette question en analysant des donn�es d'EEG intracr�nien chez des sujets �pileptiques effectuant une t�che \textit{center-out} diff�r�e. Les outils d'apprentissage machine ont permis d'identifier des marqueurs neuronaux propres aux �tats moteur ou aux direction de mouvement. En plus du r�le d�j� bien connu de la puissance, cette approche dict�e par les donn�es a permis de confirmer l'implication de la composante de phase basse fr�quence ainsi que du PAC dans les fondements neuronaux de la pr�paration et de l'ex�cution motrice. En plus de cet apport empirique, une importante partie de ce travail de th�se a consister � impl�menter des outils d'analyse et de visualisation de donn�es �lectrophysiologiques. Ceci inclue une toolbox d�di�e l'extraction et classification de marqueurs neuronaux (\textit{Brainpipe}), des outils de calcul de PAC modulaire bas� des tenseurs (\textit{Tensorpac}) ainsi qu'un ensemble d'interfaces graphiques d�di�es � la visualisation de donn�es c�r�brales (\textit{Visbrain}). Ces recherches auront permis de mieux comprendre le r�le des oscillations neuronales lors de comportements dirig�s et met �galement � disposition un ensemble d'outils efficaces et libres permettant � la communaut� scientifique de r�pliquer et d'�tendre ces recherches.}

\abstractEN{Brain oscillation power plays an important role in action execution. However, the specific role of oscillatory phase, amplitude and phase-amplitude coupling (PAC) across the planning and execution stages of goal-directed motor behavior is still not well-understood. The aim of the first part of this PhD thesis was to address this question by analyzing intracranial EEG data in epilepsy patients performing a delayed center-out task. Using machine learning, we identified functionally relevant oscillatory features via their accuracy in predicting motor states and movement directions. In addition to the established role of oscillatory power, our data-driven approach revealed the prominent role of low-frequency phase as well as significant involvement of PAC in the neuronal underpinnings of motor planning and execution. In parallel to this empirical research, an important portion of this PhD work was dedicated to the development of efficient tools to analyze and visualize electrophysiological brain data. These packages include a feature extraction and classification toolbox (\textit{Brainpipe}), modular and tensor-based PAC computation tools (\textit{Tensorpac}) and a versatile brain data visualization GUI (\textit{Visbrain}). Taken together, this body of research advances our understanding of the role of brain oscillations in goal-directed behavior, and provides efficient open-source packages for the scientific community to replicate and extend this research.}

\keywordsFR{cerveau, oscillation, apprentissage machine, logiciels, python}

\keywordsEN{brain, oscillations, machine-learning, packages, python}
