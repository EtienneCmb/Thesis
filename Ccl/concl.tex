\chapter*{Chapitre VIII : Discussion}
% \addcontentsline{toc}{chapter}{Conclusion g�n�rale}

Paragraphe 1-Vue d'ensemble de ce travail de th�se: Il s'est d�clin� en deux composante: Une contribution empirique [articles 2 et 3] et des contributions m�thodologiques (� la fois th�oriques [article 1] mais aussi en temre de d�veloppement logiciel [articles 4-6].

Paragraphe 2-R�sum� des r�sultats les plus saillants obtenus dans les �tudes empiriques (articles 2 et 3)

Paragraphe 3-Mettre en valeur la nouveaut�/originalit� de ces r�sultats (� quel point c'est diff�rent de ce qu'avait fait les autres chercheurs jusqu'ici?)

Paragrpahe 4-Regard critique sur les 2 �tudes en intra: Les limitations de ces donn�es (nombre de patients, le fait qu'il s'agit de patient, le fait que les electrodes se trouvent � des endroits diff�rents � travers les patients, le fait que le d�lain'�tait pas variable, le fait qu'on avait pas acc�s syst�matique � l'onset du mouvement, etc etc).

Paragraphe 5-Pour quoi est-ce que malgr�s ces limitations nous avons confiance dans les r�sultats obtenus?

Paragraphe 6-R�sumer l'apport de d�veloppement logiciel (re-mentionner les packages, et le nombre de lignes de code par package). Indiquer qu'ils peuvent �tre utilis�s pour des applications diverses et vari�es au-del� de ce pourquoi ces outils ont �t� d�velopp�s au d�part (les donn�es motrices en intra etc.). Indiquer en quelques lignes des d�veloppments futurs pr�vus pour certains de ces outils et le d�sir d'encourager d'autres personnes � contribuer activement au d�veloppment de ces outils das un esprit de partage/communautaire etc  (ce ne sont que des id�es bien sur)

\chapter*{Chapitre IX : Conclusion et perspectives futurs}

oki c'est finit
