\part{�tude 1: niveau de chance et �valuation statistique des r�sultats de classification par apprentissage supervis�}
\label{seuil_chance}

\begin{chapintro}
%  \malettrine{P}{ourquoi}  cette �tude? Quelles questions?
%- Seuil de chance th�orique vs pratique?
%- Impact sur des m�thodes (\cv, \clf)
%- Validation sur des donn�es r�elles (Intra MEG)
Sensibilisation � l'importance du nombre d'essais par exemple
\end{chapintro}

%%% --------------------------
%%% R�sum� de l'article
%%% --------------------------
\section{pr�sentation de l'�tude}
\subsection{Contexte}
\subsection{Probl�matique}
\subsection{R�sultats majeurs}
pourquoi cette �tude? Quelles questions?
- Seuil de chance th�orique vs pratique?
- Impact sur des m�thodes (\cv, \clf)
- Validation sur des donn�es r�elles (Intra MEG)
-  d�di� aux �tudiants
- Fournit une toolbox pour reproduire les r�sultats

%%% --------------------------
%%% Article
%%% --------------------------
\section{article}
%\includepdf[pages={2-12}]{Chap1/Etude1_Combrisson-Jerbi-JNeuroscienceMethods_2015.pdf}

%%% --------------------------
%%% Compl�ments
%%% --------------------------
\section{compl�ments d'�tude}
compl�ments sur les diff�rents types de permutation
\citep{ojala_permutation_2010}













%%% --------------------------
%%% Conclusion de l'article??
%%% --------------------------
%\section*{conclusion du chapitre}
%\addcontentsline{toc}{section}{Conclusion}
%
%Ceci est la conclusion. Personnellement, je n'aime pas que la conclusion 
%soit num�rot�, mais je veux qu'elle apparaisse dans la table des mati�re, d'o� 
%la commande addcontentsline.
